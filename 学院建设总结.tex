\message{ !name(总结报告.tex)}
\message{ !name(总结报告) !offset(-2) }
                           




山东省高等教育名校建设工程
技能型人才培养特色名校建设项目
总结报告









泰安市人民政府
泰山职业技术学院

二〇一六年五月

目   录
前  言	1
第一部分 项目建设基本情况	5
一、凝聚集体智慧,做好顶层设计	5
(一)总体目标	7
(二)具体目标	7
1.围绕“服务”价值取向,彰显办学特色	7
2.培育优势特色专业,带动专业群整体发展	8
3.加强师资队伍建设,建设专兼结合的高水平师资队伍	9
4.深化教学改革,提高人才培养质量	9
5.加强体制机制建设,增强办学活力	9
6.拓展服务项目,提升社会服务能力	10
7.加强招生就业工作,提高生源和就业质量	10
(三)建设内容	10
二、坚持多元合作,协同推进名校建设	11
(一)政府倾力支持,名校建设承诺全部兑现	11
1.高度重视,成立组织机构	11
2.政策支持,优化发展环境	12
3.投入资金,认真履行承诺	13
(二)行业进行全方位指导,搭建校企合作桥梁	13
(三)企业积极参与,互惠互利共建名校	14
(四)完善第三方评价机制,实现多维共评共管	15
三、实施项目管理,确保建设规范有序	16
(一)成立机构,加强组织领导	16
(二)统一思想,提升办学理念	18
(三)实施项目管理,明确建设责任	20
(四)强化督导考核,确保建设成效	23
四、严格预算管理,资金使用规范	26
第二部分 建设目标完成情况和建设成效	28
一、建设项目指标完成情况	28
(一)项目建设任务全面完成	28
(二)各项建设任务主要指标完成情况	29
1.办学特色	29
2.专业建设	30
3.师资队伍建设	30
4.教学改革	31
5.体制机制建设	32
6.社会服务能力建设	34
7.招生就业	35
8.国际化办学	35
二、项目综合建设层面成效显著	36
(一)顶层设计科学合理,办学特色更加突出	36
1.顶层设计清晰,凸显“服务”价值取向办学理念	36
案例1:立足服务三农,助力精准扶贫	38
2.实施“多方联动、协同发展”多元化办学模式,办学活力增强	39
案例2:泰安市机电职教集团成立	40
案例3:学院与泰安市高新技术开发区实现区校合作	40
3.完善“德技并重、理实一体”的人才培养模式,扎实推进有效教学改革	41
案例4:旅游管理专业推行“旺入淡出、产学结合”人才培养模式	42
案例5:实施有效教学改革,改变课堂生态	42
4.完善“四位一体”泰山特色校园文化,学生职业精神培养成效显著	43
案例6:泰山文化创意产品设计与开发--创新传承泰山文化途径	46
5.其它特色与创新工作	48
(二)实施“优势特色专业培育工程”,专业服务产业能力显著提升	50
1.建立专业内部有效管理机制	50
2.形成常态化的专业调研和岗位分析机制	51
3.调整优化专业结构与布局	52
案例7:构建现代职教体系,促进中职、高职、本科上下贯通	54
4.制订专业教学标准,建立专业评估机制	55
5.实施优势特色专业培育工程	55
6.实施“实训实习基地内涵提升工程”,优化校内外实训基地建设	57
7.建成资源丰富、共建共享的专业教学资源库	59
案例8: 建设专业教学资源库,推进“互联网+教育”进程	60
8.重点建设专业辐射带动作用明显,专业实力明显提升	61
案例9:以点带面,机电一体化专业带动专业群共同发展	62
(三)实施“名师递进培养工程”,促进教师团队专业化成长	63
1.优化师资队伍结构,形成专兼结合双师型教师队伍	63
2.重视专业带头人与骨干教师培养,培育专业领军人物	64
3.加强师资培训,教师素质全面提升	65
4.加强兼职教师队伍建设,提高兼职教师教学能力	68
案例10:实施“顶天立地”教师培训计划,助教师专业化成长	69
5.健全师德建设长效机制,提高师德水平	69
(四)推进有效教学改革,人才培养成效突出	70
1. 系统优化人才培养方案	71
2.创新“德技并重、理实一体”人才培养模式	71
3.实施“课程体系优化工程”	72
4.以项目为引领,深入推进有效教学改革	73
案例11: 院领导“推门听课”,扎实推进教学做一体化教学模式改革	77
案例12: 师生共建C.U服装工作室	78
案例13 :打造信息化智慧教室,构建数字化智慧教室	79
5.实施“分阶段、项目化、协同式”实践教学模式	79
6.实行信息化管理,强化顶岗实习	80
7.加强大学生创新创业能力培养	81
8.人才培养质量全面提高	83
(五)创新体制机制,实现共建共管共赢	84
1. 构建了多元化办学体制	85
案例14:首饰设计与工艺项目现代学徒制试点	89
案例15:校区战略合作模式	91
案例16:县校战略合作模式	92
2. 合理构架内部管理体制和运行机制	92
3.建立健全教学质量监控与保障体系	94
(六)主动融入地方经济发展,社会服务能力显著增强	97
1. 创新科研管理机制,突出服务社会导向	97
2.搭建服务平台,全面提升教师服务能力	98
3.开展技术培训及服务,促进地方经济社会发展	100
4.承担社会考试,服务行业发展	102
5.参与社会活动,提升社区文化水平	102
6.积极开展志愿者服务活动	103
7.对口支援中西部学校,服务国家发展战略	103
(七)加强就业创业指导,招生就业工作协调发展	104
1.拓宽招生渠道,生源质量大幅提升	104
2. 全方位服务就业,就业质量不断提高	105
案例17:麦可思公司助力学院诊断就业质量	106
(八)大力推进国际化办学,国际合作领域逐步拓宽	106
(九)加强国内外同行交流,业界影响力不断扩大	109
三、重点建设专业及专业群成效	111
项目一  机电一体化技术专业建设	111
项目二 建筑工程技术专业建设	122
项目三  会计电算化专业建设	131
项目四  旅游管理专业建设	144
项目五  计算机应用技术专业建设	157
项目六  园艺技术专业建设	169
项目七 汽车电子技术专业建设	180
项目八  服装设计专业建设	195
项目九 珠宝首饰工艺及鉴定专业建设	207
四、特色项目建设成效	221
项目十    校企合作体制机制建设	221
项目十一  教学质量监控与保障体系建设	235
项目十二  泰山特色校园文化建设	249
项目十三  数字化校园建设	260
第三部分   项目建设质量与效益	268
一、学院办学综合实力大幅提升	270
(一)学院办学特色更加鲜明	270
(二)专业建设成效明显	275
(三)师资队伍整体水平显著提升	276
(四)教学改革成效突出	278
(五)体制机制进一步完善	280
(六)招生就业更加畅通	283
(七)国际交流与合作成效初显	285
二、人才培养效益明显提高	287
(一)毕业生职业素养明显提高	287
(二)毕业生核心知识水平明显提高	288
(三)毕业生专业技术能力明显提高	289
三、社会服务能力显著增强	290
(一)为地方经济建设提供高素质人才支持	290
(二)专业紧密对接地方优势产业	291
(三)提供多元化社会培训与服务	291
四、示范带动能力进一步增强	292
(一)以点带面,带动其它专业共同发展	293
(二)对口支援,成果辐射其它职业院校	293
(三)承办大赛,搭建技能交流平台	294
第四部分 专项资金使用与管理情况	296
一、专项资金到位情况	296
二、专项资金的执行情况	297
(一)专项资金的执行情况	297
(二)子项目资金使用情况	298
三、专项资金管理情况	311
(一)健全机构,细化责任	311
(二)建章立制、规范程序	312
(三)指导执行,过程监控	312
四、专项资金使用效果	313
(一)省财政专项资金引导带动作用明显	313
(二)学院办学实力进一步提升	313
第五部分 思考与展望	315
一、突出办学导向,服务国家(区域)战略和创新人才培养	316
二、深化产教融合,加快推进校企一体、协同育人	316
三、全力打造品牌,进一步提高专业服务产业能力	317
四、强化立德树人,切实提升学生发展核心竞争力	317

前  言
泰山职业技术学院是由泰安市人民政府举办,山东省人民政府批准,教育部备案的全日制普通高等院校。学院已有60年职业教育办学历史。2002年起,由原山东省泰安农业学校、泰安机械电子工程学校和泰安贸易学校三校合并组建。
学院现占地982亩,建筑面积20万平方米;有国家级实训基地3个,实验实训室164个,教学仪器设备总值7907万元;馆藏图书70.4万册;固定资产达4.03亿。32个高职专业中,中央财政重点支持专业、省名校重点建设专业、省特色专业12个;省级精品课程17门。有国家发明专利58项,省级教改课题15项,市级以上教科研课题158项,发表论文772篇。省级以上教学、科研成果奖52项,其中,国家级教学成果二等奖、省级教学成果一等奖、省高校优秀科研成果一等奖多项。2013年以来,获省级及以上技能大赛一等奖57项,国家发明专利9项,省级教改课题7项,市级以上教科研课题90项,发表论文282篇。
现有教职工477人,其中,专任教师354人,副高以上职称112人,“双师型”教师245人,省级教学团队2个,全国人大代表、山东省十佳杰出青年、省级教学名师、省先进教育工作者、省优秀教师等14人,市级拔尖人才、功勋教师、首席技师、优秀教师、先进工作者等34人。全日制高职在校生8763人。
学院在教育部人才培养工作水平评估、省高校德育工作评估、省高校校园文明建设评估等评估中,均获“优秀”等级。获得全国职业教育先进单位、全国职业院校魅力校园、全国职业院校就业竞争力示范校、全国教学工作诊断与改正试点学校、山东省文明单位、山东省高校就业工作先进集体等省级以上荣誉31项,获得泰安市综合实力最强职业院校、泰安市改革开放三十周年杰出成就奖、泰安市科学发展综合考核先进单位、泰安市“五星级”党组织等市级荣誉17项。
2013年11月,学院被省教育厅、财政厅确定为技能型特色名校建设单位。为扎实推进名校建设工程,真正发挥特色名校的示范带动作用,根据山东省教育厅、山东省财政厅《关于山东省高等教育名校建设工程实施意见》(鲁教高字〔2011〕14号)《关于加强山东省高等教育名校建设工程管理工作的通知》(鲁教高字〔2014〕4号)和《山东省教育厅 财政厅关于做好山东省第二批技能型人才培养特色名校建设项目验收工作的通知》(鲁教职字〔2016〕号)要求,泰安市人民政府成立了泰山职业技术学院山东省高等教育名校工作领导小组,学院成立了名校建设工作小组,多次召开名校建设工作会议,制定了《泰山职业技术学院省级特色名校建设项目管理办法》《泰山职业技术学院省级特色名校建设项目专项资金管理办法》《 泰山职业技术学院名校建设项目绩效考核办法》等规章制度,扎实推进名校建设。
2015年4月,省教育厅专家组对我院名校建设工作进行中期验收检查,并给予了
高度评价,同时也指出了项目建设过程中存在的问题。学院根据专家组反馈意见,
针对存在的问题,积极整改,强化了内部管理,加大了激励考核,充分调动了全
体师生员工的积极性,提高了名校建设的质量和效率。
名校建设不仅得到了泰安市政府的大力支持,还得到了行业企业、科研院所、兄弟院校以及业内专家的悉心指导和热情帮助,是多元参与、协同推进、共建共赢的成果。经过三年建设,学院全体师生勇于担当,乐于奉献,按照建设方案和任务书约定内容和要求,圆满完成了各项建设任务,取得了国家级教学成果二等奖(省级教学成果一等奖)、省级教学团队等一系列标志性成果。学院办学条件明显改善,办学实力、人才培养质量、管理水平、办学效益和社会服务能力均显著提升,形成了可持续发展的良好运行机制,辐射带动其它职业院校改革和发展,成为特色鲜明、示范带动作用显著的高职院校。 
名校建设验收后,学院将继续以“中国制造2025” “互联网+”“大众创业、万众创
新”和精准扶贫等国家战略实施为契机,以《国务院关于加快发展现代职业教育
的决定》(国发〔2014〕19号)《教育部关于深化职业教育教学改革 全面提高
人才培养质量的若干意见》(教职成〔2015〕6号)《高等职业教育创新发展行
动计划》(教职成〔2015〕9号)《职业院校管理水平提升行动计划(2015-2018
年)》等文件精神为指导,立足经济新常态下高职发展新起点,聚焦高职发展新
问题,以学院“十三五”发展规划为蓝图,以立德树人为根本,以服务发展为宗旨,
以诊改工作为指导,以增强学生就业创业能力为核心,坚持走内涵式发展道路,
坚持完善产教融合、协同育人机制,落实精细化管理及培养的先进工作方式,突出办学特色,巩固扩大建设成果,发挥名校示范引领作用,争创优质专科学校,为打造经济文化强市、生态宜居城市和国内外著名旅游目的地城市,为推进富民强市、建设幸福泰安贡献力量。
第一部分 项目建设基本情况
一、凝聚集体智慧,做好顶层设计
学院紧紧抓住山东省高等教育名校建设重要发展机遇,围绕人才培养关键性要素,集名家之智慧,汇名校之经验,广泛听取教职工和校友意见,立足学院实际,围绕人才培养等关键性要素,从全局视角和战略高度对办学理念、办学定位和办学特色等进行系统梳理设计。名校立项建设以来,先后邀请姜大源、马树超、匡奕珍、史忠健、王汉忠、申培轩等省内外35名职业教育专家对学院发展规划及名校建设方案进行论证;教师团队先后到深圳职业技术学院、宁波职业技术学院、山东商业职业技术学院等20多所国家示范、骨干院校学习建设经验;在全院组织开展“我的泰职梦”—名校建设大讨论活动;举办“学术校庆、人文泰职、魅力名校”60周年校庆系列活动,邀请行业企业专家、校友共议名校建设大计,不断优化建设方案和任务书,明确建设目标和任务。建设期内,修订完成学院《章程》,制定了学院“十三五”发展规划以及15个专项发展规划,明确了以顶层设计引领名校建设的基本路径。
在结合学院实际、凝聚各方智慧的基础上,最终形成学院名校建设的顶层设计,
明确学院各项工作突出“服务”价值取向,不断强化和提升学院服务区域经济社会
发展、服务学生成人成才、服务教师专业成长、服务文化传承创新的意识和能力,
最终实现提升学生服务社会的意识和能力,形成了突出“服务”价值取向的名校建
设理念;其基本价值观:服务是责任,服务是修养,服务创造价值,实践改变命
运。其基本要求是:服务意识强、服务能力高、服务品质优;最后达到的目标:
通过服务区域经济社会发展,把学院建成泰安市的人才库、智囊团;通过服务学
生成长成才,把学生培养成为德高技高“双高型”技术技能人才,通过服务教师专
业成长,打造一支“四有型”教师团队,通过服务文化传承创新,建设“四位一体”
泰山特色校园文化。
图1―1建设目标和内容结构示意图
(一)总体目标
突出“服务”价值取向,提升学院“服务区域经济社会发展、服务学生成人成才、服务教师专业成长、服务文化传承创新”的能力和水平,最终提升学生服务社会的意识和能力。通过实施“优势特色专业培育工程” “课程体系优化工程” “名师递进培养工程” “实训实习基地内涵提升工程”等四项工程,实现内涵发展;通过实施创新校企合作体制机制、泰山特色校园文化建设、教学质量监控与保障体系建设、数字化校园建设等四个项目,实现特色发展;通过搭建素质教育、卓越技师培养、就业服务等三个平台,实现协调发展;通过建设大学生科技创新创业、社会培训与职业技能鉴定、技术研发与服务等三个基地,实现全面发展(以上简称“4433”名校建设工程)。通过“4433”名校建设工程,深化校企合作办学模式,创新“德技并重、理实一体”的人才培养模式,示范带动泰安中高职教育发展,到2015年末,全面建成突出“服务”价值取向的山东省技能型人才培养特色名校。  
(二)具体目标
1.围绕“服务”价值取向,彰显办学特色
学院各项工作突出“服务”思想,不断强化和提升学院服务区域经济社会发展、服务学生成人成才、服务教师专业成长、服务文化传承创新的意识和能力,最终实现提升学生服务社会的意识和能力,形成突出“服务”价值取向的办学理念。                      
健全政、行、企、所、校联合办学体制机制,加强与中职、本科院校合作办学力度,拓展国际化办学空间,形成“多方联动,协同发展”多元化办学特色。
秉承“进取、担当、包容、和谐”的泰山精神,立足泰安、面向全省、放眼全国,突出“德育为首、能力为本”的教学理念,优化“职业素养与专业技能”的教学内容,实现“理论与实践交融”的教学方式,创新完善“德技并重、理实一体”的人才培养模式,培养德高技高的技术技能型人才。
围绕育人中心任务,以提高大学生综合素质和发展潜力为目标,深入发掘泰安历史文化和产业要素,优化校园文化与行业、企业文化交互平台,增进校园文化与经济社会的融合,充实发展融传统文化、泰山文化、企业文化、大学文化“四位一体”的泰山特色校园文化。
2.培育优势特色专业,带动专业群整体发展
围绕省会城市群经济圈和泰安经济社会发展需要,建立专业调整优化机制,优化专业布局结构,招生专业数稳定在30个左右。重点建设旅游管理、园艺技术、建筑工程技术、机电一体化技术等9个专业,带动8个专业群全面发展。
3.加强师资队伍建设,建设专兼结合的高水平师资队伍
建设期内,“双师素质”教师达到221人,占专任教师的比例达到90%以上;培养1名省级教学名师或1个省级教学团队、2名市级教学名师,每个专业重点培养1名专业带头人;递进培养骨干教师100人。建立不少于300人的兼职教师资源库,兼职教师承担专业课时比例达到50%以上。建成一支名师带动、骨干支撑、专兼结合的高水平师资队伍。
4.深化教学改革,提高人才培养质量
以培养职业能力为主线,构建“平台+模块”的课程体系,9个重点建设专业率先进行工作过程系统化的课程开发;校企合作开发44门优质核心课程,编写35门具有工学结合特色的校本教材;新增25门以上省级精品课程。形成较完善的“德技并重、理实一体”的人才培养模式,9个重点建设专业形成各具特色的人才培养模式。
5.加强体制机制建设,增强办学活力
“政、行、企、所、校”多方联动,协同发展,创新多元化办学模式;深入实施“厂校一体、资源共享”的校企合作模式。进一步深化内部管理体制改革,探索实施现代大学制度建设,完善院系两级管理体制;完善竞争激励机制,实行全员聘任制;引入企业精细化管理理念,全面推行“7S”管理制度;健全财务后勤保障机制。健全教学质量监控与保障体系,形成充满活力的办学体制机制。
6.拓展服务项目,提升社会服务能力
每年开展各类培训64000人时,职业技能鉴定3000人次,培训中职教师100人次;建成软件开发等3个技术研发与服务中心,提供技术研发和服务300项。实现年收入500万元以上。建成社会培训与职业技能鉴定、技术研发与服务基地。积极参与泰安产业发展规划、职业教育发展规划等重大课题研究和学院周边社区管理,促进区域文化产业建设及泰山文化创新与传播。
7.加强招生就业工作,提高生源和就业质量
年招高职生2500人以上,办学总规模稳定在6000人以上;到2015年,总规模达到8000人。就业率稳定在97%以上,优质就业率稳定在80%以上,山东省人社厅公布的就业率排名位居全省高职高专院校10名左右。
(三)建设内容
学院总体层面建设内容主要包括:办学特色、专业建设、师资队伍建设、教学改革、体制机制建设、社会服务能力建设、招生就业和国际化办学8个方面;具体建设项目共13个。其中省财政重点支持专业建设项目共6个,分别为机电一体化技术、建筑工程技术、会计电算化、旅游管理、计算机应用技术、园艺技术;非省财政重点支持专业共3个,分别为汽车电子技术、服装设计和珠宝首饰工艺与鉴定;以及校企合作体制机制、教学质量监控与保障体系、泰山特色校园文化和数字化校园4个特色建设项目。
二、坚持多元合作,协同推进名校建设
(一)政府倾力支持,名校建设承诺全部兑现
1.高度重视,成立组织机构
泰安市委、市政府高度重视学院名校建设工作。泰安市成立由市长任组长,市委组织部、市编办、发改委、财政局等部门负责人为成员的“泰山职业技术学院山东省高等教育名校建设工程项目领导小组”,出台了《关于加快泰山职业技术学院建设山东省高等教育名校的实施意见》。市委李洪峰书记亲临学院视察名校建设并对教师进行慰问;市政府王云鹏市长专门做出批示,部署推进名校建设各项工作,并指示市政府各部门要通力合作,把学院建成“市内当排头、省内争上游、国内创一流”的高职名校;市人大副主任陈刚、孙运飞、尹衍祥等先后到学院指导工作;市政府副市长徐恩虎、成丽多次召开专题会议,调度推动学院名校建设工作;副市长刘卫东、周桂萍、陈湘安、展宝卫等先后到学院视察工作;市政协白玉翠主席带领市政协委员来学院进行“产教融合,加快名校建设”的专题调研等。 
         
图1-2建设山东省高等教育名校工作领导小组    图1-3市委书记李洪峰视察学院
     
  
图1-4市政府召开建设省特色名校领导小组会议  图1-5泰安市副市长成丽到学院调研
2.政策支持,优化发展环境
市政府把学院名校建设纳入经济社会发展和产业发展规划予以重点支持,出台了一系列加快职业教育改革发展的政策措施。相继出台《泰安市人民政府关于加快发展现代职业教育的意见 》(泰政发〔2014〕29号)《泰安市职业教育发展规划(2014年-2020年)》等一系列文件。2015年,为落实全国和全省职业教育工作会议精神,全市职业教育工作会议进一步明确政府责任,把职业教育纳入政府工作考核目标;加大财政支持,建立职业教育经费正常增长机制,将教育财政经费的增量主要用于职业教育,对就业率高、办学效益显著的职业学校,优先给予项目资金支持。
市政府牵头,由发改委、经信委、国资委、财政局、教育局、科技局、人社局等部门,以及有关行业协会、企业单位和学院共同组成泰安市职教联席会议,发挥在政策出台、资源整合、规划指导、资金筹措、基础建设等方面的决策、咨询、协调、监督和推动作用,扩大社会和行业、企业对院校办学的参与度;指导学院根据地方经济社会发展和产业结构升级要求,与行业、企业共同研究制定院校建设发展规划;推进学院与行业企业在人才培养、产学研结合、毕业生实习就业、科技开发等方面的全方位合作;为加快实施创新驱动发展战略,大力推进“大众创新,万众创业”实施意见,市政府支持泰安市创业大学落户学院,为全市经济社会发展提供创新人才服务;支持学院成立市第一公共实训中心,将其打造成为本地区综合、开放、共享的学生实训中心、技能培训中心、技术研发中心和技术服务中心。这些措施都进一步优化了学院外部发展环境。
3.投入资金,认真履行承诺
学院为市财政全额拨付事业经费单位,生均经费逐年递增,办学经费及市本级“两项附加”部分原有比例按时足额拨付到位。在实训基地、师资队伍、教学资源建设等专项资金投入方面投入也不断加大,全面支持学院基础能力建设。
建设期内,泰安市政府全力支持学院名校建设,按照省市1:3的承诺比例投入3000万元,配套资金全部落实到位,保障了名校建设各项目的顺利开展。
(二)行业进行全方位指导,搭建校企合作桥梁
学院充分利用行业与企业联系紧密优势,在办学机制、人才培养、社会服务、人才需求预测等方面与泰安市行业管理办公室及其行业协会--泰安市机械加工协会、泰安市建筑行业协会、泰安市旅游协会、泰安市冷链物流协会等开展广泛合作,聘请行业人员为学院专业建设指导委员会成员;行业发挥纽带作用,积极推进学院与企业实现校企合作深度合作,推动各专业与泰安市产业、企业、岗位对接,为学院专业设置与调整、课程建设与改革、专业人才培养方案修订等提供咨询指导。学院牵头成立了泰安市工艺美术学会和泰安市玉雕学会,为泰安市工艺美术产业和泰山玉石产业搭建了理论研究、学术交流和技术切磋的平台,形成了产学研协同创新的新机制。
(三)企业积极参与,互惠互利共建名校
学院始终坚持深化校企合作,促进校企一体化的办学方针。企业给予学院名校建设极大的支持,全程参与学院人才培养方案修订、课程开发、校内外实习实训基地建设、骨干教师培训,并捐赠资金及实习实训设备等,投入仪器设备总值1260万元,深化了学院内涵建设,推动了学院快速发展,形成了校企合作人才共育,责任共担,资源共享,共同办学,共同发展的良性机制。三年来,学院共与中科招商投资集团、银通国际集团、华中数控、顺丰快递、泰盈科技公司、山东沃尔重工等知名企业密切合作,设立了南方测绘、泰和、青岛海利尔等17个冠名订单培养班,新建“校中厂”“厂中校”23个,校企共同开发课程40门、教材78种,形成名校共建、人才共育良性机制。
(四)完善第三方评价机制,实现多维共评共管
引入第三方对人才培养质量进行独立、公正、客观的评价已成为促进高职教育适应需求、深化改革、服务社会的重要举措。为切实保障人才培养质量,按照“学院评价为核心、教育主管部门引导、行业企业广泛参与”的原则,以政府(教育主管部门)、行业企业、学校和专业社会评价机构为四个评价主体,完善了第三方评价机制,共同对人才培养质量进行评价。
1.政府评估。教育部、省教育厅以及泰安市政府、教育局定期对学院办学和人才培养工作及专业建设、就业工作等进行评估,学院根据专家组及教育主管部门提出的意见进行认真整改,不断提高教学质量。名校建设期间,学院通过了高等职业院校人才培养工作评估及评估回访、国家二类城市语言文字工作评估、山东省高校军事理论课教学检查评估和高等职业学校提升专业服务产业发展能力项目验收等各种评估验收。
2.行业企业评价。在学院专业建设指导委员会统筹组织下,各专业建设委员会每年聘请行业专家、一线技术骨干、班组长、企业管理人员等对专业建设、人才培养方案、人才培养过程管理、教学资源配置、人才培养质量等进行综合评价分析,学院根据行业企业人员意见,及时调整专业结构和课程设置,优化人才培养方案,进行人才培养模式的改革。行业企业参与人才培养全过程的检测、诊断和评价,使学院进一步紧密对接区域经济发展和产业发展需要,提高了人才培养质量,实现了人才培养与市场零距离对接。
3.引入第三方评价。与第三方权威性数据机构麦可思数据有限公司开展合作,对人才培养质量进行跟踪调查和分析,麦可思公司提供的《泰山职业技术学院应届毕业生社会需求与培养质量跟踪评价报告》成为学院进行专业设置调整和人才培养方案修订的重要依据。
      
图1-6《泰山职业技术学院社会需求与培养质量跟踪评价报告》
三、实施项目管理,确保建设规范有序 
(一)成立机构,加强组织领导
学院健全了名校建设的组织机构,强化了对建设项目的组织保障。一是成立了院长任组长的项目建设工作小组,负责统筹规划和组织学院名校建设工作。二是成立了以副书记、副院长为主任的名校建设办公室。名校办负责项目的管理、实施和协调,下达建设任务,落实建设任务书,对建设项目进行管理和督办。三是成立了以纪委书记为主任的名校建设监控办公室,负责对项目的监控、检查和考核工作。四是设置了体制机制建设、教学改革等10个工作组、泰山特色校园文化建设等4个特色项目组、机电一体化技术等9个重点专业建设项目组,每个项目组都建立党小组,党小组发挥党员领航作用,当标杆,作表率,带领全体教职工积极投身到名校建设中。
   
图1-7名校建设党小组文件
院党政定期召开专题会议,研究导向性政策,听取项目进展汇报,提出针对性、有效性推进措施。形成了市委市政府统一领导,学院党政领导负总责,项目负责人各负其责,各职能部门分工协作,全体教职工人人参与的领导体制和工作机制,为名校建设顺利开展提供了领导和组织保障。
     
图1-8名校建设组织机构图
(二)统一思想,提升办学理念
为提高全院教师的职业教育理念和对职业教育的思想认识,学院以《国家中长期教育改革和发展规划纲要(2010-2020年)》《国务院关于加快发展现代职业教育的决定》《山东省人民政府关于加快建设适应经济社会发展的现代职业教育体系》等系列文件为主要内容,每月召开一次学院领导班子读书学习会,集中学习高职教育新理念,着力解决学院改革发展中存在的突出问题;定期举办中层干部、专业带头人、骨干教师学习研讨班,选派有关部门负责人和教师前往台湾昆山科技大学、深圳职业学院、宁波职业技术学院、山东商业职业技术学院等院校和大中型企业考察学习,同时聘请国内知名专家到学院进行专题讲座和教改辅导。组织了开展全院性的名校建设大讨论,广泛征求全院教职工意见和建议,汇聚集体智慧和力量。召开了教职工代表大会讨论通过了建设方案和任务书。
自名校建设以来,学院先后邀请姜大源、马树超、匡奕珍、史忠健、王汉忠、申培轩等35名国内知名专家来学院做专题报告;邀请15名企业专家进校做专题报告30余场;实施“顶天立地”教师培训计划,组织全体342名骨干教师到全国一流高校和企业培训学习;组织79名骨干教师赴台湾培训、国外2人,组织87名专业负责人、教学骨干教师赴浙江、天津参加专业负责人培训,学习交流职业教育先进的办学理念和经验,通过“走出去、请进来”的方式,自主学习与专家辅导相结合,开阔了高职教育的办学视野,有效提升了全体教职工的教育教学理念。
 
图1-9名校建设动员大会        图1-10专家讲座
表1-1邀请专家讲座情况一览表(部分)
序号
专家
主题
时间
1
王汉忠
新形势下高职教育改革发展与高职院校内涵建设
2013.11.20
2
史忠健
关于当前高职教育的思考
2013.12.12
3
孔宪思
如何搞好技能型名校建设
2013.12.13
4
俞克新
高等职业教育人才培养中的问题与对策
2013.12.17
5
姜大源
工作过程系统化课程开发方法及理论创新
2014.01.08
6
王灵玲
关于校企合作与专业改革和“7S”管理
2014.01.06
7
陈解放
高职专业内涵建设的深化
2014.03.10
8
马树超
高职教育质量观和改革发展策略
2014.03.15
9
李  进
高职教育特色建设的思考
2014.03.27
10
王汉忠
加强教学研究,推动教学改革努力培育高水平教学成果
2014.05.13
11
张  建
大数据时代高职人文课程的开发研究与实践
2014.05.21
12
李  进
高职院校教学评估与质量建设
2014.06.03
13
张  力
高等职业教育改革发展的宏观形势和政策要点
2014.06.17
14
杜德昌
深化职业教育课程改革,提升技术技能人才培养质量
2015.06.17
15
王汉忠
学习新文件,认清新形势,适应新常态,做好新规划,促进新发展
2015.12.02
(三)实施项目管理,明确建设责任
为强化责任意识、深挖内在潜力,切实保证建设质量,自名校批准立项建设以来,经学院名校建设工作委员会研究决定,名校建设采用项目化管理,实行项目负责人负责制。整个名校建设工程项目分为9个重点专业建设项目和4个特色建设项目一级项目体系,每个项目确定一名项目负责人。在名校办指导下,项目负责人将所有建设任务分解为13个一级建设项目,99个二级项目和350个子项目,各项目组再进一步将子项目分解为3341个验收要点。项目负责人严格按照项目任务书中所列的建设进度、建设内容、预期目标及验收要点进行建设,并接受名校办和监控办的监督和检查。
院长与每个项目负责人签订责任书,各项目负责人也分别与子项目负责人层层签订责任书,确保各项任务责任到人,将每个项目落到实处。名校办统一对名校建设组织协调,及时调度解决建设过程中的问题,并总结、交流、推广项目建设先进经验;各项目负责人为该项目的第一责任人,负责本项目的管理、实施和协调建设工作,落实项目任务书,制定项目建设进程表,对项目建设情况定期汇报,并进行自我检查和监控;工作组充分发挥职能作用,负责学院综合项目建设任务,制定项目建设标准,明确建设工作流程,指导推进项目建设,督促项目落实;监控办对建设质量和进度进行督导考核。全院各部门各单位紧紧围绕名校建设目标和任务,根据任务书制定的路线图和时间节点,扎实推进名校建设。
       
图1-11名校建设工作流程图         图1-12院长与项目负责人签订责任书





表1-2建设项目验收要点一览表      单位:个
项目类别
项目名称
二级项目
子项目
验收
要点
综合项目建设
办学特色
1
1
12

专业建设
1
5
47

师资队伍建设
1
5
55

教学改革
1
8
118

体制机制建设
1
5
61

社会服务能力建设
1
4
33

招生就业
2
2
9

国际交流
1
1
12
省财政重点专业建设项目
机电一体化专业建设
8
26
318

建筑工程技术专业建设
8
30
351

会计电算化专业建设
8
28
337

旅游管理专业建设
9
33
334

计算机应用技术专业建设
8
25
327

园艺技术专业建设
7
26
340
非省财政重点专业建设项目
汽车电子技术专业建设
8
35
262

服装设计专业建设
7
29
379

珠宝首饰工艺及鉴定专业建设
8
21
169
特色项目建设
创新校企合作体制机制建设
3
10
42

教学质量监控与保障体系建设
6
23
54

泰山特色校园文化建设
5
25
51

数字化校园建设
5
8
30
合计
13
99
350
3341
(四)强化督导考核,确保建设成效
1. 建立完善管理制度。为提高项目建设质量,确保项目落地,先后制定出台了项目管理办法、专项资金管理办法、绩效考核办法、项目工作组工作职责、职教能力测试、有效课堂教学改革、项目档案管理规定等41项相关文件和制度。实施周计划、周总结和月报告制度,三年中,各项目共制定2236个周计划和周总结,召开月报告会议20次。扎实推进项目建设。
表1-3名校建设部分制度文件一览表
序号
文号
标题
成文时间
1
院字〔2013〕2号
关于调整“省示范性高职院校”工作领导小组的决定
2013.01.24
2
院字〔2013〕17号
学生顶岗实习管理办法
2013.05.16
3
院字〔2013〕32号
关于进一步加强校企合作 开展校内外服务工作的实施办法
2013.07.17
4
院字〔2013〕33号
关于成立泰山职业技术学院旅行社的决定
2013.07.18
5
院字〔2013〕51号
关于成立校企合作办公室的通知
2013.08.10
6
院字〔2013〕58号
关于调整教学督导机构组成人员的决定
2013.09.10
7
院字〔2013〕72号
关于调整学院教学工作委员会的决定
2016.10.11
8
院字〔2013〕77号
关于进一步明确学院办学目标定位与人才培养模式的意见
2013.10.24
9
院字〔2013〕80号
关于调整学术委员会的决定
2013.11.12
10
院字〔2013〕82号
关于调整校企合作工作委员会的通知
2013.11.08
11
院字〔2013〕83号
关于成立国际交流合作委员会的通知
2013.11.08
12
院字〔2013〕84号
关于成立名校建设工作委员会的通知
2013.11.08
13
院字〔2013〕86号
科研工作管理办法
2013.11.22
14
院字〔2013〕87号
科研成果奖励办法
2013.11.22
15
院字〔2013〕88号
科研工作量化考核办法
2013.11.22
16
名校字〔2014〕1号
关于加快泰山职业技术学院建设山东省高等教育名校的实施意见
2014.04.01
17
院字〔2014〕6号
访问工程师项目实施办法
2013.12.31
18
院字〔2014〕19号
省级特色名校建设项目专项资金管理办法
2014.03.05
19
院字〔2014〕20号
省级特色名校建设项目管理办法
2014.03.04
20
院字〔2014〕21号
7S实施方案
2014.03.11
21
院字〔2014〕22号
名校建设项目绩效考核办法
2014.03.17
22
院字〔2014〕42号
关于实施有效课堂教学改革的通知
2014.04.16
23
院字〔2014〕46号
名校建设项目工作组工作职责
2014.05.9
24
院字〔2014〕56号
大学生科技创新行动计划实施方案
2014.06.13
25
院字〔2014〕61号
关于成立教师发展中心的通知
2014.05.18
26
院字〔2014〕62号
专业带头人选拔培养管理办法
2014.06.24
27
院字〔2014〕68号
课程体系优化工程实施方案
2014.07.11
28
院字〔2014〕69号
自编教材(指导书)建设与使用管理办法
2014.07.11
29
院字〔2014〕70号
主要教学环节质量标准
2014.07.11
30
院字〔2014〕72号
外聘兼职教师管理办法
2014.06.13
31
院字〔2014〕109号
关于成立教育信息化领导小组的决定
2014.11.24
32
院字〔2014〕113号
泰山职业技术学院仪器设备管理办法
2014.12.03
33
院字〔2014〕116号
校企合作管理办法
2014.12.02
34
院字〔2014〕126号
学分制与弹性学制管理办法(试行)
2014.12.20
35
院字〔2014〕220号
与汶上县职业中等专业学校结对帮扶工作实施方案
2014.12.30
36
院字〔2015〕11号
现代学徒制试点工作实施方案
2015.01.22
37
院字〔2015〕16号
关于表彰山东省高等教育名校2014年度优秀建设项目的决定
2015.03.16
38
院字〔2015〕20号
科技特派员工作管理办法
2015.03.15
39
院字〔2015〕45号
科研项目研究经费资助办法
2015.10.14
40
院字〔2015〕48号
关于聘请行业企业技术专家为学院校外兼职专业带头人的意见
2015.11.23
41
院字〔2015〕62号
职业教育教学改革项目研究经费管理与奖励办法
2015.12.15
2.强化督导考核。建立名校建设项目诊断改正机制,对项目建设过程采取全程跟踪督导检查,发现问题及时整改,对标找差,纠偏补遗,确保每一项目建设严格按照任务书规定标准执行。通过召开专题例会、推进会、反馈会、协调会、专家报告会等方式解决项目建设存在的问题。自项目启动以来,共组织中期检查、年度检查、专项检查、校内预验收等检查、验收、整改活动20次。
制定了《泰山职业技术学院省级特色名校建设项目绩效考核办法》,建立整体绩效考核和分级绩效考核两级考评体系,实行不定期的跟踪检查和阶段性(学期)考核及终期验收机制。根据项目建设进度和质量进行奖惩。对建设成效显著,考核成绩突出的项目给予表彰和奖励,对未达到项目进度和质量标准的项目,进行约谈通报批评,并视情节后果进行处罚。在历次检查考核过程中,根据建设质量和进度,学院共发放绩效考核奖金41.9万元。这些措施有力保证了项目建设质量。
3.营造浓厚氛围。建立了名校建设专题网站,刊发简报,加强名校建设信息的内外互通。定期发布国家、省有关教育特别是高等职业教育的政策法规,及时向社会公布展示学院名校建设进度与成果。建设期内,共发布通讯479篇,简报37期。专门建立了FTP电子网站,全部建设项目形成电子档案,及时上传至FTP网站,实现了对名校建设档案信息化管理。
   
              图1-13名校建设专题网站        图1-14名校建设工作简报
四、严格预算管理,资金使用规范
根据名校建设方案,省财政投入1000万元,市级配套财政投入为3000万元,学院投入为1040万元,行业企业投入500万元,共计5540万元。实际投入11841万元、执行11574万元。
印发了《泰山职业技术学院省级特色名校建设项目专项资金管理办法》,坚持“集中使用,突出重点,总体规划,分年实施,项目管理,绩效考评”的原则,纳入学院总体预算,按照专帐核算、专款专用的管理模式,严格经费使用管理。
表1-4名校建设资金预算、到位与执行一览表
单位:万元
资金来源
预算额
到位额
执行额
执行率(执行额/预算额)
省财政资金
1000
1000
1000
100%
市财政资金
3000
3000
3000
100%
行业企业
500
1262
995
199%
学校自筹
1040
6579
6579
632%
合计
5540
11841
11574

第二部分 建设目标完成情况和建设成效
一、建设项目指标完成情况
(一)项目建设任务全面完成
在山东省教育厅和财政厅的领导下,在泰安市委市政府的大力支持下,学院严格按照省教育厅批复的建设方案和任务书,扎实推进名校建设工作,项目建设任务目标全部完成,部分指标超额完成。“服务”价值取向已经成为指导学院办学的核心理念,“多方联动,协同发展”多元化办学体制机制已经形成,“德技并重,理实一体”人才培养模式逐渐成熟,融传统文化、泰山文化、企业文化和大学文化为一体的泰山特色校园文化氛围更加浓厚,独具泰山品牌高职办学特色日益凸显,“四有型”专兼职教师队伍结构更加合理,重点建设专业与泰安市主导产业契合度进一步提高,人才培养更加符合地方经济社会发展需要,学院综合办学实力和核心竞争力不断增强,影响力不断扩大,服务区域经济社会发展能力进一步增强。项目建设任务完成情况见表2-1。
表2-1 项目建设任务完成情况一览表
项目类别
项目名称
验收要点
实际完成
完成率(%)




综合建设项目
办学特色
12
12
100%

专业建设
47
47
100%

师资队伍建设
55
55
100%

教学改革
118
118
100%

体制机制建设
61
61
100%

社会服务能力建设
33
33
100%

招生就业
9
9
100%

国际交流与合作
12
12
100%
省财政支持重点建设专业(专业群)
机电一体化技术
318
318
100%

建筑工程技术
351
351
100%

会计电算化
337
337
100%

旅游管理
334
334
100%

计算机应用技术
327
327
100%

园艺技术
340
340
100%
地方财政支持的重点专业(专业群)
汽车电子应用技术
262
262
100%

服装设计
379
379
100%

珠宝首饰工艺与鉴定
169
169
100%
特色建设项目
校企合作体制机制
42
42
100%

教学质量监控与保障体系
54
54
100%

泰山特色校园文化
51
51
100%

数字化校园建设
30
30
100%
合  计

3341
3341
100%
(二)各项建设任务主要指标完成情况
经过三年建设,学院各项建设指标全部完成,部分指标超额完成。建设项目主要指标完成情况见表2-2至2-9。
1.办学特色






表2-2 办学特色建设主要指标完成情况一览表
建设目标
验收
要点(个)
完成
情况(个)
完成率(%)
办学理念
形成突出“服务”价值取向的办学理念
3
3
100%
办学模式
形成“多方联动,协同发展”的多元化办学模式
3
3
100%
人才培养模式
创新完善“德技并重、理实一体”的人才培养模式
3
3
100%
泰山特色校园文化
充实发展融传统文化、泰山文化、企业文化、大学文化“四位一体”的泰山特色校园文化
3
3
100%
2.专业建设
表2-3  专业建设主要指标完成情况一览表
建设目标
主要指标
单位
完成
完成率
专业调研、岗位分析
专业调研和职业岗位能力分析
3
次
3
100%

人才需求调研报告
31
个
31
100%
专业布局与调整机制
专业布局和调整
3
次
3
100%

专业调整方案
3
个
3
100%
优势与特色专业
省级特色专业
3
个
1
33%
实训条件
校内实训室
65
个
70
108%

校外实习实训基地
68
个
74
109%
专业标准与评估
专业标准
31
个
31
100%

专业评估办法
1
个
1
100%

专业年度质量报告
2
次
2
100%
3.师资队伍建设

表2-4  师资队伍建设主要指标完成情况一览表
建设内容
主要指标
单位
完成情况
完成率
生师比
生师比
18: 1
A/B
15.31:1
完成
师资队伍结构
专任教师高级职称比例
29.9
%
31.64
106%

专任教师硕士学位以上学历比例
59.1
%
60.17
102%
专业带头人与骨干教师
专业带头人
46
人
46
100%

骨干教师
100
人
162
162%
教师培训与双师素质提升
教师培训人数
320
人次
883
276%

省级教学团队
1
个
1
100%

省级教学名师
1
人
1
100%

市级教学名师
2
人
2
100%

院级教学名师
-
人
20
超量
兼职教师
兼职教师资源库人数
300
人
331
110%
4.教学改革
表2-5   教学改革主要指标完成情况一览表
建设目标
主要
指标
单位
完成
情况
完成率
人才培养方案优化
专业人才培养方案
31
个
31
 100%

岗位调研分析
31
个
31
 100%
教学内容和课程体系改革
人才培养方案
31
个
31
 100%

课程标准
410
门
410
 100%

省级精品课程
25
门
7
 28%

特色教材
35
门
45
 128%

专业教学资源库
9
门
30
 330%

省教改立项
-
项
7
 超量
人才培养模式创新
人才培养模式创新
9
个
9
 100%
实践教学模式
新增、改扩建实训室
65
个
70
 109%
教学模式改革
省级以上教学成果奖
1
项
8
 超量

名师工作室
-
个
4
 超量
大学生创新创业能力培养
大学生科技创新立项
60
项
95
 158%
突出技能大赛引领作用
省职业技能大赛获奖
-
项
297
 超量
5. 体制机制建设
表2-6  体制机制建设主要指标完成情况一览表
建设目标
主要指标
单位
完成情况
完成率
校企合作体制机制
职业教育联席会
1
个
1
100%

校企合作理事会
1
个
1
100%

校企合作委员会
9
个
9
100%

校企合作制度
8
个
13
162.5%

校中厂(教学型工厂)
9
个
11
122%

厂中校
10
个
16
160%

订单班、冠名班
6
个
22
367%

泰安市公共实训中心
-
个
1
100%

泰安市创新创业教育中心
-
个
1
100%

泰安市创业大学
-
个
1
100%

现代学徒制试点
-
个
1
100%

泰山职业技术学院东平移民学院成立
-
个
1
100%

泰安市泰山玉雕人才培养中心
-
个
1
100%

泰安市工艺美术学会等行业协会
-
个
12
100%

合作企业投入资金(累计)
-
万元
113.8
100%

合作企业投入设备(累计)
-
万元
619.52
100%

合作企业学年投入人力(累计)
-
人天
6064
100%

合作企业提供顶岗实习岗位数(累计/年均)
-
个
992
100%

泰安市现代职业教育研究院
-
个
1
100%

新增合作企业
68
个
68
100%

新增职教集团、合作联盟
-
个
6
超量

合办专业
-
个
3
超量
内部管理
学院章程
-
个
1
100%

十三五规划
-
套
1
100%

学院管理水平提升行动计划
-
个
1
100%

院系二级管理
-
套
1
100%

内部管理制度(100个)文件汇编
1
册
1
100%

激励考核
5
个
6
120%

7S管理
2

文件2个,检查通报14次
100%
教学质量监控与保障机制
监控主体
年度人才培养质量报告
年度
3
100%


专业人才培养状况报告
年度
1
100%

教学质量监控与保障的标准体系
专业人才培养方案
年级
3
100%


课程标准
门
410
100%


修订、新增教学管理制度、文件
个
25
100%


专业标准(2015级)
个
29
100%

教学质量监控与评价体系
教师职教能力培训与测评
人次
296
100%


新增设专业
个
5
100%

教学质量反馈与调控体系
教学常规整改报告
学期
4
100%

完善人才培养工作数据采集平台
数据采集、分析报告
学年
3
100%

教学管理信息平台
教务管理系统
套
1
100%


专业教学资源库
个
18
100%


专业核心课程
门
48
100%
6.社会服务能力建设
表2-7  社会服务能力建设主要指标完成情况一览表
建设目标
主要指标
单位
完成情况
完成率
社会培训
社会培训
192000
人时
1052622
548%

成人函授人数
1800
人
2000
110%
职业资格考核鉴定
职业资格考核鉴定
6000
人
10309
171%
技术研发与服务
技术研究中心
3
个
3
100%

立项纵向课题
55
项
90
223.6%

横向课题


33


技术服务(项)
300
项
306
102.0%

技术服务收入
1500
万元
2128
141%
对口支援
对口支援西部学校
2
个
2
100%

对口支持地方学校
6
个
7
100%

培训中职师资
300
人
360
166%
7.招生就业
表2-8   招生就业建设主要指标完成情况一览表
建设目标
单位
主要指标
完成情况
完成率
生源质量
一志愿上线率
%
90
116
129

新生录取报到率
%
84
87
104%

单独招生比例
%
30
31
103%

在校生数(全日制高职)
人
6000
8763
146%
就业率与就业质量
毕业生总体就业率
%
98
99.9
102%

专业相关度
%
80
80
100%

用人单位满意度
%
92
93
101%
8.国际化办学
表2-9  国际化办学建设主要指标完成情况一览表
建设目标
主要指标
单位
完成情况
完成率
国际交流与合作
友好学校
5
个
7
140%

合作学校
2
个
3
150%

培训师资
103
名
211
205%

学术交流
2
次
5
250%

互派留学生
40
名
56
140%

引进课程
0
门
26
260%
二、项目综合建设层面成效显著
(一)顶层设计科学合理,办学特色更加突出
学院坚持以服务为宗旨,以优质就业为导向,以人才培养为根本,各项工作凸显“服务”价值取向,创新“多方联动,协同发展”多元化办学模式,深化“政行企所校”多元参与办学,激发办学活力;创新完善“德技并重,理实一体”人才培养模式,全面提高人才培养质量;以培养学生工匠精神和职业精神为重点,深入挖掘泰山文化精髓,推进产业文化进教育、企业文化进校园、职业文化进课堂,增进校园文化与经济社会的融合,充实完善“四位一体”泰山特色校园文化。
1.顶层设计清晰,凸显“服务”价值取向办学理念
学院依托厚重的泰山文化优势,站在建设高水平特色名校,为区域经济做出更大贡献的历史高度来谋划未来的发展,站在以文化引领内涵发展彰显办学特色的高度来思考顶层设计。基于此,学院明确办学定位,理清发展思路,提出了建设“服务”价值取向的建设理念,构建“服务”为核心的价值观。在发展思路上,按照“一二三四五”名校建设基本思路,推进名校建设工程。“一”是围绕一个中心,即围绕“服务”价值取向建设名校。“二”是实施“两轮驱动”,以建设“学习型” “创新型”校园为基础,实现名校建设的学习驱动和创新驱动;“三”是明确三项工作要求,包括服务意识强,服务能力高,服务品质优;四是突出“四条主线”,即以“服务学生成人成才” “服务区域经济社会发展” “服务教师专业成长”和“服务文化传承创新”为主线,将“服务”价值取向的办学理念落实到各项具体工作中;五是突出“五个重点”,即以专业建设为龙头,突出体制机制、师资队伍、教学改革、办学条件、社会服务和校园文化五个重点;通过纵向的十三个建设项目(9个重点建设专业和4个特色项目)、横向的十四个载体(“4433”名校建设工程),形成纵、横交织的工作网络,全面推进名校建设,从而达到内涵发展、特色发展、协调发展和全面发展。











图2-1名校建设思路示意图
案例1:立足服务三农,助力精准扶贫
为全面建成小康社会,推进社会主义新农村建设,为农村培养更多懂技术、会经营、善管理、素质高的新型职业农民,大力提高农村基层人员技术和素质,园艺技术专业突出“服务”价值取向,立足服务三农,以切实帮助村民脱贫致富、增强自我发展能力为目标,对农技人员开展技术培训、技能鉴定等各种社会服务。
发挥学院山东省现代农业培训基地和新型农民阳光工程培训基地的优势,积极联系泰安市农业局科教站,按照《山东省基层农技人员培训实施方案》和《山东省基层农技推广人员能力提升工程实施方案》的要求,以园艺专业建设为龙头,带动园林、食品加工和畜牧兽医等专业,对泰安市基层农技人员开展能力提升培训、“庄稼医生”培训,以及面向社会开展多种形式的培训和技能鉴定工作。园艺专业科技团队深入田间地头、走进蔬菜大棚,现场对农民进行技术技能培训。三年来,对基层农技人员培训305人;农村劳动力阳光培训1200人。通过技术服务、培训服务,给农村带来翻天覆地的变化,给农民带来实实在在的实惠,为泰安市新农村建设做出积极的贡献。

图2-2  园艺技术专业对基层农技人员进行培训
2.实施“多方联动、协同发展”多元化办学模式,办学活力增强
依托市职教联席会议、企业与企业家协会等平台,统筹政、行、企、所、校资源,健全了政府主导、行业指导、企业参与、科研机构联合的职业教育办学体制;通过实施“实训实习基地内涵提升工程”和创新校企合作体制机制项目,以“校中厂” “厂中校”为依托,进一步完善“厂校一体、资源共享”的校企合作模式。通过与中职、本科开展“3+2”贯通中职-高职-本科培养;拓展国际合作领域,积极开展国际合作办学等形式,多元化办学体制初步建立,办学活力增强。
与市国家级高新区建立了区校战略合作关系,与东平县合作建立东平移民学院;与华中数控股份有限公司合作成立教育联盟,与泰安市农业科学研究所共同开展科学研究与社会服务;与汶上职业中专等7所中职学校和山东科技大学分别开展中高职、专本“3+2”贯通培养合作办学,参与构建了中高职衔接的现代职业教育体系;与澳大利亚启蒙思学院、泰国、台湾昆山科技大学等境外、国外教育或培训机构合作办学,办学活力不断增强。
    
图2-3学院与武汉华中数控签订校企合作协议 图2-4东平移民学院揭牌仪式
案例2:泰安市机电职教集团成立
2013年,泰安市政府主导,我院牵头的泰安市机电职教集团在我院成立,并召开了第一届理事会。市政府、市经信委领导,各县市区职教中心、全市机电行业企业的代表出席了会议。会上举行了泰安市机电技术职业教育集团揭牌仪式,并审议通过了《泰安市机电技术职业教育集团章程》、《泰安市机电技术职业教育集团运行管理实施细则》、《泰安市机电技术职业教育集团三年发展规划》。
泰安市机电技术职业教育集团联合了省市内中、高职院校,大、中型企业40余家,深化了校企合作体制机制,搭建高校人才成长“立交桥”,切实为我市机电产业发展提供人力资源和产学研保障。
案例3:学院与泰安市高新技术开发区实现区校合作
为充分发挥学院资源为地方区域经济服务,学院与泰安市高新技术开发区经洽谈协商,形成了区校战略合作协议。区校双方根据高新区经济社会发展需要和学院教育资源实际,就企业用人订单培养、科研技术合作和学生就业创业等方面进行深度合作。学院安排1名副院长到高新技术开发区挂职。
此次区校联动、战略合作,是我院创新校企合作体制机制、服务地方经济社会发展的一项重要举措,也是我院深化校企合作,实现校企优势互补、资源共享和互惠共赢的重要途径。
3.完善“德技并重、理实一体”的人才培养模式,扎实推进有效教学改革
为适应学生职业生涯发展和社会经济发展需要,学院不断完善“德技并重、理实一体”人才培养模式,人才培养质量显著提升。在实践中明确了“德技并重、理实一体”人才培养模式的内涵,主要包括:“德高技高”的人才培养目标,“德育为首、能力为本”的教学理念,“职业素养与专业技能并重”的教学内容,“理论与实践交融”的教学方式。通过实施“优势特色专业培育工程”,对接行业标准,优化人才培养方案,将人文素养和职业素质教育纳入人才培养方案,培养德高、技高“双高型”技术技能型人才。通过实施“课程体系优化工程”,以职业能力培养为主线,以工作过程系统化的课程改革为重点,完善“平台+模块”的课程体系;全面实施以项目课程和有效课堂改革为重点的有效教学改革,推行项目式、任务驱动教学,促进了职业技能培养与职业精神养成的融合;实施“分阶段、项目化、协同式”实践教学模式改革,提高了学生的实践动手能力。为地方经济社会发展培养出大批德高、技高“双高型”技术技能型人才。各专业根据人才培养规律,结合企业生产、用工实际情况等不同特点,分别探索出各具特色的人才培养模式。
案例4:旅游管理专业推行“旺入淡出、产学结合”人才培养模式
依托泰安市优质旅游企业,旅游管理专业构建了“旺入淡出、产学结合”人才培养模式。学生在每年6-10月份的旅游旺季,深入企业顶岗实践,进入旅游淡季则返校进行理论知识学习和基本技能训练。在学校学习阶段,学校教师为主,企业人员为辅;在企业顶岗实习阶段,企业技术、管理人员为主,学校教师为辅,校企合作培养学生,提升专业技能和职业素质。该专业毕业生深受企业欢迎,毕业前与企业签订有较高待遇的就业合同。
案例5:实施有效教学改革,改变课堂生态
突出“学生主体,能力本位”,进一步加强内涵建设,课堂教学改革,推进以课程改革和课堂教学改革为主要内容的有效教学改革。突出“能力本位、素质渗透、项目载体、成果检验”的基本理念,进行项目化课程改革。按照《泰山职业技术学院关于开展教师职教能力培训与测评活动的通知》《关于实施有效课堂教学改革的实施意见》,对全院教师进行了三期以课程整体教学设计、单元教学设计为主要内容的培训与测评,累计有296名教师测评合格。有效地推进了项目化课程改革。
有效课堂教学改革,坚持“以学生的学习成效”为评价标准,突出“重设计,以做促学促教;重实效,以学评教评师”的基本理念,按照“明了任务、讲解示范、合作学练、评议展示、总结提升”的基本结构,以学生到课率、参与率、达标率等六项内容为具体指标,以随机听课为评价方式,推进课堂教学改革。
通过实施顶层设计、整体推进、分步测评、全面实施等系列措施,以学生的发展为中心,深入融合“理念、模式、实践”三创新,将课程改革与课堂教学改革有机结合在一起,解决了教(学)什么、怎么学、怎么教、如何评等问题,有效提升了广大教师的现代职教理念,进一步促进了各专业教学模式改革,全面提升了整个课堂的教学效率,从而带动了学院整体教育教学水平提升,保障了教学的有效性,全面提高人才培养质量。教务处、督导室和学院工会联合组织开展了6次专题教研、优质课评选、信息化教学比赛、公开课等活动,同步推进有效教学改革。
4.完善“四位一体”泰山特色校园文化,学生职业精神培养成效显著
培养具备精益求精、专心敬业工匠精神的高素质技术技能型人才,是时代赋予高职教育的伟大使命。围绕育人中心任务,坚持文化育人理念,以培养大学生职业精神和持续发展能力为目标,以泰山书院为载体,大力实施泰山特色校园文化建设项目。通过深入发掘泰安文化产业要素,优化校园文化与行业、企业文化交互平台,进一步充实完善了融传统文化、泰山文化、企业文化、大学文化“四位一体”的泰山特色校园文化,校园文化底蕴更加深厚。








图2-5校园文化建设示意图
(1)建设泰山书院
以泰安悠久深厚的地方历史文化为背景和资源,对接地方优势传统文化,重建千年泰山书院。先后完成博物馆、文渊阁等书院场馆建设,搜集整理泰山文化书籍、文献,出版了《泰山书院》刊物,开办讲书堂,举办了包括泰山文化、大学文化、传统文化、企业文化在内的系列文化教育讲座40余次,泰山书院获批为山东省社科普及基地和泰安市社科普及基地。开展国内外文化交流上百次,先后迎接中国水墨艺术研究院、省内外高校等国内各界人士以及美国、澳大利亚、韩国等国外文化教育界人士前来书院参观约百次。泰山书院成为研究、传承和弘扬泰山文化,培养学生职业精神的重要基地和平台。
(2)环境文化建设
按照推进产业文化进教育、企业文化进校园、职业文化进课堂,增进校园文化与经济社会的融合建设思路,完善“两场、两园、六区”文化建设。“两场”即:青春广场、创业广场。“两园”即:后山生态园和望岳园。“六区”(功能定位)即:办公区、教学区、实训区、生活区、公寓区、运动区,增设体现泰山精神的雕塑、石刻、主题景点、展板等物质文化形式。
(3)制度文化建设
进一步修订完善学院与部门规章制度,建立高效、民主的内部管理运行机制。建立、完善校企合作制度。促进企业文化与大学文化在制度方面的融合。
三是完善教职工职业道德规范评价办法,将遵守《高等学校教师职业道德规范》情况纳入教师绩效考核。完善体现职业特点的学生评价体系,促进学生综合职业素质的养成。建立、推行“7s”管理制度,促进企业文化与大学文化相融合。
(4)行为文化建设
开设泰山文化“讲书堂”,通过“读书月” “读书推介活动” “好书天天读” “早说晚练” “泰山文化演讲、征文比赛”等系列活动,引导师生多读书,读好书,好读书,建设浓郁的书香校园。
开展“三月雷锋月” “四月读书月” “五四青年节” “四月、九月体育节” “九九感恩父母老人节” “十月份大学生科技文化艺术节”等大型活动,达到陶冶学生情操,增强学生体质、弘扬传统美德之目的。
组织开展丰富多彩的社团主题活动。支持学生组织的望岳文学社、摄影、书法、绘画、武术、科技创新等社团70个左右,积极为社团活动提供支持,提升学生综合素质,增强学生的团队意识。
开展志愿者服务活动。教职工通过工会组织建立泰山职业技术学院志愿者服务队,学生通过团委和系部牵头建立朝阳志愿者等服务队,师生志愿者服务队同市文明委开展的菜单式志愿者服务活动相衔接,走进泰山景区、社区、敬老院、福利院、医院等地,开展导游疏客、敬老助残、解危助困、科技下乡等形式多样的服务活动。
   
图2-6泰山书院                   图2-7专家讲座
案例6:泰山文化创意产品设计与开发--创新传承泰山文化途径
名校建设期间,依托泰山书院平台,不断弘扬传统文化,探索文化传承创新途径,完成泰山百景诗书画印作品创作。搜集整理泰山著名百景,以其为素材,以本院教师为主要创作人员,联合本地知名书画家,完成了对泰山百景的梳理和诗书画印综合创作,印刷成册。在泰山文化传承上,首次创造性地对泰山一百个著名景点以诗书画印的形式进行了系统创作,受到了泰安市社科联和书画界的高度评价。
在创作完成基础上,加强成果转化力度,不断将本系列创作成果作为文化元素,应用于文化旅游产品设计当中,实效显著。2015年,应用产品设计荣获中国泰安(泰山)平安文化旅游商品大赛金奖。泰山百景篆刻在茶具设计得到开发应用。

图 2-8篆刻作品“对松叠翠”与参赛服饰设计作品及获奖奖杯

图2-9  泰山百景篆刻设计茶具设计应用
5.其它特色与创新工作
(1)搭建人才培养“立交桥”,现代职业教育体系建设加快
积极开展现代职业教育体系研究与建设,探索建立高职与中职、与应用本科相互贯通的人才培养机制,搭建人才培养“立交桥”。与泰安市域内6所中职学校和汶上县职业中专建立了合作关系,与4所中职学校8个专业开展3+2中高职人才贯通培养,与山东科技大学、泰山学院分别开展2个专业高职本科分段贯通培养。基本形成以专科层次高等职业教育为主,中等职业教育和应用型本科教育贯通的现代职业教育体系。     
(2)推进集团化办学,探索校企合作新模式
泰安市政府主导,我院牵头,联合省市内中、高职院校,大、中型企业40余家,共同组建泰安市机电职教集团。职业教育集团的成立,促进了办学主体的多元化,深化了产教融合,校企合作,为进一步探索混合所有制办学奠定了基础。
(3)积极推进现代学徒制试点,引领人才培养模式改革
本着“互利共赢、完全自愿”的原则,学院与泰安市泰山玉研发有限公司、泰安市先锋自动化数控设计中心、泰安市遇石记玉石制品有限公司、北京李博生造型艺术有限公司合作的首饰设计与工艺鉴定专业开展现代学徒制试点,联合培养企业急需的人才,2016年被批准为省级试点专业。
(4)大力推进国际化办学,国际合作领域不断拓宽
学院积极参与国际合作与交流,大力推进国际化办学,与澳大利亚启思蒙学院、加拿大荷兰学院和卡纳多文理应用学院等3所大学签署合作办学协议,引进国外先进教育理念、课程和教学管理模式,派遣教师到国(境)外培训,互派留学生、进行学术交流,国际合作领域不断拓宽,国际化办学水平显著提高。
(5)校园文化建设特色鲜明,职业精神培养成效显著
发挥文化育人独特优势,以培养学生职业精神、奠定学生可持续发展能力为目标,充实、发展融传统文化、泰山文化、企业文化、大学文化“四位一体”的具有鲜明泰山特色的校园文化,培养具有“进取、担当、包容、和谐”泰山精神的“泰职人”,实现以文化人,文化育人功能,学院成为社会主义先进文化的重要基地和大学生文化素质教育基地,形成了泰山职教品牌。
(6)成立创业大学,开启创新创业教育新篇章
实践“大众创业、万众创新”和“中国制造2025”国家发展战略,培养更多创新型人才。整合资源,成立泰安市创业大学,把创新创业能力培养纳入人才培养方案,构建了以课堂教学为基础、以教育活动为载体、以实践锻炼为手段、以创业大学建设为平台,“四位一体”的创新创业教育体系。通过举办创新创意比赛和社团活动,指导学生制订创业规划,进行创业实践,对有创业潜力的项目进行创业孵化;通过“大学生科技创新项目”,鼓励学生参与企业技术革新、工艺改造,提升学生创新实践能力,促进创新成果形成与转化。
(二)实施“优势特色专业培育工程”,专业服务产业能力显著提升
学院立足产教融合,校企合作,以“优势特色专业培育工程”为抓手,以建立常态化专业调研、专业调整与优化、专业评估与内部有效管理机制为重点,大力推进专业建设。紧紧围绕区域重点优势产业发展格局和经济方式转变,重点建设机电一体化、旅游管理、园艺工程技术等9个优势和特色专业,专业与产业契合度进一步增强,专业服务产业能力显著提升。
1.建立专业内部有效管理机制
按照“对接市场需求,优化专业结构,突出重点专业,带动专业群建设”的专业建设思路,健全了由学院领导、教务处、系主任、专业负责人、职教专家、企业人员等参与的学院专业建设指导委员会,进一步明确了院系两级专业管理权责,通过定期召开委员会议,研讨专业调整优化、人才培养方案修订、实训室建设等专业建设重点工作。各系健全了由系主任负责的专业建设委员会,各专业成立由专业负责人任组长的专业建设团队,形成了系主任全面负责专业建设,教学副主任协助管理专业建设,专业建设团队落实好专业建设的专业管理氛围。
学院修订了《人才培养方案优化原则意见》《专业评估实施方案》《专业调研实施办法》《顶岗实习管理办法》等一系列专业管理制度,进一步规范了专业内部管理机制。
2.形成常态化的专业调研和岗位分析机制
由学院专业建设指导委员会总体部署,各系专业建设委员会制订调研制度、明确职责、强化考核,各专业建设团队确保每个专业每年进行一次专业调研与职业岗位分析。围绕省会城市群经济圈发展和泰安市旅游文化、高压输变电设备、汽车、电子信息、金融、房地产等产业,各专业深入与本专业人才培养密切相关的行业企业开展专业调研。通过现场走访、召开座谈会、问卷调查、电话网络调研等方式,对政府主管部门及相关机构、行业企业负责人、技术骨干、毕业生等,对行业、企业发展现状及趋势,职业岗位、工作任务、工作内容,人才需求及知识、能力、素质要求等内容进行调研,形成专业调研报告,为专业人才培养提供可靠依据。
三年内,9个重点专业对90余家企业和50多所国家示范和骨干、省特色名校进行了调研。通过调研,对行业企业人才需求状况进行统计、监测和职业岗位分析,进一步明确了专业服务面向、人才培养目标、数量和规格,并形成专业建设调研报告,为学院专业设置和调整优化、专业人才培养方案修订、课程结构优化、教学内容调整、学生就业指导等提供依据,使学院专业人才培养紧密对接岗位标准和要求。
根据专业调研结果,对现有招生专业进行研判,对连续三年招生形势不好的,逐步调整招生计划,直至停招退出招生。
3.调整优化专业结构与布局
坚持专业设置与产业需求相结合,适应性与前瞻性相统一、稳定性与灵活性相结合,由学院专业建设指导委员会根据专业建设调研报告,对泰安经济社会发展规划重点建设的产业结构和市场需求,对所有招生专业进行科学论证,提出专业增、留、改、并、撤意见,确定专业调整方案。形成了主动适应区域经济社会发展,与市场接轨,适合自身发展的专业布局与结构调整机制。
三年来,学院服务区域经济社会发展,对接泰安重点产业,相继增设了航空服务、道路桥梁工程技术、汽车检测与维修技术、学前教育、工业机器人技术5个三年制高职专业,并将计算机辅助设计与制造和计算机网络技术两个专业调整为模具设计与制造和物联网应用技术,淘汰专业保险实务等10个专业,专业设置与泰安市重点产业布局更加紧密。
到2015年,学院现设有中央财政重点支持专业、省特色专业6个,2015年学院招生专业29个,涵盖了农林牧渔、土建、制造、电子信息、财经、艺术设计、旅游、轻纺食品、交通运输等9个大类。有4个专业大类专业设置数量达到3个或以上,其中制造大类7个专业、财经大类4个专业、电子信息大类5个专业,专业群优势较明显。2013年与澳大利亚启思蒙学院联合举办的建筑工程技术和机电一体化技术两个专业;2015年与山东科技大学开展“3+2”专本对口贯通培养机械设计及其自动化专业;2016年珠宝首饰工艺及鉴定专业开展山东省现代学徒制专业试点。与宁阳、东平、新泰、岱岳等职业中专合作举办8个“3+2”专业,多种形式办学,促进专业更快发展。
表2-10专业结构调整与优化一览表
专业调整与优化
专业名称
备注
新增专业
航空服务
2013年新增

汽车检测与维修技术


道路桥梁工程技术


无
2014年新增

无


工业机器人
2015年新增

学前教育

淘汰专业
保险实务
2015年未招生

食品加工技术
2015年未招生

商务英语
2015年未招生

金融与证券
2015年未招生

动物防疫与检疫
2015年未招生

焊接技术及自动化
2015年未招生

资产评估与管理
2015年未招生

楼宇智能化工程技术
2015年未招生

物业管理
2015年未招生

纺织品检验与贸易
2015年未招生
调整优化专业 
将计算机网络技术调整为物联网应用技术专业
2013年调整优化

将计算机辅助设计与制造,调整为模具设计与制造
2013年调整优化
案例7:构建现代职教体系,促进中职、高职、本科上下贯通
学院与宁阳职业中专、新泰职业中专等4所学校的8个专业,与山东科技大学、泰山学院2所本科院校的2个专业开展“3+2”合作办学,实现中职、高职、本科上下贯通。
学院与政府职能部门、行业企业、职业院校等多方联动,以研制人才培养方案为抓手,以教学资源开发为核心,积极推进中职、高职到应用型本科的上下贯通。一是联合研制人才方案。以人才培养方案研制为抓手,分别明确中职阶段技能操作的培养目标,高职阶段工艺、项目实施的培养目标,本科阶段研发设计的培养目标。近年来,研制五年一贯制专业人才培养方案 7个、三二连读专业人才培养方案 8 个。二是共同开发专业课程体系。立足于学生职业素质和能力不断提升,教学内容不断线、不重复,明确中职阶段以文化基础课程和专业操作课程为主、高职阶段以专业基础课程和综合实训实习课程为主、本科阶段以技术应用和设计开发课程为主的开发思路。如,3+2 高职本科分段贯通培养机电一体化专业根据中职毕业生特点,高职阶段增加公共基础课程和专业基础课程比重,突出互换性及技术测量和数控加工及编程的课程设置;本科阶段注重机械创新设计、技术应用研发、管理能力等方面课程设置。
4.制订专业教学标准,建立专业评估机制
坚持学校与行业企业、专业设置与职业岗位、课程内容与职业标准、教学过程与生产过程深度对接的“四结合”原则,完善专业标准。以国家专业教学标准及省专业教学标准为依据,所有招生专业都修订完善了专业标准。
健全了常态化专业评估制度。2014年,在调研分析的基础上,参考省级品牌专业评估标准,完善专业评估的指标体系,在各专业建设委员会充分讨论的基础上,制定了《专业评估实施方案》。2015年,由学院专业建设指导委员会组织,通过查阅资料、数据分析、访谈、听课、实地查看等形式,对技能型特色名校9个重点专业实施了诊断性评估。每个专业每年都形成了年度质量报告,有效监测了专业建设质量,提高了人才培养水平。
学院每年接受山东省职业教育评估所的专业质量评估,按照评估指标体系,对各专业建设情况进行梳理总结,提交相关数据及佐证材料。在2014年省高职专业评估中,农林牧渔、轻纺食品、艺术设计、财经、电子信息、制造等专业大类,分列第5、6、10、11、14、21位。2015年,在山东省专业质量评估中,各专业排名稳中有升。
5.实施优势特色专业培育工程
根据泰安市现代服务业、先进制造业、现代农业等产业升级要求,学院不断强化专业建设,逐步提高专业群集中度,凝练专业特色,打造了8个质量高、特色鲜明的品牌特色专业(群),形成以点带面促进专业群协同发展的专业建设格局。
围绕泰安市建设国际旅游名城、打造山东旅游文化产业高地,推动文化产业成为国民支柱产业发展战略,打造旅游文化、艺术设计与加工专业群;围绕高压输变电设备产业基地,打造机电装备与制造专业群;围绕建设我省重要汽车基地,打造汽车与电气技术专业群;围绕信息、金融、房地产等服务业,打造信息技术、经贸服务和建筑工程专业群;围绕泰安加快现代农业发展,建设都市圈特别是省会城市的高品质农副产品供应基地的人才需求,打造现代农业专业群。
目前已建成电子商务和园林技术专业是中央财政支持的提升服务产业能力专业,应用电子技术、机电一体化技术、会计电算化、服装设计等4个省级特色专业;机电一体化技术、会计电算化、建筑工程技术、计算机应用技术、旅游管理、园艺技术、汽车电子技术、服装设计、珠宝首饰工艺及鉴定9个专业重点建设任务全部完成。以重点专业建设为引导,示范带动相应专业群发展,学院专业建设整体水平得到全面提升。











图2-10学院“区域产业-专业群-重点专业”布局图
6. 实施“实训实习基地内涵提升工程”,优化校内外实训基地建设
坚持投资多元、共建共享的原则,突出服务教学、服务科研、服务社会功能,按照实境化、生产性、先进性和开放性原则,新建2万平方米的“教学做一体化”实训楼(行知楼)并投入使用,拓展了校内实训空间。按照“一系一楼、资源共享”的思路,以专业群资源共享为依据,进行了实训室的布局调整,提高了实训室的使用效率。近三年来,新建、改扩建70个校内实训室或实训中心;建设具有生产能力的“教学型”工厂3个。以专业技能培养为核心,以“教学做一体化”室和“教学型”工厂建设为载体,形成了较为完整和先进的校内实训实习体系。经过持续建设,学院现有校内实训基地164个,建筑面积达54502.9平方米,教学仪器设备总值达7907万元,新增设备值2941万元,设备总数达8167台套,为实践教学提供了较为坚实的基础条件。校内实训基地整体数据如下表所示:
表2-11 校内实训基地情况
基地数
建筑面积(平方米)
设备值(万元)
设备数(台套)
专职管理人员(名)



设备总数
大型设备

164
54502.9
7907
8167
263
34
学院秉持“产业文化进教育、行业文化进校园、企业文化进课堂”理念,借鉴企业“5S”模式,推行“7S”实训室建设,实现实训室的规范管理与有效使用。加强实训室专业文化氛围建设,使企业文化、专业文化、实训室文化融为一体,耳濡目染,陶冶学生的职业文化素养。通过“7S”实训室管理,深化了产教融合,完善了管理使用制度,推动了管理理念和管理效能的提升,改善了校风、教风与学风。
 
图2-11推行“7S”管理
各实训室根据职业能力培养要求,对应课程与能力培养,对实训室工位进行合理规划设置,有效开展实训项目,充分发挥实训室功能。本着“校企共赢、合作共建、共管共享”的原则,不断加强校外实训基地建设,为学生校外实习及顶岗实习提供条件保障。各专业根据专业核心能力要求,遴选与专业对口度高、实训实习岗位充足的校外企业合作共建校外实训基地,把校外实训基地作为学生实习锻炼的平台、企业职工提升职业能力的平台。学院校企合作办会室牵头搭线,各系各专业以具体实训项目与指导教师为合作载体,明确校企双方职责,各司其职,校企及时相互沟通,保障基地建设与运行的顺利进行。各专业教研室制定实习实训计划,实习指导教师落实实施,学院落实相关经费保障,实习实训实现项目化、标准化,校企共同对实训学生进行考核评价。
近三年,新增了与重点建设专业(群)对应的73个校企深度合作的校外实训实习基地,基本满足了所有专业的实践和顶岗实习需要。学院校外实习基地239个,接待学生量7569人次。
表2-12 校外实训基地情况(2015届)
校外实习基地数
接待学生量(人次)
其中接受半年顶岗实习学生数(人)
学校派指导教师/学生管理人员(人次)
接收应届毕业生就业数(人)
181
6397
968
301
623
7.建成资源丰富、共建共享的专业教学资源库
根据教育部专业教学资源库的建设规范,2015年学院与中唐方德公司合作构建了技术先进、功能全面、重在应用的专业教学资源库平台,与超星尔雅、中国知网等企业合作,共同开发了以学习者为中心,以项目化课程资源建设为重点,集教学资源、互动学习、评价考核于一体的共享共建先进的信息化教学资源。
建设期内,建成了机电一体化技术、建筑工程技术、会计电算化、旅游管理、计算机应用技术、园艺技术、汽车电子技术、服装设计、珠宝首饰工艺与鉴定9个专业教学资源库。带动专业群启动建设专业教学资源库18个。开展了两期48门核心课程的数字化资源建设工作,前后举办了6期中唐方德教学资源库平台、实习实训管理系统等的使用培训会,全面提高我院信息化管理与应用水平。学院被省教育厅确立为山东省首批教学信息化试点单位。
案例8: 建设专业教学资源库,推进“互联网+教育”进程
学院高度重视教育教学信息化工作,先后邀请24位知名专家进校讲座,助推学院教学改革和信息化建设工作。作为中国职教学会教学资源建设研究会常务副主任单位、山东省教育信息化建设试点单位,进一步推进了学院信息化及教学资源库建设和应用。
学院与北京中唐方德科技有限公司合作,共同开发构建了技术先进、功能全面、重在应用的“共享型专业教学资源库平台与网络教学平台”,平台系统包括8个主要功能:专业资源库平台、课程资源平台、校企合作平台、课程开发平台、教学应用系统、移动学习系统、能力测试系统、实习实训系统,各系统间相互独立、资源共享,为建设专业教学资源库、建设优质网络课程资源、推行MOOCS、翻转课堂等教学改革,为学院、企业、系部、教师和学生搭建了良好的的信息交流和监控管理平台。目前已建设48门专业核心课程资源,在教育教学中推广应用。 

图2-12  2015年全国职业技术教育学会教学资源库建设研究会年度工作会议在我院召开
8.重点建设专业辐射带动作用明显,专业实力明显提升
通过校企合作体制机制创新、“平台+模块”课程体系构建、实训基地内涵建设、专业教学资源库建设,形成了专业群资源共享机制,重点专业辐射带动其它21个专业共同发展。建设期内,累计建成4个省级特色专业,1个省级示范专业,2个国家财政支持的提升服务产业能力专业,中外合作办学专业2个,“3+2”专本贯通专业1个,9个省级特色名校重点建设专业。在2014年省高职专业评估中,农林牧、轻纺食品、艺术设计、财经、电子信息、机电等专业大类,分列第5、6、10、11、14、21位。专业实力明显提升。
案例9:以点带面,机电一体化专业带动专业群共同发展
机电一体化专业积极探索专业群资源共享机制,发挥优势,实现群内师资、实训设备共建共享,提升群内专业建设整体水平。建设期间,在机电一体化专业的带动下,适应“中国制造2015”国家发展战略和泰安市装备制造业发展需求,学院成功申报工业机器人专业,拓宽了专业范围。为促进教师能力提高,机电专业一方面积极邀请国内知名专家来校开展学术讲座,提高教育教学理论水平,另一方面组织骨干教师利用寒暑假到泰航特车、泰和电力、泰开集团等企业顶岗锻炼63人次,分别参加厦门大学、山东职业学院等省级及以上培训45人次。通过培训,教师开阔了教师眼界,增强了理论水平,更新了教学理念,提高了教学能力。三年来,重点培养5名数控技术专业骨干教师,培养4名模具专业骨干教师。数控技术专业、模具专业等利用机电一体化实训室先进设备,提高技术服务水平,加强对口服务,积极开展自动化生产线维护培训、数控维修培训、维修电工职业资格培训与鉴定;自2013年以来多次为山东省职业技能大赛、山东省机电产品创新竞赛、泰安市职业技能竞赛等技能大赛提供了设备和技术保障。
(三)实施“名师递进培养工程”,促进教师团队专业化成长
实施“名师递进培养工程”,进一步优化师资队伍结构,落实引进培养聘用计划,学院建成了一支名师带动、骨干支撑、专兼结合的高水平师资队伍,教师专业水平全面提升。
   
图2-13 名师递进培养示意图
1.优化师资队伍结构,形成专兼结合双师型教师队伍 
通过引进、培养和聘用,不断优化师资队伍结构。学院积极争取上级人才政策,不断引进高尖端人才,出台相关政策文件,鼓励学院教师在职攻读硕士、博士学位,依托校企合作,选聘企业能工巧匠担任兼职教师、兼职学科带头人。设立兼职教师岗位、特聘专家岗位,吸引有专业经验的高级工程师、技术、技能人才到学校从教,使学校成为各类高层次应用型人才传授技术知识的聚集地。
近三年,共引进了泰安市拔尖人才2人、博士2人,调入、招聘教师18人,其中13人为全日制硕士研究生学历。现学院专任教师354人,其中副高以上职称112人,占比为31.64%,硕士以上学位为213人,占比为60.17%,双素质教师245人,占专任教师69.21%;2015年,外聘教师199人。生师比为15.31:1。
2.重视专业带头人与骨干教师培养,培育专业领军人物
实施“名师递进培养工程”,按照“教学新星、教坛英才、教学名师、功勋教师”四个层次,递进培养专业带头人、骨干教师,形成以专业带头人为核心、骨干教师为主体的专业教学团队。新引进教师以专业工作经历、职业资格高学历为基本条件,以测试专业技能和执教能力为主要内容,并根据专业需要,进一步优化学缘结构。完善《专业带头人选拔培养管理办法》,每个专业选拔1名优秀“双师素质”教师作为专业带头人培养对象,科学规划,重点培养,共培养校内专业带头人35人。重点专业实施“双带头人”制度,在重点培养9名校内专业带头人的同时,从行业企业中引进9名熟悉相关专业的前沿技术、较好地把握专业发展方向、具有一定影响力的管理专家和技术骨干为校外专业带头人。根据专业的工作实际情况,培养选拔了36名专业负责人。通过举办校内培训班、有计划地安排专业负责人、专业带头人参加学习考察、学术交流、专题研讨会,提高他们的现代职业教育理念和理论水平,掌握职业教育理论和教学方法,能够胜任专业核心课程或综合实践课程的教学。通过培养,专业带头人、专业负责人的教学、实践、科研、服务能力显著提高,在专业建设、课程改革、社会服务等方面充分发挥了示范引领作用,取得了国家级教学成果奖、省级精品课程、省级教学团队等一系列标志性成果。
建设期内,学院引进泰安市拔尖人才2人、博士2人,新增省教学名师1人,省级教学团队1个,泰安市功勋教师1人,泰山名师1人,泰山教坛英才2人,泰山教学新星2人。
3.加强师资培训,教师素质全面提升
学院成立教师发展中心,健全了教师培训体系。继续实施教师素质提升计划,以促进教师素质能力提升为重点,组织开展教师全员培训,对所有教师进行职教能力达标测评。三年内所有专任教师轮训一遍。实施访问工程师项目,完善专业教师每年利用寒暑假进行4周以上的生产实践锻炼,每年选派专业教师到企业进行实践锻炼,参与企业技术攻关和合作研发,使专业教师具有行业企业发展前沿的视野、较高的专业理论水平,同时又有丰富的企业实践经验和先进技术的应用能力。既能担当专业理论的教学,又能进行实践操作指导,还可承担企业、公司的一些项目研发,成为一支融理论教学、实训指导、项目开发为一体的“一专多能”的师资队伍。
建设期内,先后派出专业课教师、教学管理人员、政治辅导员等各类人员培训883次,84名教师国(境)外进修,47名专业带头人到浙江,40人到天津学习培训。2015年12月实施“顶天立地”培训,342名教师分别到北京、深圳、武汉等全国一流高校和专业对应行业的骨干企业进行理论学习和实践动手能力培训。先后邀请台湾崑山科技大学丁仁方、河南科技大学黄智高、马树超、李进、王汉忠等专家来校作报告。通过“走出去,请进来”,深入企业锻炼,大大提高了教师的专业水平和执教能力。
以赛促教,学院鼓励教师参加各级各类讲课比赛、职业技能大赛,对每年参加技能大赛和质量工程建设获奖的集体和个人进行了表彰奖励,三年奖励金额超过180万元。

图2-14  机电一体化教师在武汉华中数控有限公司进行机器人培训

表2-13 泰山职业技术学院境外教师培训统计表
序号
人数
姓名
培训地点
培训内容
时间
1
1
张庆臣
澳大利亚
澳大利职业教育培训
2013.06
2
3
孔令明、李文峰、朱广幸
澳大利亚
澳大利职业教育TAE四级证书
2014.01
3
24
马培安、李金伟、刘永海、邵荣振、赵成龙、夏乐斋、曾晓东、谭  毅、朱振国、胡泽生、李志强、耿忠义、李新林、张晓伟、韩廷水、陈咏梅、宋丽玲、辛太宇、赵京岚、李  倩、张  莹、陈卫华、张建明、张  丽
台湾昆山科技大学
台湾职业教育教学与管理
2014.02
4
20
王元成、屈克英、李晓东、王桂珍、何平和、孟宪玲、王  伟、李  萍、 李  莉、张  童、李德安、董  丽、孙康岭、黄永红、张  伟、张  勤、袁梦醒、李桂菊、张小童、张玲玲
台湾昆山科技大学
台湾职业教育教学与管理
2014.07
5
2
侯加阳、万文慧、
台湾昆山科技大学
台湾职业教育教学与管理
2014.09
6
26
刘洪东、韩  玮、亓玉华、倪   斌、孙衍训、张  勇、甄亚丁、张   冉、潘  平、孙国波、王  萍、李厚清、 张海鹏、于运会、苗  霞、李  琦、 刘向东、赵秀荣、张子学、吴殿勇、 陈振超、高明阳、耿国卿、刘洪岩、 秦世胜、杨立平
台湾昆山科技大学
台湾职业教育教学与管理
2015.01
7
2
辛显雪、程春艳
台湾昆山科技大学
台湾职业教育教学与管理
2015.01
8
2
张强、孙涛
台湾昆山科技大学
台湾职业教育教学与管理
2015.08
9
2
辛雷、王霞
台湾昆山科技大学
台湾职业教育教学与管理
2016.02
10
2
刘洪庆、王健
台湾朝阳科技大学
台湾职业教育教学与管理
2016.02
合计
84
 
 
 
 

表2-14教师赴浙江、天津培训情况
培训内容
培训人数
培训地点
培训时间(天)
专业负责人培训
47
浙江机电职业技术学院
7
高级研修
40
天津大学
8

表2-15学院“顶天立地”骨干教师寒假培训一览表
培训内容
培训人数
培训地点
培训时间
泰山职业技术学院师资综合能力提升班培训
28
湖南
2016.01.18-01.27
服装模板技术
11
深圳
2016.01.18-01.25
造型艺术设计及雕刻技术
8
福州
2016.01.18-01.25
工业机器人技术应用培训
19
武汉
2016.01.19-01.28
骨干教师培训:职业教育课程开发;资源库教育技术;理实一体化教学 ;德国工业4.0等。
18
武汉
2016.01.19-01.29
电气控制教师技能培训
14
 浙江
2016.01.18-01.26
Mathematic 教学软件培训
12
北京
2016.03.16-03.23
全国职业教育财经大类专业建设教育学改革与创新高级培训班
52
广州
2016.01.18-01.26
汽车新能源技术
21
广东
2016.01.18-01.27
厦门顶岗实习(旅行社及酒店考察、兄弟院校茶艺学习)
15
福建
2016.01.18-01.24
国际交流中心教师寒假培训
9
上海
2016.01.22-01.26
Android的UI设计
24
广东
2016.01.19-01.22
IOS开发设计的研讨和交流学习
24
深圳
2016.01.22-01.25
信息化技能、数字化校园
11
南京
2016.01.23-01.30
图书信息技能
11
福州
2016.02.24-02.26
专业骨干教师师资培训培训(高校)
33
上海
2016.01.17-01.24
专业骨干教师师资培训培训(企业)
32
北京
2016.02.21-02.25
合   计
342
4.加强兼职教师队伍建设,提高兼职教师教学能力
完善专业兼职教师聘任管理办法,出台了《外聘兼职教师管理办法》,建立了331人的兼职教师库,设立了100个兼职教师岗。兼职教师承担的专业课学时比例达50%。通过举办讲座、学术交流和报告,加强对兼职教师高等教育理论、教学规律和教学方法的培训,提高兼职教师的教育教学能力。兼职教师积极参与学院改革发展,在专业建设规划的制定、人才培养方案修订、有效教学改革中,提出的宝贵意见得到重视和采纳,归属感和荣誉感不断增强。
案例10:实施“顶天立地”教师培训计划,助教师专业化成长
为提高全体教师的理论水平和实践技能,适应高职教育新形势,学院实施了假期充电式“顶天立地”教师培训计划。
通过整合各方培训资源,精心选择培训内容,按需施训,学院共派出342名教师,参加了17个项目的培训。所谓“顶天”就是到全国一流高校学习本专业最前沿专业理论知识,了解本专业技术发展的最新动态;所谓“立地”主要是到全国行业实力最雄厚、技术最先进的企业进行实践锻炼,重在提高教师的实践动手能力。受到广大教师的热烈欢迎,大大促进教师的专业化成长。
通过培训,解决了当前教师普遍存在的理论水平不足,实践动手能力不强的问题,极大调动了教师们的积极性,取得了良好的培训效果。
5.健全师德建设长效机制,提高师德水平
成立师德建设委员会。根据教育部《高等学校教师职业道德规范》《关于加强教师队伍建设的意见》等文件,修订了教师职业道德规范、学术道德规范,健全师德考评制度,将师德表现作为教师绩效考核、聘用和奖惩的首要内容,实行师德“一票否决制”,构建了师德教育、师德宣传、师德考核、师德监督和师德激励长效机制。
以提升教职工职业道德素质、提高学生满意度为着力点,以经常性的学习教育和“师德建设教育月活动”为基本形式,学院每年都大力开展师德建设教育活动,评选表彰在教学管理服务中涌现出来的先进典型,有力地提升了广大教师的职业道德和职业素养。三年来,共评选表彰“师德标兵”30余人,组织师德演讲比赛2次,教职工撰写优秀师德征文300余篇,1个部门获省级工人先锋号荣誉称号,2个部门获泰安市工人先锋号荣誉称号,1人获省巾帼建功立业标兵称号,1人获振兴泰安劳动奖章。
(四)推进有效教学改革,人才培养成效突出
坚持立德树人,促进学生全面发展,按照企业技术标准和岗位要求优化人才培养方案;坚持以“德高技高”的人才培养目标,“德育为先、能力为本”的教学理念,创新“德技并重,理实一体”人才培养模式;突出职业素养和专业技能培养,校企共同修订专业教学标准,共建对应专业群的课程群平台、对应专业的课程模块,构建了“平台+模块”的课程体系;遵循职业教育和人才成长规律,突出“学生主体,能力本位”,大力实施以项目为引领的课程与有效课堂教学改革,不断推进有效教学改革;按照校企合作、工学结合的要求,实施“分阶段、项目化、协同式”实践教学模式,全面提高学生实践动手能力;以创业大学为平台,着力培养学生的创新创业能力,促进创新成果形成与转化。
1. 系统优化人才培养方案 
按照秉承“进取、担当、包容、和谐”的泰山精神,实施“德技并重,理实一体”的人才培养模式,培养“德高技高”的技术技能型人才的培养目标,各重点专业在专业调研的基础上,根据区域产业发展方向,校企深度研讨,聚焦专业服务面向。通过召开专业研讨会和岗位分析会,按照市场调研→职业和岗位分析→人才培养目标和规格→构建课程体系→制定人才培养方案→运行评估→市场调研……流程,对2013级、2014级、2015级专业人才培养方案进行了修订。引进企业技术标准和职业资格标准,校企共同设计了基于工作过程的“平台+模块”的课程体系,实施“分阶段、项目化、协同式”实践教学模式和“双证书”制度,优化培养过程,实现了专业教学要求与职业岗位任职要求的逐步对接。
按照专业人才培养方案修订的原则意见,2015级将艺术素养教育、创新创业教育等纳入方案,从课程体系、课程名称、实践课时数以及公共课教学组织形式等都进行了调整,更有利于培养技术技能型人才。
2.创新“德技并重、理实一体”人才培养模式
形成人才培养模式创新长效机制,制定了《泰山职业技术学院关于进一步明确学院办学目标、定位与人才培养模式的意见》(泰职院字〔2013〕77号)《关于实施“德技并重、理实一体”人才培养模式的意见(试行)》(泰职院字〔2014〕15),不断深化专业人才培养模式改革。在此基础上,9个重点建设专业结合专业特点分别探索形成了“淡入旺出”“两线四段三融合”等各具特色的人才培养模式。
表2-16 重点专业的人才培养模式
专业
人才培养模式
机电一体化技术
行业主导、双业贯通、双证融合、项目推进
会计电算化
虚实结合,四段递进
建筑工程技术
岗位引导、以岗定教
计算机应用技术
阶段培养、能力递进
旅游管理
旺入淡出、产学结合
园艺技术
两线四段三融合
汽车电子技术
以真实工作任务为载体的“4+1+1”工学结合模式
服装设计
工学交替、能力递进
珠宝首饰工艺及鉴定
校企共育、三位一体
3.实施“课程体系优化工程”
按照课程体系开发的基本思路,即专业调研(职业调研/职业资格研究)→目标职业群→核心岗位群→典型工作任务→学习领域课程开发→学习情景(课程内容)设计→教学组织设计,突出职业素养和专业技能培养,校企共同修订专业教学标准,共建对应专业群的课程群平台、对应专业的课程模块,构建了“平台+模块”的课程体系(见下表)。
表2-17“平台+模块”课程体系框架
平台
通用素质平台课程(必修)
1.人文素养类课程模块(思政课、就业指导、人文素质等)


2.身心素质健康类课程模块(心理健康、体育与健康、军事教育与训练等)


3.基本知识素质能力课程模块(英语、数学、计算机文化基础等)

通用素质拓展平台课程(选修)
公共选修课程模块(国学、社会、科技、经济等)

专业群平台课程(必修)
设置的课程能反映各专业共同具备的知识、技能和素质等内容。
模块
专业核心模块课程(必修)
模块1
模块2
模块3
模块4
…

专业拓展模块课程(选修)
模块1
模块2
模块3
模块4
…
4.以项目为引领,深入推进有效教学改革
遵循职业教育和人才成长规律,借鉴国内外职业教育办学经验,突出“学生主体,能力本位”,进一步加强内涵建设,提高教师的执教能力,实施了以项目为引领的课程与有效课堂教学改革,不断推进有效教学改革。
2014年4月份,学院制定下发了《关于开展教师职教能力培训与测评活动的通知》《关于实施有效课堂教学改革的实施意见》和有效课堂评价表,修订了《主要教学环节质量标准》,实施“有效课堂”改革,对教师课堂教学评价的重点,由教师教的成效转变为学生的学习成效。所有专业都进行体现“理实一体”理念的“教学做一体化”教学模式改革,制定并实施符合本专业特点的教学做一体化教学模式改革方案,灵活采用项目导向、任务驱动等教学模式。推进体现“学思结合、知行统一”的启发式教学、案例教学、讨论式教学等多样化的教学方法改革,注重将道德修养、职业素养融入教学过程。
2015年,学院与行业企业共同开发紧密结合实际岗位能力要求的课程,重视优质教学资源和网络信息资源更新和利用,不断推进教学资源共建共享,提高优质教学资源的使用效率,扩大受益面。学院开发建设的课程全部上传至网络课程中心,方便学生自学和教师交流共享。改造升级后的课程在考核方式上,形式更加多样化,更加注重过程考核和技能考核,突出了职业能力导向,充分发挥了学生学习的主动性和自觉性。
2013年以来,学院建设省级精品课程7门,市级精品课程10门,院级精品课程59门。建设专业优质核心课程53门,校企合作开发课程40门,自编校本教材(讲义)25种,公开出版教材38种,评选学院正式出版项目化课改教材28种,2015年获省职业院校优秀教材评选一等奖1种,三等奖3种。
教师积极参加省级以上各类职业院校教学比赛,青年教师、信息化教学设计、信息化课堂教学、信息化实训教学、多媒体课件制作和微课等比赛,三年来获省级以上教师教学奖项总计24项。
表2-18 近三年省级以上教师教学比赛获奖统计表
等级
一等奖
二等奖
三等奖
国家级

1
3
省级
3
11
6
学院不断探索研究教育教学改革,完善了《教学改革研究项目管理办法》,每名专任教师每年都至少参与一项教学改革研究项目。设立专项资金,对人才培养模式、课程改革、教学模式、教学方法、考试方式等改革项目给予重点支持。
通过以上改革,使教学模式更加灵活高效,更好地服务于人才培养,引领教学方法、教学手段的改革;更有利于激发学生的学习动力,调动学生的学习积极性,更有利于学生的专业成长和个性发展。2014年荣获国家教学成果奖二等奖1项,省级教学成果奖一等奖1项、三等奖3项。在2015年度全省职业教育教学改革项目立项中,我院共有7个项目获准立项,共获自助经费19万元,其中重点项目2项,一般项目5项。
表2-19  学院2014年获奖的省级以上职业教育教学成果名单
项目名称
项目负责人
等级
等次
高职“分阶段、项目化、协同式”实践教学模式研究与实践
毕于民
国家级
二等奖
高职“分阶段、项目化、协同式”实践教学模式研究与实践
毕于民
省级
一等奖
高职电工电子类专业基础课行动导向教学模式的研究与实践
屈克英
省级
三等奖
高职会计电算化专业人才培养体系的研究与实践
于运会
省级
三等奖
<Java项目综合实训>课程数字化教学资源库建设与应用研究
李倩
省级
三等奖


表2-20 学院2015年省级以上职业教育教改项目立项名单
项目名称
项目负责人
资助金额
备注
以建设高职名校为目标的办学特色培育模式的研究与实践
毕于民
5.5
重点项目
基于全人教育理念的高职院校“三导师制”人才培养模式研究与实践
屈克英
5.5
重点项目
高职院校有效教学的研究与实践
马培安
2.5
一般项目
高职院校专业教学资源库建设研究与应用
李志强
2.5
一般项目
国际合作背景下高职“能力本位”课程评价体系的研究与实践—以泰山职业技术学院和启思蒙TAFE学院合作专业为例
张庆臣
1
一般项目
高职《思想道德修养与法律基础》课程设计与实施
唐木兰
1
一般项目
基于实用心理技术的高职生积极学习心理培养的实践与研究
韩  玮
1
一般项目

表2-21  教师职教能力培训与测评活动安排表
起止时间
工作内容
责任部门
2014年3月下旬
教师职教能力培训与测评动员大会
测评办公室
2014年3月-4月
组织第一批参加测评人员报名
测评办公室
2014年4月-8月
第一批参加培训与测评的教师自选一门课,完成“课程的整体教学设计”和“课程的(一次)单元教学设计”两个文本,同时准备讲解自己的教学设计、实施单元教学。
各系部
2014年9月
根据专业特点有针对性地聘请校外专家进行对口指导、培训,然后初步确定骨干队伍人选。
测评办、各系部
2014年9月
学校聘请专家组织第一批测评,从中选出校级、系级测评委员会人选。
测评办、各系部
2014年10月
组织第二批参加测评人员报名
测评办、各系部
2014年10月-12月
第二批参加测试人员进行课程设计、论证、修改,准备参加测试
各系部
2015年1月
组织第二批测评
测评办公室
2015年3月
组织第三批参加测评人员报名
测评办、各系部
2015年3月-6月
第三批参加测试人员进行课程设计、论证、修改,准备参加测评
各系部
2015年7月
组织第三批测评
测评办公室
2015年9月
阶段总结
测评办公室
2015年9月-11月
未达标者自行进行课程设计、论证、修改,准备参加最后测评。
各系部
2015年11月
组织未达标者进行测评
测评办公室
2015年12月
整体测评工作总结
测评办公室
案例11: 院领导“推门听课”,扎实推进教学做一体化教学模式改革
近年来,学院以提高人才培养质量为核心,以深化教育教学改革为重点,突出“以学生为中心,以能力为本位、以项目为载体”的整体思路,积极调整专业结构,创新教学模式,坚持产教融合、工学结合,利用信息化教学手段,实施有效教学改革,不断推进教学做一体化教学模式改革,提升学生专业技能,全面提高教育教学质量。
为更好引领全体教师重视和推进有效教学改革,促进教师队伍整体成长,提高教师教育教学水平,加快推进符合专业特点的教学做一体教学模式改革步伐,从3月13日开始,学院采取不打招呼、不提前通知、不定时间,随机“推门听课”的形式,了解教师真实课堂教学的情况。





图 2-15 学院领导“推门听课”,了解“教学做一体”课堂教学情况
案例12  师生共建C.U服装工作室
服装设计专业将“泰安市C.U服装名师工作室”和学生服装设计工作室,合并成立了服装设计专业师生共建共享的“C.U服装工作室”。构建了基于“C.U服装工作室”的“教、学、做”一体化实境式实践教学模式,为教学改革、教学研讨、校校合作、学生创业实践等提供了优势。2015年工作室参加泰山旅游产品开发大赛,荣获泰山旅游产品开发金奖1项、铜奖两项;工作室与汶上县职业中专和岱岳区职业中专开展结对帮扶活动,指导学生参加技能大赛,分别获得济宁市2015年职业技能大赛一等奖和泰安市2015年职业技能大赛一等奖。

图 2-16  师生合一的C.U服装工作室
案例13 :打造信息化智慧教室,构建数字化智慧教室
2016年,依托学院建设的共享型数字化教学资源平台,建成了信息化智慧教室。该教室为教师和学生提供了教学软件系统和硬件平台。课堂教学中,教师利用现代化数字移动终端设备,在课堂中,向学生推送文字、动画、音频、视频等数字化课程资源,让学生开展自主学习;利用该平台进行网上互动、在线考核,强化学生参与;还可利用平台的数据统计分析功能,即时掌握学生的考勤、课堂表现,加强课堂管理。信息化智慧实训室的教学模式,有效解决学生上课时参与度低、玩手机、睡觉、不认真听课、逃课等问题,保障了教学的有效性,全面提升了整个课堂的教学效率。







                                                 
 图2-17  信息化智慧教室
5.实施“分阶段、项目化、协同式”实践教学模式
按照校企合作、工学结合的要求,结合各专业特点进行实践教学改革,实施“分阶段、项目化、协同式”实践教学模式,“项目化”构建实践教学内容,“分阶段”实施实践教学进程,“协同式”进行实践教学运行与管理。实践教学内容、教学形式、教学平台与教学主体等实践教学要素相互融合,构建了“多维一体”的实践教学体系,优化了实践教学运行与管理,安排教师进行实习巡回指导,加强校外实践教学检查,提升了实践教学质量。该成果获全国教学成果二等奖,省教学成果一等奖。

图2-18 推广“分阶段、项目化、协同式”实践教学模式
6.实行信息化管理,强化顶岗实习
完善了《学生顶岗实习管理办法》,明确了学院、实习单位、学生三方的权利和义务,学院指导教师和企业指导教师职责,规范学生顶岗实习的管理与考核,将考核成绩作为学生毕业成绩的重要组成部分,提高学生顶岗实习的积极性,确保学生不少于半年的顶岗实习。
和中唐方德共同开发应用顶岗实习信息管理平台,加强信息化管理水平。2016届毕业生全面启用了实习实训管理系统,对实习计划、实习评价标准、学生实习日志、四方(企业、学校、教师、学生)联系沟通等进行信息化管理,实现对顶岗实习的全过程、即时性监督与管理。

图2-19  数字化实习实训系统

图2-20 数字化实习实训系统-教师端
7.加强大学生创新创业能力培养
学院高度重视学生的创新创业的培养和教育,把创新创业能力培养纳入人才培养方案,构建了以课堂教学为基础、以教育活动为载体、以实践锻炼为手段、以创业大学建设为平台,“四位一体”的创新创业教育体系。通过开设《职业生涯规划与就业指导》《大学生创新创业教育》等创新创业课程,开展《创新型人才培养研究》《文化创意产业人才培养研究》课题研究等方式来培养学生创新创业意识和能力;通过举办创新创意比赛和第二课堂社团活动,指导学生制订创业规划,进行创业实践,对有创业潜力的项目进行创业孵化;通过“大学生科技创新项目”,鼓励学生参与企业技术革新、工艺改造,提升学生创新实践能力,促进创新成果形成与转化。2015年,泰安市创业大学成功落户我院。近三年,市级立项大学生科技创新项目20项,获得市级资金支持总计7.4万元;院级立项大学生科技创新项目共75项,院级自助资金6.5万元。
表2-22 2014年-2015年我院市级立项大学生科技创新项目汇总表
项目编号
项目名称
项目主持人
所学专业
资助经费
(万元)
2014D097
多功能除湿器设计与开发
王彤宇
电气自动化技术
0.5
2014D098
全自动智能抽油烟机控制装置设计与开发
刘吉晓
电气自动化技术
0.5
2014D099
基于EXCEL的高职学生综合素质评价软件开发研究
周洁
会计电算化
0.5
2014D100
太阳能机械式温控器开关的设计开发
王全蕾
数控技术
0.3
2014D101
紫薇组培快繁技术
聂树松
园林技术
0.3
2014D102
“印象泰山,我定制”—DIY数码热转印产品
梅雪
装饰艺术设计
0.3
2014D103
风力发电节能灯的设计与应用研究
陈树亮
建筑工程技术
0.3
2014D104
断线式防盗报警器的设计与开发
张嘉伟
电气自动化技术
0.3
2014D105
多功能便携充电装置
周正
电气自动化技术
0.3
2014D106
网式除油机的研制
王鹏起
机电一体化
0.3
2015D105
便携式汽车除霜器设计
郑甲琪
汽车检测与维修
0.5
2015D106
基于EXCEL的县域城乡一体化评价软件开发
宋雪菲
 会计电算化
0.5
2015D107
小口径深井救援装置的研制
郭善发
数控技术
0.5
2015D108
基于泰山石敢当文化的泰山玉石雕刻
马元昊
 珠宝首饰工艺及鉴定
0.5
2015D109
智能气囊辅助轮椅研究
马洋洋
计算机应用
0.3
2015D110
微生物有机肥的制备及对黄瓜生长的影响研究
王晓燕
食品营养与检测
0.3
2015D111
神经康复治疗仪设计
郭璐璐
应用电子技术
0.3
2015D112
五自由度大型装潢灯具升降装置的设计与制作
王乾鹏
机电一体化
0.3
2015D113
汽车防涉水报警装置设计
吕向东
汽车电子技术
0.3
2015D114
家庭照明语音控制系统设计
朱孙波
机电一体化
0.3
合计
7.4
8.人才培养质量全面提高
内涵发展使人才培养质量提升成为必然。三年来毕业生就业率和企业满意度逐年提高,2015年就业率达99.9%,在134所高校中排名第2,获省高校就业工作先进集体。麦可思调查显示,学院培养毕业生就业现状满意度和职业期待吻合度明显高于全国高职院校。三年来,学院以技能大赛为抓手,“以赛促练、以赛促改”,深化教学内容、教学方法和考核评价制度改革,提升学生专业技能。建设了卓越技师培养基地,每个专业都组建了专业核心技能训练队,由企业技术能手和学校专业带头人、骨干教师重点训练,培养预备技师。实施“技能竞赛月”活动,每年开展一次面向全体学生的技能竞赛。师生们积极参加各级技能大赛,各专业屡创佳绩。师生在省级以上技能大赛中,获奖297项,其中,全国一等奖9项、二等奖19项、三等奖45项,省特等奖、一等奖48项,二等奖71项、三等奖105项。
表2-23 近三年来学院技能大赛获奖统计表
年度
省级以上
市级

特等奖
一等奖
二等奖
三等奖
优秀奖
一等奖
二等奖
三等奖
2013年

5
17
26

5
9
9
2014年
1
31
43
62

23
33
19
2015年

19
30
62

11
19
16
(五)创新体制机制,实现共建共管共赢
学院通过创新“政行企所校”联动合作机制,开展多元合作办学,对接龙头企业、携手行业协会,成立校企合作理事会、校企合作委员会以及泰安市机电职教集团、教育联盟等校企合作组织机构,形成政行企所校多元协同发展、共同育人长效机制;以现代大学制度建设为重点,合理构架内部管理体制和运行机制,积极探索内部管理机制改革,完善院系二级管理,构建责权利相结合的内部管理机制,进一步健全了人事激励考核机制,释放出内部活力;积极开展教学工作诊断改正工作,构建“4433”教学质量监控与保障体系,建立了年度质量报告制度;引入麦可思公司等社会第三方评价,全面推行“7S”管理,促进了良好校风、教风和学风形成;完善财务年度预决算、定期审计、效益分析制度,建立后勤服务体系,健全了财务后勤保障机制。
1. 构建了多元化办学体制
(1)健全校企合作组织,完善校企合作制度
①创新“政行企所校”联动的合作机制。依托市政府牵头、多方参与,泰安市发改委、经信委、国资委、财政局、教育局、科技局、人社局等部门,以及有关行业协会、企业单位和职业院校组成的泰安市职业教育联席会议,发挥在政策出台、资源整合、规划指导、资金筹措、基础建设等方面的决策、咨询、协调、监督和推动作用,扩大社会和行业、企业对学院办学的参与度;指导学院根据地方经济社会发展和产业结构升级要求,与行业、企业共同研究制定学院建设发展规划;推进学院与行业企业在人才培养、产学研结合、毕业生实习就业、科技开发等方面的全方位合作;为学院发展提供政策支持,优化了校企合作外部环境。
②健全校企合作组织机构。学院联合山东沃尔重工、泰安银行、等企业成立泰山职业技术学院校企合作理事会,制定了章程。成立由院长担任主任的校企合作委员会,主导校企合作体制机制建设,负责学院发展规划、专业建设、人才培养改革和发展等重大事项的决策和部署,每学期召开专门会议,全面指导协调学院校企合作各项工作,开展咨询指导和评议监督。委员会下设校企合作办公室,负责健全和完善校企合作制度,推进实施校企合作具体事宜;成立校企双方共同参与的“专业建设指导委员会”,围绕专业建设、校企合作、教学改革等事关学院建设和发展的重大问题进行专题研究和指导。相关系部建立校企合作办公室,开展校企合作、专业建设、人才培养方案修订、教育教学改革等,统筹规划人、财、物,明确责任、权利和义务,细化分工,规范校企合作工作运行;系部组建泰安市域职教集团,搭建人才培养、技术合作与交流、学生实习与就业平台,以“校中厂”“厂中校”建设为抓手,完善“厂校一体,资源共享”的校企合作模式,充分发挥校企双方资源优势,构建人员双向交流、校企双向服务、实践基地共建、学生培养与就业共担、技术公关项目共研、校企资源共享等共赢长效机制,共同进行人才培养方案制定、课程体系构建、专业核心课程和教材开发、教学团队建设、实训基地建设、顶岗实习管理、技术推广和产品研发、员工培养等“八共合作”,形成与企业共生共存共赢的新型合作关系。
③建立健全校企合作制度。修订了《泰山职业技术学院校企合作管理办法》,制定《校企合作工作考核指标体系》《校企合作工作委员会会议制度》,完善了《加强校企合作开展社会服务工作实施办法》《校企合作委员会会议制度》《访问工程师实施办法》等校企合作各项管理制度。
表2-24 校企合作制度一览表
序号
制度名称
1
关于调整校企合作工作委员会的通知(含《校企合作委员会会议制度》
2
校企合作委员会会议制度(见《关于调整校企合作工作委员会的通知》)
3
泰山职业技术学院校企合作管理办法;(含《泰山职业技术学院校企合作考核指标体系;》
4
泰山职业技术学院校企合作考核办法(考核指标体系)
5
泰山职业技术学院校企合作实施方案
6
校企合作理事会章程
7
关于进一步加强校企合作开展社会服务工作的实施办法
8
泰山职业技术学院校中厂,厂中校运行管理办法
9
兼职教师聘任制度
10
教师企业顶岗锻炼制度
11
学生顶岗实习管理办法
12
校外实训基地建设制度
13
教师服务企业制度
(2)践行校企双主体办学,推进校企一体化协同育人
①积极开展订单培养。坚持“严格选拔、双向选择”的原则,既考虑合作企业的条件、要求,也尊重学生及其家长的意愿,达成订单培养协议,参照具体职业岗位任职要求共同制定“订单班”的人才培养方案,实现企业、学校和学生的“共赢”;学院与泰盈科、速恒物流、百世汇通、泰安遇石记玉石制品有限公司、北汽福田汽车股份有限公司、泰山石膏股份有限公司、青年汽车有限公司合作成、南方测绘、青岛海利尔、旺旺集团等合作成立订单培养班,企业设立“校企合作奖学金”。校企合作工作在广度和深度方面有了新突破。
②校企人力资源双向交流。学院与合作企业签订协议,实现学院教师到企业实践锻炼、企业技术人员到学院兼职教学的双向交流。学院建立“兼职教师库”,企业和学院按照相关要求对挂职教师和兼职教学的企业技术人员共同管理、共同考核,业绩考核结果作为双方各自评先评优、职务评聘的依据。完善专业技术职务评聘标准,将教师参与企业技术应用、新产品开发、社会服务等作为专业技术职务评聘和工作业绩考核的重要指标;对参加企业实践锻炼的专职教师按一定比例计算工作量。
③校企合作建立教学资源共享平台。通过现代信息技术、网络技术,建设专业教学资源库和“校企合作信息管理平台”,共享精品课程、项目课程、特色教材、典型案例、教学管理制度、管理模式等教学资源,实现教学资源的网络化、开放性,为职业院校教师教学、企业职工培训、学生和社会学习者自主学习服务。
④实现实训基地共建共享。强化校内外实习实训基地的多元化功能,探索建立基地管理新模式,实现校企共同建设、共同管理、共同分享。进行“校中厂”(教学型工厂) “厂中校”建设,建成“校中厂”11个、“厂中校”16个。
(3)开展多元合作,推进产教、校企协同发展
①积极拓展合作办学空间。进一步提高校企合作的质量,扩大合作领域,深化合作层次,更好地满足“校企融合,实境育人”的要求,各系部将校企合作贯穿到人才培养方案的修订与实施之中,每个重点建设专业完成与省级重点企业(规模以上)的深度合作,实现与国家级重点企业深度合作。三年内新建73个校企合作基地。
案例14:首饰设计与工艺项目现代学徒制试点
学院与泰安市泰山玉研发有限公司、泰安市先锋自动化数控设计中心、泰安市遇石记玉石制品有限公司、北京李博生造型艺术有限公司合作的首饰设计与工艺专业现代学徒制试点,获2016年山东省职业院校现代学徒制试点项目。
学院成立工艺美术大师、行业协会专家和骨干教师等组成的专业建设委员会,以区域经济泰山玉石设计制作人才需求和学院人才培养优势为校企合作的纽带,依托校内外实习实训基地,对接泰山珠宝玉石产业链,建立起校企共赢的合作平台,拓展了校企合作渠道。聘请了国家级工艺美术大师高毅进、宋建国、李博生作为客座教授与专业建设顾问,审议本专业各年级人才培养方案,提出合理建议。分两批派4名教师赴北京李博生工作室进行为期2个月的系统学习,在此基础上重构了课程体系和课程教学内容与形式,按照玉产品加工的工作流程重新整合课程体系,建立起了以工作过程为导向的项目化教学模式,实现了玉器设计与工艺制作的综合培养目标。学院建成了大师工作室、泥塑工作室、数码雕刻工作室、e加创客工作室;通过和企业的合作,把大师和企业引入学院,校企共建核心课程、共建育人环境、共建师资队伍,在‘真业务、真环境、真过程’的情景中完成人才培养工作,形成‘校企共育,三位一体’人才培养模式。
②主动搭建合作办学平台。充分发挥政府主导作用,加强校企智力资源、人力资源、设备资源、生产资源、市场资源的整合,搭建起校企合作的平台,促进了校企深度融合。学院牵头组建泰安市机电职教集团等。新建“教学做一体化大楼”,筹建完成了集创业教育、创业实训、创业服务、创业孵化等多功能于一体的创新创业人才服务体系,泰安创业大学挂牌成立。成立泰安市第一公共实训中心,并将其打造成为本地区综合、开放、共享的学生实训中心、技能培训中心、技术研发中心和技术服务中心。与泰安市行业管理办公室及其行业协会—泰安市机械加工协会、泰安市建筑行业协会、泰安市冷链物流协会、泰安市旅游协会、寿光市蔬菜病虫害防治协会,共同签署了校行战略合作协议,共建基地,互认挂牌。
学院与山东省沃尔重工科技有限公司、泰安华泰种业有限公司、济南禾立达种子有限公司、山东禾宜生物科技有限公司、泰安大观旅游商品设计开发有限公司、泰山天成国际旅行社、泰安宝龙新时代商贸有限公司、泰安轻松表计有限公司等互设“校中厂”和“厂中校”。
③创新了校区(校行、校县)合作模式。为了增强学院服务区域经济发展和现代产业体系建设的能力,更好地发挥“示范引领,服务社会”的作用,学院与东平县政府合作成立了移民学院,与泰安市高新技术开发区签订战略合作协议,与区域内行业企业开展广泛合作。
案例15:校区战略合作模式
学院深化与省级经济开发区、山东岱岳工业园区—“青春创业园”的合作,2014年完成与泰安市高新技术开发区的战略合作任务,并与所属泰开集团、青年莲花汽车、航天特车、太和集团、尤洛卡股份等泰安市骨干企业培植工程进行全方位、多元化合作,专业涉及机电一体化、数控技术、模具设计与制造、汽车技术、应用电子、计算机、物联网、建筑工程、电子商务、会计电算化、物流管理、国际贸易等,实现规模化校企合作。
案例16:县校战略合作模式
2014年学院与东平县人民政府签署战略合作协议,为泰山职业技术学院东平移民学院揭牌。东平移民学院的成立,是双方本着优势互补、资源共享、互惠共赢、共同发展的办学理念,在实践中创新合作办学的新探索,旨在把移民学院办成移民群众创业就业的致富摇篮,让广大移民群众特别是年轻一代掌握一技之长、实现创业梦想,开创了全省移民教育培训的新模式。
2. 合理构架内部管理体制和运行机制
(1)以制定学院《章程》为抓手,加快推进现代大学制度建设。完善了内部治理结构,理顺了内部关系,进一步明确了学院内部不同利益群体间的权责分配、角色关系及功能构建等,完善了运行管理制度对应决策与监督制度,实现了政治权利、行政权力、学术权力、民主权力制衡,为学校改革和发展提供有力制度保障。启动了对学院各项制度立改废工作,出台了新的包括102项制度的《制度汇编》,推进了现代大学制度建设。
(2)推进管理模式改革,建立了完善的考核激励机制。按照“党委统一领导,行政组织实施,责权利相统一,集权与分权相协调”的原则,构建以绩效考核为核心,充分体现人才价值,鼓励有利于教学和改革创新的分配激励机制。出台了教学、行政、后勤管理等部门的各项管理办法,严格执行领导干部公推竞岗制度,建立和完善有利于激发办学活力的人、财、物有效管理运行机制,深化了内部管理体制机制改革。
出台《院系两级管理办法》,在全院逐步推行院、系两级管理。通过明确二级管理职责权限,优化调整学院和系部两级管理职能职责以及规范管理工作流程,实现了管理权的下移,扩大系部办学自主权,激发了系部办学活力。
按照“适岗适位、结构合理”原则,实行全员聘任制。2013年实行了干部公推竞岗任用和教师(含工人)岗位竞争聘用制度。坚持 “工作项目化、管理精细化、服务人性化”的原则,按照“过程监督、目标考核、效能综合评价”的要求,进一步细化各岗位工作职责,落实责任。完善了《工作效能责任追究管理办法》《年度考核实施意见》,确保学院干部和教职工正确、高效行使职权,提高工作质量,进一步完善了有利于人员合理流动、智慧潜力得到充分发挥的良好用人机制。
健全了专业技术职务聘任、晋级、考核制度。制定了《专业技术职务评聘办法、晋级聘任办法》,做好职称评聘工作,向教学一线倾斜,将指导学生情况纳入专业技术职务评聘的重要指标。2014年在全市率先完成了专业技术人员晋级,2015年对等级聘任和新聘任教职工工资进行了调整。
完善了《师生员工奖励管理办法》,开展了先优推荐评选工作。定期评选表彰在师德建设、教书育人、教学改革、技术研究中做出显著成绩的教师,充分调动教职工的工作热情与积极性。对获得优秀的教职工、参加技能大赛和质量工程建设获奖的集体和个人进行了表彰奖励。2013年选拔了首届101名递进人才,并为每名递进人才发放了津贴,极大调动了教职工的工作积极性。
全面推行“7S”管理机制。落实“服务”价值取向,学院引入企业精细化管理理念,成立了“7S”推行委员会,制定下发了《泰山职业技术学院关于推行“7S”规范管理的通知》《泰山职业技术学院“7S”实施方案》《关于开展课前课后两分钟“7S”活动的通知》。分类别明确了办公室、实验实训室、教室、宿舍等主要场所“7S”管理基本要求及考核评分要点;各系部结合实际情况,制定制定了推进实施细则,明确、规范标准,制定计划,责任到人。将“7S”管理落实到每个岗位、每个办公区域、每位师生,全面推行“7S”管理。学院“7S”推行委员会发布检查及整改通报14个,使企业精细化管理理念深入校园,创造了整洁有序、节能高效、安全和谐的环境,促进了师生自我管理、服务、发展和良好校风、教风、学风形成。
3.建立健全教学质量监控与保障体系
扎实推进“432”质量监控与保障体系建设,积极开展教学诊断与改正工作,制订《泰山职业技术学院内部质量保证体系诊断与改进实施方案》。学院入选全国教学工作诊断与改进试点校。
(1)明确了四个监控主体
建立了多元化的教学评价保障机制,进一步明确政府、行业企业、学校、社会公众四个监控主体。充分发挥学院校企合作委员会和专业建设指导委员会的作用,对校企合作课程开发、教材开发、教师挂职锻炼、兼职授课、实践教学、社会服务等进行全程指导和调控。通过教学管理制度修订、年度人才培养质量年度报告发布、完善人才培养工作状态数据采集与管理平台,加强专业教学资源库平台和实习实训的信息化管理,形成自我检测、自我修正、持续改进的教学质量监控和保障机制。建设期间,完成了2013-2015年专业评估数据采集与上报工作,9个重点建设专业评估顺利完成。
(2)构建了三个体系
完善了目标与标准体系,教学监测评价体系、反馈与调控体系。按照专业人才培养方案修订的原则意见,每年审核修订人才培养方案。制定了课程标准。完成了9个重点专业的专业标准的制订和论证工作。
以国家及省颁发的专业教学标准为指导,依据职业岗位需求及职业资格标准,完善专业标准和专业评估指标体系。对新上专业实施每年一次、所有专业三年一轮的评估,每个专业每年形成年度人才培养状况报告,有效监测专业建设质量,提高了人才培养水平。
成立教学督导室,加强了对系部的教学督导和检查,每学期平均听课600余节次,对教学质量监控发挥了较大作用。认真落实教学信息员制度,每周汇总整理学生反映的教学情况,及时对学生反馈的信息进行汇总、分析和通报工作;督导室向系部通报教学活动及管理中存在的问题,提出整改意见并组织实施。
(5)完善了两个平台。完善了人才培养工作状态数据采集与管理平台。学院形成人才培养工作状态数据采集与管理常态化工作机制,健全组织机构,实施动态采集数据,即时分析。每年制定详细的实施方案,完成了2013-2015年三年的状态数据采集和状态数据分析报告的撰写,为学院决策提供数据支持。自2015年始全面启动网络版数据平台采集,逐步实现数据的动态管理。
加强了教学管理信息化建设。启动办公信息化OA系统,完成了教务管理信息系统升级工作,完善了管理功能。与北京中唐方德科技有限公司合作构建了技术先进、功能全面的专业教学资源库平台,与超星尔雅、中国知网等企业合作,共享共建先进的信息化教学资源,“互联网+”背景下的数字化校园建设成效初显。加强对实习实训的信息化管理。2016届毕业生全面启用了实习实训管理系统,前后举办了4期教学资源库平台、实习实训管理系统等的使用培训会,全面提高我院信息化管理与应用水平。
教师积极参加山省教育厅举办的2015年山东省职业院校信息化教学比赛,荣获信息化教学设计大赛一等奖一名、信息化实训教学比赛二等奖一名、信息化课堂教学比赛三等奖一名;并获得2015年山东省高职高专院校微课教学比赛一等奖2个,三等奖3个。
(六)主动融入地方经济发展,社会服务能力显著增强
学院紧紧抓住省会城市群经济圈重大机遇,秉承“服务”价值理念,以提升社会服务能力为重点,积极主动融入地方经济发展。依托专业优势,紧密对接泰安市国际旅游名城建设及先进制造业、现代农业等重点产业需求,加强重点专业建设,培养大批高素质技术技能型人才,夯实学院技术技能积累的基础;积极搭建服务平台,促进教师深度参与科技研发、技术改造、产品升级、职工培训和农村劳动力转移等服务,成效显著;发挥人才资源优势,着力打造泰安市人才库和智囊团,为地方经济社会发展提供强大智力支持,并积极参与区域内重大社会活动;对口支援中职院校和西部学校,示范辐射带动作用明显;在名校建设过程中形成的制度、先进理念和人才培养模式吸引兄弟院校前来考察学习;建设成果在校内外推广应用,效果良好。
1. 创新科研管理机制,突出服务社会导向
学院积极推行科研管理体制机制改革,按照“党委统一领导,行政组织实施,责权利相统一”的原则,建立了由学院统一领导、分管院长具体负责,院、系两级管理体制和绩效考评机制。科研管理部门创新科研管理运行模式、制定相关政策,搭建平台,发挥技术研发与服务中心、社会培训与职业技能鉴定中心作用,积极全方位开展技术研发、技术改造、产品升级、横向课题研究,对企业职工、退伍军人培训及农村劳动力转移等服务,全面提高学院服务经济社会发展能力,助力区域经济社会发展。先后制定完善了《泰山职业技术学院科研工作管理办法》(泰职院字〔2013〕86号)《泰山职业技术学院科研成果奖励办法》(泰职院字〔2013〕87号)《泰山职业技术学院科研工作量化考核办法》(泰职院字〔2013〕88号)《泰山职业技术学院教科研经费资助办法》(泰职院字〔2015〕45号)《泰山职业技术学院科技特派员管理办法》(泰职院字〔2015〕20号)等规定,为规范科研管理、技术服务等工作提供了制度保障,形成了激励与约束相结合的科研管理机制。
2.搭建服务平台,全面提升教师服务能力
坚持科研服务地方经济社会发展宗旨,走产学研结合的道路,依托平台整合资源,与企业合作开展科技协同创新,教师在为企业开发项目的实践中社会服务能力不断提升。
(1)加强产学研平台建设,为区域经济发展提供智力支持。学院先后成立了“泰安市现代职业教育研究院”“泰安市工艺美术学会”等研究机构。依托计算机应用技术、建筑工程技术和汽车电子技术等重点建设专业,通过整合校企相关研发和人才资源,建设了软件开发、建筑工程质量检测、汽车检测等技术研发与服务中心3个技术研发中心,面向泰安市中小微企业开展企业关键技术、共性技术研发,为社会提供技术成果、技术咨询、产品检测鉴定和人才培训等服务,为地方优势产业的持续发展提供技术支持。
依托现代职业研究院,开展泰安市文化旅游业开发研究,提升了泰安市旅游业的知名度和影响力。学院专家积极参与国家、省市旅游体制机制顶层设计、十三五旅游规划编制;牵头组织了省市旅游产业对经济、财政和就业贡献的研究,全域旅游开展、乡村旅游开发研究;参与了市旅游总体规划、市旅游营销策划的设计、论证、评审;参与了肥城和岱岳乡村旅游规划论证、市电子商务设计规划评审以及徂徕山汶河景区规划建设、项目策划等,共计34个项目。为各级各部门建言献策152条,其中进入领导决策135条。《泰山旅游文化策划—以泰山百景诗书画印系统创作为载体的泰山旅游品牌的推广研究》课题,被泰安市社科联立项为2015年社科类重大课题。
名校建设期间,学院共有主持市级及以上各类科研课题90项;参与企业技术研发、技术改造等技术服务306项;发表论文282篇;取得国家专利9项;获市级及以上各类成果奖励34项;取得技术服务收入2128万元。
表2-25 名校建设期间科研成果汇总表
 类    型
2013年
2014年
2015年
立项课题
16
29
45
技术服务项目
81
103
122
发表论文
85
94
103
市级以上成果奖
10
14
10

   
图2-21 2015年社科类重大课题结题鉴定会   图2-22 泰安市现代职业教育研究院成立
(2)搭建社会服务平台。依托汽车电子专业,学院成立了具有独立法人资质的驾校,为广泛开展驾驶员培训创造了条件;依托旅游专业,成立了具有独立法人资质的旅行社,师生全面参与公司管理、运作,在提升旅游专业职业指向性的同时,提升了教师的社会服务能力。与泰安市科技局合作,26名专业技术人员被为泰安市科技特派员领导小组任命为“泰安市第一批企业科技特派员”,占全市特派员总人数的61%;与泰安市政府项目评审中心合作,学院12名专业技术人员被该中心聘为“泰安市工程项目评审专家”;政府科技特派员工作和政府工程项目评审工作的开展,为学院教师开展社会服务搭建了平台,为教师社会服务能力的提升创造了条件。
3.开展技术培训及服务,促进地方经济社会发展
坚持学历教育与非学历培训并举,全日制与非全日制并重办学思路,充分利用学院人才优势和资源优势,为社会开展各级各类技能培训和职业技能鉴定,在技术革新和工艺创新等方面为企业开展各项技术服务,为地方社会经济发展提供了强有力的人才支撑。
(1)技术培训
近年来,学院擦亮职业培训品牌,不断拓展社会服务功能,主动对接企业、行业、中职院校和社区的需求,进一步加大了企业员工的岗位技能培训、职工培训、退役士兵、学生的技能培训。增强了服务地方及区域经济社会的能力,提高了学院对地方及区域经济建设的贡献率。
依托退伍士兵培训基地、现代农业示范县农技推广人员培训基地、“阳光工程”培训基地、农村剩余劳动力转移培训基地等,承担由政府公共财政支持的一系列培训项目。三年来共承担全市退役士兵职业与技能培训500人,政府采购代理机构人员培训300人,全市基层农技人员305人,“阳光工程”培训1200人,驾驶人员2800人。选派3名管理干部担任“第一书记”到新泰、东平包村联户,帮助农民脱贫致富。
          
图2-23 退役士兵开学典礼                         图2-24 社会培训
(2)职业技能鉴定
在大力开展社会培训的同时,学院依托学院职业技能鉴定所,积极开展职业技能鉴定工作。学院共设立会计从业资格等20多个职业资格(工种)技能鉴定点,面向学生和社会人员开展技能鉴定培训完成技能鉴定10309人。我院毕业生的“双证率”达到100%。
4.承担社会考试,服务行业发展
学院充分利用校园占地面积大、考场多以及信息化先进等条件,主动承担全市各类社会化考试。三年来,共承接教育局、人力资源与社会保障局、市建委等单位组织的全国研究生统一考试、行业职业资格考试、公务员考试等各种社会考试18项,服务考生30多万人次,成为全市最大的社会考试基地,实现了学院资源共享、服务社会的宗旨。
5.参与社会活动,提升社区文化水平
发挥人才和文化高地优势,发挥泰山书院的泰山文化研究与展示中心作用,向周边社区开放学院体育场馆、图书馆等场所,举办各种学术讲座,传播泰山文化。举办了包括泰山文化、大学文化、传统文化、企业文化在内的系列文化教育讲座近30次。牵头组织泰安市“3311”品牌项目,学院的泰山名师和部分专业负责人参加全市名师送课活动。

图2-25  毕于民教授在泰山公益大讲堂作报告
6.积极开展志愿者服务活动
学院成立泰山文明志愿者服务协会,设置20多个泰山文明服务岗,与泰山区嘉德社区、泰安市科技馆、天外村游人中心、泰安市图书馆、泰安聋哑学校、泰安市社会福利院、泰山区华新社区、泰安市中心医院等单位签订志愿合作协议,定期组织科普教育、就业指导、互动展品讲解、旅游咨询、卫生清洁、服务读者、文体活动指导、心理辅导、帮助孤残儿童和“三无”老人、医疗陪护、秩序疏导等活动。在市福利院、聋哑学校、冯玉祥纪念馆、火车站等德育基地,定期组织文明服务活动。被中共泰安市委宣传部等四部门表彰为文化科技卫生“三下乡”志愿者服务先进集体,有2人荣获先进个人;被表彰为2014年度“泰安市青年志愿服务贡献单位”。近三年来,共有1万余名大学生志愿者参与到志愿服务活动中来,为街道、社区文化科技服务做出了应有的贡献。
    
   图2-26学院在大陡山村设立青年志愿服务站  图2-27“朝阳”志愿者协会获荣誉称号
7.对口支援中西部学校,服务国家发展战略
发挥名校示范引领作用,对口支援中西部职业学校,积极服务国家发展战略。利用名校建设形成理念、制度以及先进模式等建设成果,对口支援汶上、宁阳、新泰职业中专等7所中职学校。学院安排管理干部和专业带头人,对其专业建设、师资队伍建设、课程建设及教学改革进行指导。学院承担山东省援疆任务,接收培训新疆喀什地区岳普湖县的大学毕业生培养工作,累计三批共80人,现已全部顺利结业,回到新疆开始工作。同期培养新疆来访教师6人。刘洪岩副教授赴新疆喀什岳普湖县,负责当地技能训练中心技术工作。苏大鹏老师到新疆岳普湖县职业学校支教。
(七)加强就业创业指导,招生就业工作协调发展
学院鲜明的办学特色赢得了社会各界的充分肯定和广泛赞誉,吸引了广大学子报考学院,生源质量大幅提升;整合资源,成立创业大学,创新就业模式,全方位服务学生的办学理念和实践,促使就业质量不断提高,呈现“进口畅,出口旺”良好局面。
1.拓宽招生渠道,生源质量大幅提升
学院深入贯彻落实省教育厅关于深化考试招生制度改革相关文件精神,积极探索“文化素质+职业技能”评价方式。根据不同生源特点,实行春季高考、夏季高考、单独招生、注册入学和技能考试、对口贯通培养、订单班等多种形式,为各类学生提供多样化的成长成才路径,生源吸引力大大增强。与山东科技大学合作,在机电一体化专业开展“3+2”专科与本科对口贯通分段培养模式本科招生试点,实现专本对接,贯通培养;与省市内新泰、宁阳县等中等职业学校通过“3+2”模式合作招生,分段培养,促进了中高职衔接,初步构建了现代职教体系。三年来,招生人数年年递增,特别是2015年三年制大专招生人数创历史新高,连同五年一贯制等在校生突破万人大关。良好的社会声誉吸引了考生和家长的高度关注,生源质量大幅提升。
表2-26近三年学院招生录取报到情况一览表
指标名称
2013年
2014年
2015年
招生计划
2826
2945
3732
录取人数
2601
3383
4595
春季报到人数
107
180
576
单招报到人数

361
770
报到人数
2167
2785
4005
报到率(%)
83.3
82.3
87.2
最高分
516(文)
487(文)
497(文)

499(理)
502(理)
494(理)
最低分
180(文)
212(文)
292(文)

180(理)
208(理)
250(理)
2. 全方位服务就业,就业质量不断提高
贯彻落实服务学生成才理念,以优质就业为导向,以充分就业为目标,不断完善“12345”就业工作机制,加强课程建设,实施职业生涯规划和创业教育;提高创业指导师资水平,开展创业培训,构建就业服务体系,搭建就业创业平台,拓宽就业渠道。
以泰安市创业大学为依托,加强创业教育,提供创业服务,提高创业能力,推动“以创业带动就业”战略,实现创业带动就业。
通过一系列措施,学院就业质量不断提高。2013年就业率达98.39%,在专科毕业生中就业率排名第7位,在同类院校中排名第5位。2015届毕业生就业率达99.9%,在全省同类院校就业率排名第一。
案例17:麦可思公司助力学院诊断就业质量
据社会权威数据机构麦可思公司提供的《泰山职业技术学院就业质量报告2014》显示“我校2013届毕业生毕业一年后就业率为95.2%,比全国高职院校毕业半年(90.9%)高4.3个百分点。一年后月收入为3334元,比全国高职院校毕业半年后(2940元)高394元。本校2013届毕业生就业现状满意度为64%,比全国高职院校(54%)高10个百分点。”这说明我院毕业生一年后的就业率较高,就业现状满意度、职业期待吻合度明显高于全国高职院校,就业落实情况较好,且自身对就业质量的满意程度较高。
(八)大力推进国际化办学,国际合作领域逐步拓宽
为增强教师国际化意识,培养具有国际视野的优秀人才,提升学院的国际知名度,提高学院服务泰安市国际化旅游名城建设的能力,学院确立国际化办学战略,大力推进国际化办学,不断拓展国际合作领域。先后与澳大利亚、泰国、加拿大、台湾等国家和地区的院校建立了友好合作关系,开展联合培养、师资培训、留学生互访等项目合作。
三年内,与澳大利亚启思蒙学院中外合作办学项目,机电一体化技术和建筑工程技术2个专业招生121人;与加拿大荷兰学院、韩国世明大学等7所国外高校建立友好合作关系;先后派出84名教师到国外或境外学习培训,使有国际教育背景的教师达到20%以上;中泰“3+2”专本连读合作办学项目启动招生;56名学生先后赴台进修交流。国际交流与合作,为多元化办学开辟了新的空间,提供了广阔平台。
表2-27  建设任务完成情况统计表
序号
建设内容
建设目标
完成情况
完成率
1
友好学校(所)
5 
7 
140%
2
合作办学(所)
2 
3 
150%
3
师资培训(名)
103 
211 
205%
4
学术交流(次)
2 
5 
250%
5
互派留学生(名)
40 
56 
140%
表2-28  合作办学统计
合作时间
合作办学院校
合作办学专业
2013.10
澳大利亚启思蒙学院
建筑与建造、机电一体化两个专业
2014.09
泰国斯巴顿大学
工商管理、物流管理、航空服务等三个专业
2015.08
加拿大荷兰学院
会计电算化专业



表2-29  师资培训统计表
序号
培训人数
培训时间
培训地点
培训内容
1
26
2013.05
泰山职业技术学院
日本动漫设计专家应邀来我院举行专题讲座,对计算机专业及其他相关专业教师26人进行培训。
2
1
2013.06
澳大利亚北墨尔本TAFE学院
教务处张庆臣副处长到北墨尔本TAFE学院进行教师培训和交流访学。
3
3
2013.07
山东省电力学校
首批3名中澳合作办学教师完成澳大利亚TAFE四级资格证书国内培训。
4
3
2014.01
澳大利亚启思蒙学院
首批3名中澳合作办学教师完成澳大利亚TAFE四级资格证书国外培训。
5
24
2014.02
台湾昆山科技大学
学院公派第一批24名教学管理骨干教师到台湾昆山科技大学进行职业教育交流培训。
6
20
2014.07
台湾昆山科技大学
学院公派第二批20名教学管理骨干教师到台湾昆山科技大学进行职业教育交流培训。
7
3
2014.07
山东省电力学校
第二批3名中澳合作办学教师完成澳大利亚TAFE四级资格证书一期培训。
8
2
2014.09
台湾昆山科技大学
学院公派2名教学管理骨干教师到台湾昆山科技大学带队交流并参加培训。
9
40
2015.11
泰山职业技术学院
澳大利亚启思蒙学院Andrew Miller为教师做澳洲职业教育培训
10
26
2015.01
台湾昆山科技大学
学院公派第三批26名教学管理骨干教师到台湾昆山科技大学进行职业教育交流培训。
11
3
2015.01
福建电力职业技术学院
第二批3名中澳合作办学教师完成澳大利亚TAFE四级资格证书二期培训。
12
6
2015.01
郑州电力学校
第三批6名中澳合作办学教师澳大利亚TAE四级证书培训。
13
2
2015.02
台湾昆山科技大学
学院公派2名教学管理骨干教师到台湾昆山科技大学带队交流并参加培训。
14
40
2015.05
泰山职业技术学院
台湾昆山科技大学丁仁方国际长为师生做台湾职业教育培训
15
6
2015.06
泰山职业技术学院
启思蒙学院派出国际合作办学项目负责人Helen、课程开发专家Pat、Rene和Angus来我院对专业课教师进行课程开发培训。
16
2
2015.08
台湾昆山科技大学
学院公派2名教学管理骨干教师到台湾昆山科技大学带队交流并参加培训。
17
4
2016.02
台湾昆山科技大学、台湾朝阳科技大学
学院公派4名教学管理骨干教师到台湾昆山科技大学带队交流并参加培训。
合计
211



(九)加强国内外同行交流,业界影响力不断扩大
学院加强与国内外同行的交流,不断深化与同行的合作,职教话语权不断提升,业界影响力不断扩大。加入中国教育国际交流协会,为学院开展国际交流与合作拓展了全新平台。毕于民出席世界职教联盟大会2014年会,代表中国职教在专题会议上做了发言,扩大了学院影响,展示了学院良好形象和办学水平;1人当选为中国职业教育学会工作委员会专业教学研究会副主任,3人任常务理事;1人为中国高等教育学会大学生素质教育研究会副秘书长,1人为理事;1人为中国职业技术教育学会职业院校技能大赛委员会理事;2人分别当选中国职业技术教育学会教学工作委员会职业教育教学资源建设研究会第二届理事会常务副主任、副主任;1人当选山东省职业技术教育学会职业院校战略管理工作委员会理事;学院成为中国成人教育杂志理事会常务理事单位;入选了山东首批教育信息化试点单位;作为普及传播泰山文化的泰山书院成为全市第一家省级社科普及教育基地;已连续五年顺利通过省级文明单位复评;5人被全国职业院校技能大赛组委会聘为赛项评审专家、监督组长、监督员和裁判员;7人为省教指导委专家;37人被选派为泰安市科技特派员、职业资格鉴定员、政府招标专家。学院名校建设取得良好示范效应,厦门技师学院、山东城市建设职业学院等50多所院校来院交流考察有效课堂教学改革、泰山特色校园文化建设经验。学院在业界影响力不断增强。



三、重点建设专业及专业群成效
项目一   机电一体化技术专业建设
(一)建设目标完成情况
项目立项建设以来,专业团队按照建设方案和任务书完成8个二级项目,26个子项目和318个验收要点,全部任务按方案、任务书要求完成,达到预期建设目标。计划投入资金643万元,实际投入790.7万元。主要建设任务指标完成情况见表2-30。
表2-30  机电一体化专业建设项目主要建设指标完成情况
二级项目
子项目
验收要点数量
实际完成
完成情况
体制机制建设
校企合作体制机制创新
14
14
100%

校企合作制度建设
16
16
100%

内部管理、激励机制建设
11
11
100%

质量评价与保障体系
17
17
100%
人才培养模式与培养方案
人才培养优化
12
12
100%

人才培养模式创新
12
12
100%

顶岗实习管理
9
9
100%
课程体系构建与核心课程建设
课程体系构建
14
14
100%

课程建设
9
9
100%

教材建设
9
9
100%

教学资源库建设
9
9
100%

教学模式改革
17
17
100%
教学团队建设
专业带头人培养
18
18
100%

骨干教师培养
12
12
100%

双师素质教师培养
9
9
100%

兼职教师队伍建设
7
7
100%
教学实验实训条件建设
校内实训基地建设
18
18
100%

校外实训基地建设
11
11
100%
社会服务能力建设
社会培训
15
15
100%

技术研发与服务
8
8
100%

技能鉴定
18
18
100%

辐射带动
12
12
100%
专业群建设
实训基地建设
3
3
100%

师资队伍建设
18
18
100%

课程体系建设
8
8
100%
专业文化建设
专业文化建设
12
12
100%
(二)项目建设成效
1.建成校企合作体制机制,实现互惠双赢
(1)成立机电一体化技术专业建设委员会,完成内部管理机制建设
以合作办学、合作育人、合作就业、合作发展为主线,依托泰安市机电职业教育集团,联合山东沃尔重工科技有限公司、山东蓝田工程机械科技有限公司等企业成立机电一体化技术专业建设委员会。
在学院名校办和学校领导的指导下,系部分别成立了专业建设小组、项目建设质量控制小组和党员领导小组。
(2)校企合作制度更加健全
依托泰安市机电职教集团,先后与47家企业建立了长期稳定合作关系。聘请企业专家为专业建设委员会成员共同制定完善了《校外实训基地建设和管理办法》、《学生顶岗实习管理办法》、《教师企业顶岗锻炼实施细则》、《教师服务企业管理办法》、《兼职教师聘任和管理办法》等校企合作制度,增强办学活力。
(3)内部管理、激励机制建设更加完善
严格落实学院的《师生获奖奖励办法》、《人才递进工程实施意见》、《教学运行管理办法》、《教学质量评价办法》、《教学档案管理办法》、《兼职教师管理办法》;修订了教师、职工、学生考核评价办法;落实相关配套制度,设立兼职教师岗,鼓励校企人员双向兼职,将服务企业纳入绩效考核并作为专业技术职务评聘指标。完善了内部管理制度与考核激励机制。
(4)构建教学质量评价与保障体系,全面提高人才培养质量
召开6次专业建设委员会会议,根据会议讨论决议,在学院教学监控体系的基础上,结合系部实际,改革《教学考核办法》,构建教学质量评价和保障体系。
2.完善人才培养模式,成效显著
(1)深入企业调研,不断进行人才培养方案优化
①完成调研报告,把握需求,准确定位产业发展。
②企业参与,建立人才培养方案的优化机制,不断完善人才培养方案。
(2)不断创新人才培养模式
以岗位任务为核心,开发了项目化专业课程;按照企业的真实岗位设置了教学环境并组织教学。通过基本职业素养培养、专业核心模块、专业拓展模块、顶岗实习、毕业设计五个阶段的教学实施,逐步提高学生的职业能力,同时将职业道德与职业素质培养贯穿始终。
 (3)完善顶岗实习
学生前2.5年在学校完成基本知识学习和基本技能训练后,依托校外实习基地,在后半年开展“集中顶岗”实习,对学生进行全面、全程职业能力和职业素质训练。
3.构建“平台+模块”的课程体系,日趋完善
(1)构建“平台+模块”的课程体系
实施课程体系优化工程,以培养职业能力为主线,以工作过程系统化的课程改革为重点,结合“项目为导向,教学做一体”的教学模式,构建了以能力为主线、双证融合的“平台+模块”课程体系。
(2)专业核心课程资源建设
由企业技术骨干和专业教师成立学习领域开发建设团队,依托企业和校内实训基地,围绕专业核心能力,设计课程内容,建立了机电一体化技术专业教学资源库,重点开发了6门核心课程的数字化教学资源。
(3)专业教材开发
以行业要求和职业标准为依据,以强化技术应用能力培养为重点,按照易教、易学、符合岗位实际需要的原则,把行业的新知识、新技术、新工艺、新方法固化到教材中,校企合作开发教材10部。
(4)完成专业教学资源库建设
采用引进与自主开发相结合、动态更新积累的方式,建设了以“核心课程”为主的专业教学资源库。资源库建设既符合了高职教育特点,又突出了本专业及其所属行业特色。重点建设了6门专业核心课程数字化教学资源库。
(5)教学模式的改革
推行了“项目为导向,教学做一体”的教学模式,以“项目教学”为载体,推动了“理实一体”的教学做一体化教学模式改革。教学项目全部来机电设备安装、调试、操作、维护、维修等岗位的真实案例,并将项目与人才培养要求的素质、知识、技能目标结合起来,形成具有真实生产情境的教学项目。
(6)考核改革
考试考核结合课程的性质和特点,采用综合测试、调研报告、论文答辩、开卷考试、现场技能操作、上机操作、实验测试、产品制作、竞赛等形式,并加强学习过程考核,加大其考核比例。将职业资格证、上岗证等证书的考取列入考核范围,重视考核后信息的分析、处理和意见反馈。
(7)开展卓越技师培养
以赛促学,以赛促教,纳入整个教学体系。规范技能大赛将电气控制、机电设备维修、机电产品创新大赛等列入人才培养方案,组建学生技能比赛队伍和专业骨干教师训练团队,将技能大赛普及化、制度化、规范化。获得市级以上奖励65项,每年11月开展了“技能竞赛月”活动。
4.内培外聘、专兼结合,建设“双师型”教学团队
通过参加国内外师资培训、学术交流和研讨、企业顶岗、挂职锻炼等方式,培养名师3名,教坛英才6名。从行业、企业聘请专家作为兼职教师,优化了教学团队师资结构。
(1)专业带头人素质不断提升,引领带头成效明显
按照《专业带头人培养方案》,通过参加学习、学术交流、研讨会等形式对专业带头人进行培养,提高其科研和实践教学能力。
落实双带头人制度,新培养校内专业带头人辛太宇,聘请山东沃尔重工科技有限公司总工程师赵德正作为校外机电一体化专业带头人。
(2)骨干教师队伍日益壮大
培养了13名骨干教师,建设成一支具有“真才实学,德才兼备”的教师队伍。
(3)“双师”素质教师能力稳步提升
安排6名教师脱产进行为期半年的校外实训基地顶岗挂职。全部专业课教师,利用寒暑假进企业顶岗锻炼每年至少1个半月。鼓励专业教师参加本专业相关高级技能鉴定,专业教师“双师型”比例已达到100%以上。
(4)通过多种形式的培训,提升教师能力
教师进行国内外、境外培训,国外(境外)培训专业教师13人次、省级及以上培训80人次、澳大利亚教师资格证培训(TAEE)5人次。
(5)兼职教师队伍建设,建成兼职教师资源库
建立兼职教师库,新聘12名企业内有丰富实践经验的专家、工程师作为兼职教师;新聘12名企业一线高级技师为兼职教师指导学生顶岗实习与毕业设计。
(6)以科研带动教研、以竞赛推动教学,建成一支“四能”教师队伍 
开展山东省高校科技计划项目《基于无线传感器的泰山名贵药材人工培育生长参数测控平台设计与实现》等5项省级课题,《基于CAD/CAM/CAE的模具专业课改革的研究》等市级课题9项,发表论文33篇。
参加各级技能大赛,成绩斐然,其中国家二等奖1项,国家三等奖1项,省级二等奖1项,省级三等奖2项,市级一等奖2项,市级二等奖4项,市级三等奖5项。
5.厂校携手、优势互补,共建校内外实训基地
(1)校内教学实训基地建设成效
建成了融教学、培训、技能鉴定、技术研发等功能于一体的校内实训基地。在原有实训条件基础上,新建、改建、扩建14个实训场所。可满足高级维修电工、高级焊工和高级数控机床操作工等的培训工作,承担本专业(群)实习实训等教学任务。增设大学生科技创新中心。
校内实践课程开出率100%,同时可以开展数控车工、维修电工和焊工职业技能鉴定,校内毕业生双证书获取率达100%。
(2)校外实训基地建设成效
在原有实训基地的基础上,再增加长期稳定、可进行顶岗的实习校外实训基地7家,确保所有学生毕业前的校外顶岗实习不少于6个月。与8家企业建立密切合作关系。在青岛DND集团胶州分公司,潍坊歌尔声学有限公司,山东沃尔重工科技有限公司建立“厂中校”,与山东泰山科技股份有限公司共同建立“校中厂”,实现校企共建。 
6.产研对接、资源共享,社会服务能力提升
成立机电技术服务中心,推进产学研结合,以“2011年泰安市十大优秀专利发明人”李琦为主,成立专利技术创新与推广工作室。完成应用研究与技术开发服务项目2项。重点建设专业技术服务项目50个,横向课题6个。
(1)开展社会职业技能培训 
发挥专业实训基地的优势,为泰安及周边地区输送技术型、高层次技能型机电装备与制造专业(群)人才的同时,为企业职工提供技能提升培训服务;扩大城镇下岗再就业和农村劳动力转移培训项目,根据受训人员意愿,开发适用、有效的培训课程,科学实施,完成技能培训。
(2)加强与同类院校的对口支援与交流工作
通过联合培养学生、教师培训与交流等,辐射和带动泰安及周边职业院校教育发展,与宁阳职教中心、新泰职教中心和胶州职教中心建立长期的校校合作关系,与汶上县职教中心实施跨区域专业结对帮扶。无偿开放实验室并提供优质教学资源;对泰安县市职教中心的师资进行培训;为受援学校专业建设、课程建设、教材建设等方面提供帮助;选派三名专业教师到县、市职教中心对口支援。
7.带动作用凸显,专业群快速发展
(1)课程建设和新专业申报成果显著
以机电一体化技术专业建设为重点,同时带动数控技术、模具设计与制造、焊接自动化专业的课程建设,建设完成4门核心课程。建设完成相关专业的专业课、基础课的课程标准、教案、整体设计、单元设计等相关课程资源。
根据当前科技发展和市场需求,积极拓宽专业方向,成功申报工业机器人技术专业。
(2)师资队伍力明显加强
组织专业骨干教师利用寒暑假到泰航特车、泰和电力、泰开集团等企业顶岗锻炼63人次,分别参加厦门大学、山东职业学院等省级及以上培训45人次。数控技术专业、模具专业培养骨干教师共9名。
(3)实训条件明显改善,技术服务能力明显提高
共享机电一体化专业校内校外实训基地,建成数控加工仿真实训室。开展自动化生产线维护培训、数控维修培训、维修电工职业资格培训与鉴定;多次为泰安市职业技能竞赛等技能大赛提供了设备和技术保障。
8.校园文化建设成果斐然,师生精神面貌焕然一新
(1)在实训区、教学区走廊,教室墙壁张贴能体现机电专业文化的文字及图片。
(2)完善“7S”管理制度,使企业精细化管理理念深入办公室、教室、实训实验室,学生宿舍等场所。
(3)开展了多种富有专业文化色彩的主题活动。引导和指导学生建立和发展学生社团7个。每年3月和11月积极开展森林消防防火活动等。
(三)特色与创新
1.学生创新能力显著提高
现有机电设备维修协会、机电创新小组、CAD/CAM协会等社团5个。2013年参加全国大学生信息技术竞赛获二等奖2项,2014年第十届山东省机电产品创新设计竞赛获二等奖2项,三等奖4项,2015年第十一届山东省大学生机电产品创新设计竞赛,取得二等奖1项,三等奖4项。2015年获全国三维数字化创新大赛龙鼎奖。
参与泰安创业大学建设,组织有潜力的项目入驻大学生创客空间。同时以大学生科技创新为载体,把创新创业教育纳入学生培养方案、不断推动教学改革。2013年以来,已经成功申报泰安市大学生科技创新计划5项,院级大学生科技创新计划14项,获得经费5万余元。
2.实训条件明显改善   
新增校内实验实训室14个,实训设备223台套,实验工位共562个,实训面积1390m2;校外实训基地7处。利用基地设备,开展校企合作产品开发,进行社会培训,技能鉴定,产生广泛的社会效益。通过技术服务,社会培训,提高了教师实践能力,展现了教师风采,提高学院的社会影响力。
3.办学实现多层次化
与山东科技大学成功进行了合作办学,成立了“3+2”机械设计制造及其自动化试点本科专业,共同培养本科层次学生。
与澳大利亚启思蒙学院开展合作办学,成立机电一体化技术“中澳合作班”,开展国际化办学。
与昆山科技大学进行合作办学,每年派出机电一体化技术专业学生去台湾昆山科技大学学习交流。
与泰安市5所中职学校实施五年制职业教育“三二分段”培养,拓展了办学模式和培养模式。
与汶上县职教中心实施跨区域专业结对帮扶,挑选专业骨干教师对该校进行定向定期指导。
项目二 建筑工程技术专业建设
(一)建设任务完成情况
根据《建筑工程技术专业建设方案》和《建筑工程技术专业建设任务书》,项目建设共分8个一级验收项目、30个二级验收项目、348个验收点,经过三年的不懈努力,各项建设预期目标全部实现,部分指标超额完成。主要建设任务指标完成情况见下表。
表2-31  建筑工程技术专业建设项目目标完成情况一览表
序号
建设内容
关键指标
单位
预期目标
实际完成
完成度
1
体制机制建设
成立专业建设委员会
个
1
1
100%


制定校企合作制度
项
4
4
100%


制定内部管理制度
项
2
2
100%


人才培养质量调查问卷
份
-
150
净增


签订合作行业
个
1
1
100%


签订合作企业
个
6
9
150%
2
人才培养模式与培养方案
形成专业调研报告
个
3
3
100%


专业人才培养方案
套
3
3
100%


人才培养模式完善计划书
份
3
3
100%


人才培养模式改革总结
份
3
3
100%
3
课程体系构建与核心课程建设
优化后的课程体系
份
3
3
100%


制订课程标准
个
5
5
100%


建设核心课程
门
5
5
100%


建成课程资源库
门
5
5
100%


编写校本教材
本
2
2
100%


公开出版教材
本
0
12
净增


建设院级精品课程
门
5
5
100%


开展教师教学设计能力达标
人
0
24
净增


院级以上教改项目立项
项
2
4
200%


创新创业教师团队
个
1
1
100%


创新社团
个
2
2
100%


市级及以上技能竞赛获奖
个
6
42
700%


“行走课堂”项目
项
0
1
净增
4
教学团队建设
培养专业带头人
名
2
2
100%


培养骨干教师
名
10
10
100%


培养青年教师
名
6
6
100%


培养双师素质教师
名
10
10
100%


双师素质教师比例
%
70
100
142%


聘任兼职教师
名
20
20
100%


专兼职教师比例

1:1
1:1
100%
5
实训条件建设
新建校内实训室
个
3
18
600%


新增校外实习基地
个
6
9
150%
6
社会服务能力建设
技术培训
人时
19200
79000
410%


完成技能鉴定
人
600
683
113%


开展技术服务
项
10
10
100%


辐射带动中职学校
所
1
1
100%


培训中职教师
人次
20
20
100%
7
专业群建设
制订人才培养方案
套
2
2
100%


培养专业带头人
名
2
2
100%


培养骨干教师
名
6
6
100%


聘任兼职教师
名
6
6
100%


建设核心课程
门
4
4
100%


新建共享实训室
个
2
18
900%
8
专业文化建设
实训室融入优秀企业文化
处
1
1
100%


文化长廊
处
1
1
100%
(二)项目建设成效
1.创新体制机制,增强专业建设和发展活力
三年的建设工作,我们所取得的主要成绩可以概括为“一个特色、一个成果、一个突破”。
一是走出“一个特色”:培养“重德重技、以岗定教、短训强化、岗教融合”的专业特色;二是取得“一个成果”:全面实现了“人才共育、资源共享、互惠共赢”的校企合作新局面;三是实现“一个突破”:通过实训基地建设,解决了学校设备、兼职教师不足等难题。
(1)成立建筑工程技术专业专业建设委员会,提升了专业建设水平。
(2)修订完善了校企合作的制度,促进校企合作的健康发展。
(3)构建了由行业、企业、学校、学生多方构成的质量评价体系,实施了人才培养质量评价,为人才培养方案优化、课程建设、教学改革提供了科学的依据,提升了专业建设水平。
2.推进人才培养模式改革,提升人才培养质量
(1)人才培养方案以培养职业能力为核心、以培养创新能力为提升、以强化人文素养为根本,更加适应经济转型升级背景下对高素质技术技能型人才的培养要求。
(2)深化“岗位引导、以岗定教”的人才培养模式改革
①实施了项目化课程教学设计,教师职教理念和课程设计能力有了较大提升。
本专业对16门课程进行了项目化课程整体教学设计,建筑工程技术专业教师全部通过学院教学设计能力达标考核。 
②工学结合实施的条件得到极大完善
加强了校内外实训条件建设,增加18个校内实训室,9所校外实训基地。校内专任教师双师率达到了100%,为工学结合的实施创造了条件。
③完善了人才培养模式改革的激励机制、考核机制
修订《教学质量评价办法》和《教职工绩效考核办法》、《教学督导实施办法》三个考核制度,加强对课堂项目化教学、实践教学、双证书考核培训、课程建设的引导与评价。
④以学院卓越技师培养平台为依托,实施五项技能大赛,以赛促学,加强了学生五项职业核心能力和创新能力培养。
建立了五项大赛组织机构,完善大赛实施方案,达到了以赛促教、以赛促改、以赛促学、以赛促进校校交流、校企合作的目的。学生先后获得国家级三等奖2项,省级奖项37项,其中一等奖14项。
⑤成立了两个专业社团:工程测量社团、造价新星社团。
⑥推行了双证书制度,增强学生就业的针对性,提高就业竞争力。  
⑦推进了课程考核方法改革,加强实践技能的考核。
⑧开展第二课堂活动,加强学生创新能力培养。
开展学生建筑创新设计大赛等活动,丰富了学生的生活,激发了学生灵感和创新思想,提高了他们的信心。
3.课程建设更加适合人才的培养和学生终身发展的需求
(1)构建了“平台+模块”的课程体系。
(2)课程建设水平明显提高
建设期内,校企合作制订了5门专业核心课程标准,建设了5门院级精品课程;编写了2门校本教材,公开出版教材12本;完成了5门核心课程资源库建设。
(3)实施了项目化教学改革,推进了“项目导向、任务驱动”等教学方法的实施,课堂教学效果显著提高。
建设期内,完成了1项省级教学研究立项、3项院级教改课题立项,完成了5门课程整体教学设计,10位专业教师通过了学院项目化教学课程设计达标考核,教师的教学改革能力显著提高。
(4)学生顶岗实习实现规范化、信息化管理
4.建设一支专兼结合的优秀教学团队
建设期内,选派3人参加出境培训,3人次参加国培项目,10余人次参加省培项目,40余人次参加了行业技术及其他培训。双师素质的专业教师比例达到100%,建立了50人的兼职教师资源库,专兼职教师的比例达到1:1。
5.实践教学条件显著改善,职业能力训练体系得到完善
新建建筑施工工种实训室、建筑工程技术资料实训室、建筑工程服务工作站等实训室或实训场所18个,实训开出率达100%。拓展了校外实训基地,增建9家校外实训基地。制定和完善了《建筑工程系校内实训室管理办法》等有关制度。
6.社会服务能力显著增强
先后有5名教师深入到企业中去为建筑类企业开展技术服务。
建筑工程检测中心能够进行建筑空气质量检测、建筑材料质量检测,路基路面质量检测等服务项目,拓展了我专业的社会服务范围。
完成建筑类职业资格培训7000人时、泰山学院学生建筑施工技能培训24000人时、退伍军人建筑技能培训48000人时。
7.发挥重点专业的示范带动作用,带动专业群协调发展
(1)建筑工程技术专业的建设极大带动了工程造价专业和道路桥梁工程技术专业的建设和发展,改善两个专业办学条件,提高了教师的教学能力和课程建设水平,在实训条件、师资队伍、教学资源等方面专业群实现资源共享,优势互补。
(2)专业群专业建设水平有了较大提升,优化了人才培养方案,推进了“岗位引导、以岗定教”的人才培养模式实施,教师的实践能力、教学能力有了较大提升,课程建设水平明显提高,人才培养质量显著提高。
8.建设鲁班文化,构建“泰山特色”文化育人环境
(1)将“成熟稳重”的泰山文化融入到以“质量和安全重于泰山”为核心的专业文化建设中,将以绿色建筑、节能建筑等为时代特征的建筑企业文化要素融入到人才培养模式改革的各个环节,实现专业文化建设与建筑企业文化的对接。
(2)充分利用实训基地、教学楼走廊,将企业文化、技术标准、安全标准、精品样板展示出来,让学生了解企业管理理念、发展理念。
(三)特色与创新
1.创新教学过程“四化”教考新体系
即:核心课程教学标准化(制定统一的课程标准,统一的教学方案,统一的授课模式,统一的工程案例,统一的教学课件);专业操作技能系统化(健全了专业实践教学体系,各课程技能训练项目前后连贯,能力训练逐步深入);课程教学资源通用化(各课程采用同一工程案例,各课程教学资源可以贯穿使用);毕业素质考核统一化(制定了统一的毕业生素质技能鉴定系列考核标准,毕业考核不因教师、班级不同而有所改变)。
2.创新“行走课堂”流动学习新模式
充分利用现代信息技术,与企业合作建立学生“行走课堂”,利用我院综合实训楼各实训室、走廊、墙壁、梁、柱以及楼宇内各建筑构件,将专业知识点分布在不同的建筑部位,学生可以在楼内行走时,通过二维码相嫁接,通过互联网和手机客户端,于不同的位置通过手机平台扫描“二维码”获得该建筑部位的相关建筑知识,拓宽了学生获取知识的途径,提高了学生的学习兴趣,增加了信息量和直观性。
3.创新“互惠共赢、资源共用”校校合作新体制
加强与本地高校的相互合作,做到师资、实训设备等共享共用,在技能大赛训练、教学实训等方面相互合作,相互支持。山东科技大学、山东农业大学、泰山学院多次派教师到校授课或进行训练指导,学生的实践能力、团队合作能力、创新能力有了很大的提高,我系各项大赛成绩突飞猛进,其他院校学生多次利用我院虚拟仿真等实训系统学习、实训。
4.创新“虚实融通、内外结合”的实训新模式
投资100多万元,建设建筑识图、手工算量、建筑施工、建筑测量等虚拟仿真实训室,核心技能操作实现虚实结合双重实训模式,虚拟实训技能全方位覆盖,实际操作重点加强,室内外校内外结合,学生技能水平大幅度提高。

项目三  会计电算化专业建设
(一)建设任务完成情况
项目自立项以来,专业团队按照建设方案和任务书要求,完成了28个子项目,337个验收要点,实现了预期建设目标。该项目预算投入资金627.8万元,实际投入资金1114万元,资金执行率为177%。在完成的42项关键考核指标中,超额完成13项。主要建设任务指标完成情况见表2-32。
2-32 会计电算化专业主要建设任务指标完成情况一览表
序号
建设内容
关键指标
单位
预期目标
实际完成
完成度
1
体制机制建设
校企合作办公室
个
1
1
100%


校企合作管理制度
个
8
8
100%


内部管理制度
个
3
3
100%


合作行业数
个
1
1
100%


年度质量报告
份
3
3
100%
2
人才培养方案优化与人才培养模式改革
专业建设委员会
个
1
1
100%


人才培养方案优化机制
项
1
1
100%


集中专业调研
次
3
3
100%


专业调研报告
份
3
3
100%


人才培养方案
份
3
3
100%


人才培养模式改革方案
份
3
3
100%


人才培养模式改革总结报告
份
3
3
100%
3
课程体系构建与核心课程建设
人才需求分析报告
份
3
3
100%


“平台+模块”课程体系
个
1
1
100%


优质课
门
8
8
100%


省级精品课
门
4
4
100%


院级精品课
门
1
9
超额


制定课程标准
门
8
22
超额


编写教材(专业课)
部
5
10
超额


    其中:公开出版
部

10
超额


会计电算化专业教学资源库
个

1
超额


数字化课程资源
门
8
8
100%


课改测评达标人数(专业群)
人

55
超额


技能大赛获奖
项

47
超额
4
教学团队建设
聘任专业带头人
人
1
1
100%


培养校内外专业带头人
人
2
2
100%


培养骨干教师
人
14
14
100%


培养双师素质教师
人
6
6
100%


培养青年教师
人
4
4
100%


教师参加国内培训
人
26
60
超额


企业顶岗锻炼
人
26
60
超额


兼职教师资源库
人
26
61
超额
5
实训条件建设
新建改建校内实训室
个
12
12
100%


新增校外实习基地
个
8
8
100%
6
社会服务能力建设
技术服务
万元
6
8
完成


社会培训收入
万元
360
367.9
完成


辐射带动中职学校
所
1
3
超额
7
专业群
人才培养方案
份
3
3
100%


制定课程标准
门
8
18
超额


院级精品课群
个
1
1
100%


公开出版教材(通用素质平台)
本

2
超额


培养骨干教师
人
12
12
100%


共享实训基地
个
7
7
100%
(二)项目建设成效
1.校企共育走向深入,管理水平显著提升
(1)形成了“政、行、企、校”联动的校企合作体制机制,校企共育走向深入
经过三年的建设,成立了由泰安市财政局会计科、会计师事务所、企业、学院四方共同参与的校企合作办公室;逐步形成了“政、行、企、校”联动的校企合作体制机制;校企在实习就业、课程开发、教材编写、人才培养方案制定等领域,取得丰富建设成果,校企共育走向深入。
(2)内部管理规范日臻完善,管理水平显著提升
在项目建设期间,不断加强内部管理,规范教学秩序,全系管理水平显著提升。首先构建了财经系教学质量监控与保障体系,提升了教学质量;其次,对教学常规、毕业生信息管理等工作实行了信息化管理,提高了工作效率;第三,在教学区、实训区、办公区、宿舍区等实行了“7S”管理,规范了师生学习工作行为;第四,制定了《财经系师生员工奖励管理办法》,完善了激励机制。
2.专业对接产业更加紧密,学生职业能力不断提高
(1)形成人才培养方案优化机制,专业对接产业更加紧密
三年间,组织教师到30多家企业,开展了3次集中专业调研,先后组织召开3次人才培养方案优化论证会,优化完成了2013级、2014级、2015级人才培养方案。
(2)人才培养模式日趋成熟,学生职业能力不断提高
探索实施了“虚实结合,四段递进”的人才培养模式,学生职业能力不断提高。“虚实结合”指将校内虚拟实训和岗位实际操作相结合,培养学生的职业能力。“四段递进”是指按照基本能力培养阶段、专业及专业拓展能力培养阶段、综合能力培养阶段和就业创业能力培养阶段进行递进培养。同时,德育教育和人文素质教育贯穿人才培养的全过程。
3.课程体系全面优化,课程建设成果丰硕 
(1)实施“课程体系优化工程”,构建了“平台+模块”课程体系
依托人才培养方案优化机制,在专业建设委员会指导下,确定了专业的初始化就业岗位、专业发展岗位、专业拓展岗位。围绕岗位群,提炼了32项典型工作任务,凝练出18项核心职业能力,针对核心职业能力确定了16门主要专业课程。同时结合岗位群对基本知识、基本素质、基本能力的要求,确定了基本素养课程。根据专业群的课程需要,将课程划分为通用素质平台课程、通用素质拓展平台课程、专业群平台课程、专业核心模块课程、专业拓展模块课程,最终构建形成“平台+模块”课程体系。
(2)参照学院“分阶段、项目化、协同式”总体实践教学模式,构建了“四段递进”的实践教学体系
第一阶段,基本能力培养阶段。本阶段使学生掌握会计基本技能和会计核算基本方法,为专项技能实训打下基础。
第二阶段,专业及拓展能力培养阶段。本阶段使学生掌握专业岗位技能,为综合实训打下基础。
第三阶段,综合能力培养阶段。本阶段使学生具有会计的综合账务处理能力。
第四阶段,创新与就业创业能力培养阶段。本阶段通过企业经营管理沙盘实训、顶岗实习等环节,培养学生创新能力和就业创业能力。
(3)以教改促课程建设,取得丰硕成果
三年间,4门课程被立项为山东省精品课程,1门课程被立项为市级精品课,9门课程被立项为院级精品课程;编写12部教材,公开出版10部,由赵秀荣老师主编的《基础会计》教材在高等教育出版社出版,并在山东省优秀教材评选中荣获一等奖;校企合作开发完成8门课程的数字化资源。
(4)建设完成了数字化教学资源和专业教学资源库
建成了5门省级精品课的教学资源,教案、课件、资料等已全部上网;校企合作,共同完成了8门课程数字化教学资源;完成了会计电算化专业专业教学资源库。
(5)形成了“项目导向、任务驱动”的教学做一体化教学模式
首先,改革教学方法。在有效课堂推进中,突出“学生主体,能力本位”理念,多采用小组学习法、案例教学法、角色扮演法、模拟教学法等多种教学方法,调动学生学习积极性。同时对会计综合实训课程,采用“双讲师”制,由专业教师和兼职教师共同指导实训。
其次,改革教学手段。利用虚拟实训环境,开展教学,提高学生的专项技能和综合技能;依托学院的数字化校园建设,充分运用动画、音频、视频等现代化的多媒体教学手段开展教学,提高教学直观性、生动性。
第三,改革考试评价方法。结合课程实际,选用以证代考、能力测试等多种考评方式;课程考核中增加实践考核比例。
结合上述改革理念,《基础会计》、《财务软件操作》等17门专业课完成了项目化课程整体教学设计和至少一个单元的教学设计,并通过了学院职教能力达标测评。
(6)多措并举,大学生创新与就业创业能力明显提升
首先,优化课程体系,调整课程设置,开设《职业生涯规划与就业指导》、《大学生创新创业教育》等课程。帮助学生树立正确的人生观、就业观、择业观、创业观。其次,以“大学生科技创新项目”为载体,鼓励学生积极参与课题研究,指导学生立项5项院、市级课题。同时,依托学院创业大学平台,指导学生进行创业实践。以上举措,使学生创新与就业创业能力明显提升。
(7)健全顶岗实习管理制度,实现顶岗实习信息化管理
制定《财经系学生顶岗实习管理办法》、《财经系顶岗实习实施方案》,规范管理顶岗实习工作。依托学院数字化实习实训系统平台,对2013级学生实行信息化管理,实现对顶岗实习过程的全面性、即时性监督与管理。
(8)建立卓越技师训练队,技能大赛取得优异成绩
按照学院卓越技师训练要求,组建了会计技能、市场营销技能训练队,每学期定期安排骨干教师指导训练。三年间,在学生技能大赛中,累计获奖47项。其中在山东省职业院校技能大赛中获得二等奖4项,三等奖6项。技能大赛获奖奖次和层次与名校建设前相比,有大幅提升。
4.教师队伍结构更加合理,“专兼结合”、“双师型”教师队伍形成
依托学院教师发展中心,培养了2名专业带头人,14名骨干教师,4名青年教师,6名双师素质教师。组织教师参加校内集中培训32次,参加国内培训98人次,参加境外培训13人次。通过培训,收到明显成效。3名专业课教师被评为学院名师,有55名教师通过了职教能力达标测试,发表论文72篇,主持、参与课题30项,教师获山东省信息化教学大赛二等奖2项,三等奖1项,有25名教师指导的技能大赛训练队获省市级以上奖励。教师素质显著提升。
目前,在岗专任教师比2013年增加了12人,专任教师队伍拓展到了60人,兼职教师资源库教师队伍增至61人,专兼教师比例近1:1,兼职教师承担的专业课时比例达到50%以上,生师比达到17.8:1;双师型教师比例由63%提升至97%;高级职称教师比2013年增加了4人,达到22人;教师队伍得到充实优化,结构更加合理,“专兼结合”、“双师型”教师队伍形成。
5.实习实训条件明显改善,实践教学水平大幅提升
三年间,学院累计投入会计电算化专业硬件建设资金570余万元,用于新建改建扩建校内实训基地。其中包括改建4个实训室,新建3个实训室,新建1个校内综合实训基地,新建1个信息化智慧实训室;购买网中网会计综合实训平台1套,购买用友T3、U8软件各1套。新增泰安市永信会计教育培训学校、山东泰盈科技有限公司等8家紧密型校外实习基地,新建“校中厂”2家,“厂中校”3家。建成了“四位一体”实践教学平台,实践教学水平大幅提升。
6.社会服务能力显著增强,区域辐射带动效果凸显
与泰安市永信会计教育培训学校合作成立了五岳志诚会计咨询服务中心,为企业提供技术服务28项;教师完成横向课题4项;有7名教师被泰安市政府聘为科技特派员,社会服务能力显著增强。利用优势师资资源,开展了会计从业资格证、营销、就业、高中升学等培训工作,累计实现社会服务培训收入360余万元。对汶上职业中专、泰山文化中专等中职学校开展对口帮扶,进行课程建设、技能大赛指导培训,区域辐射带动中专学校效果凸显。 
7.共享核心专业建设成果,专业群建设成效突出
通过共享会计电算化专业的建设经验与成果,专业群建设成效突出。国际经济与贸易、电子商务、物流管理三个专业都构建了“平台+模块”课程体系,创新了人才培养模式,修订人才培养方案,开发18门课程标准,立项1门市级精品课,8门院级精品课,建成3个专业的专业教学资源库,有8门课程的整体教学设计和单元设计通过课改达标测试,培训专业群专业带头人3人,培训专业群教师35名,建设12个共享型校内实训基地和8个共享型校外实习基地。
8.开展多种形式的专业教育,学生职业素质明显提高
依托校园文化建设项目,以营造专业文化氛围,举行学生主题活动,推行“7S”管理等为主要方式,开展“诚信、严谨、担当、创新”的专业教育。累计制作宣传板110块,开展专业文化类竞赛12次,主题活动38次,班会133场。学生的职业道德、工作态度、敬业精神、创新意识等职业素养都有明显提高。
(三)建设质量与效益
1.创新人才培养模式,强化职业能力培养
为实现会计电算化专业高素质技术技能型人才的培养目标,创新实施了“虚实结合,四段递进”的人才培养模式,强化职业能力培养,有效解决了学生实习实训中存在的问题。
(1)解决了学生职业能力培养缺乏系统性的问题
该模式对学生职业能力培养全过程进行了系统设计,从专业基本能力培养到创新与就业创业能力培养,逐层递进,环环相扣,更加符合人才技能培养规律,真正做到学习内容与工作过程的对接。该人才培养模式使毕业生的职业岗位适应性和竞争力都有较大提升。
(2)解决了会计类学生到企业顶岗实习的难题
因会计岗位的特殊性,会计类专业学生无法通过企业顶岗形式接触到核心业务,而虚拟实训环境的构建圆满解决了这一问题。既能使学生接受岗位技能训练,又能体验职场氛围,增强了学习的积极性。
通过实施“虚实结合,四段递进”的人才培养模式,近三年毕业生职业能力不断增强。根据招生就业处提供数据,学生就业率由92%提高至99%,其中对口就业率较2013年提高了近十个百分点,达到了80.6%。对学生就业企业的调研结果显示,企业对学生职业能力的满意度达到了94%,较2013年提高了9%。
2.“双线”同步推动教学改革,课堂面貌焕然一新
2014年初,为落实学院教师职教能力培训与测评工作,会计电算化专业开始课程改革和课堂教学改革。课程改革以项目化课程改革为支撑,课堂教学改革以“项目导向,任务驱动”的学院有效课堂活动为平台,同步推进,及时将课程改革的成果应用到课堂教学上,通过课堂教学实施,将问题反馈到课程改革中,相互作用,相互促进,使课程改革和课堂改革都落地生根,有效解决了“改而不用”的问题。
在课程改革的推动下,有55名教师通过了学院职教能力达标测试,完成了37门课程的整体教学设计和单元设计,其中有4名教师被学院聘为测评专家,指导全院测评工作;在课改过程中,按照项目化、模块化思路对课程内容进行了重构,对教学过程进行了重新设计,其中有12位教师的课改案例被学院收录;出版项目化教材10部。
在“项目导向,任务驱动”的有效课堂活动的推动下,“学生主体,能力本位”的课堂教学理念被广泛应用,课堂教学实现“翻转”,教师变身导演,学生成为课堂主角,教师将工作项目、任务布置给学生,学生通过小组讨论、自主探究等学习方法自主学习,教师在这一过程中任务是指导、考核、总结,这种教学模式有效激发了学生的学习积极性,课堂气氛变得活跃,学生的到课率、抬头率、达标率明显改善,课堂状况焕然一新。
3.“四位一体”实践教学平台构建完成,实践教学条件省内领先
首先,以会计基本技能实训室和手工记账实训室为主,构建了职业基本技能训练平台。实训室都配备了多媒体教学设备、性能先进的一体机、翻打传票机等设备,一体机中安装了用友U8、T3财务软件等软件,可满足专业基本技能实训需要。
其次,以4个会计电算化实训室为主,构建了职业专项技能训练平台。每个实训室都配备了性能先进的计算机设备、多媒体设备,同时安装了用友U8、T3财务软件,安装了网中网综合实训平台,可以满足出纳业务处理、成本核算、纳税申报、单证处理等专项技能的实训需要。
第三,以会计综合实训基地为主,构建了职业综合技能训练平台。按照用友VBSE理念,设置仿真工作岗位,可以满足会计业务综合实训和跨专业虚拟商业社会环境实训。
第四,以校内ERP沙盘实训室、会计咨询服务中心、校外实训基地,构建了创新和就业创业职业技能训练平台,用于培养学生创新能力,适应岗位需要。
“四位一体”的实践教学平台构建完成,工位数较建设前增加了405个工位,达到了855个;建设面积较建设前增加了1049平米,达到了1769平米;计算机及网络设备性能优异,实训区实现无线网络全覆盖; 采用国内先进的虚拟实训软件;实践教学条件达到省内领先水平。
4.全方位开展职业素养教育,学生职业素养显著提升
围绕学院“德高技高”的人才培养总体目标,贯彻学院“德育为首,能力为本”的教学理念,结合“诚信、严谨、担当、创新”的专业文化特点,对会计电算化专业学生开展了全方位德育教育,学生职业素养显著提升。
首先,加强课堂德育教育。在课程设置上,除保留国家教育部要求开设的德育课程外,新增加了《论语导读》、《国学智慧》等课程,强化德育教育。另外在课程改革中,要求明确每节课的素质目标,将素质培养贯彻教学始终。
其次,营造专业文化氛围。在教学楼、实训楼张贴悬挂专业文化宣传板136块,其中包括职业素养类30块,专业发展类28块,专业群类26块,学生主题活动类19块,系部介绍类6块。同时还依托学院网站、系部网站、黑板报、宣传栏,构建线上线下宣传平台。图文并茂地宣传专业文化,营造专业文化氛围。
第三,开展“早说晚练”、“辩论赛”、“职业生涯规划大赛”等富有专业文化色彩的主题活动,提高学生职业技能、强化学生的职业体验,增强学生的岗位责任感。
第四,推行“7S”管理模式,培养学生职业习惯。在实训区、教学区、办公区,“7S”管理制度上墙,标准明确,师生已养成自觉遵守“7S”管理的习惯,教学区、办公区、实训区卫生整洁、工作流程规范,有效提高了工作效率。








项目四   旅游管理专业建设
(一)建设目标完成情况
旅游管理专业是山东省技能型人才培养特色名校建设项目省财政支持建设重点专业。专业共有建设项目8个,建设内容33项,验收要点336个,预算资金为530万元,实际投入资金664万元。经过三年建设周期努力,共完成验收要点336个,要点完成率100%。
表2-33  旅游管理专业主要建设指标完成情况一览表
建设内容
建设任务
单位
预期目标
实际完成
完成率
体制机制建设
成立旅游管理专业建设委员会
个
1
1
100%

召开旅游管理专业建设研讨会议
次
3
3
100%

成立旅游管理专业校企合作理事会
个
1
1
100%

召开校企理事会会议
次
12
12
100%

旅游管理专业建设委员会工作制度
份
1
1
100%

旅游管理专业校企合作理事会章程
份
1
1
100%

旅游管理专业校企合作工作实施方案
份
3
3
100%

旅游管理专业订单班管理规定
份
1
1
100%

旅游系“7S”管理建设方案
份
3
3
100%

旅游管理专业人才递进培养实施方案
份
3
3
100%

培养名师
名
2
2
100%

培养教坛英才
名
2
2
100%

培养教学新星
名
2
2
100%

旅游系教学例会制度
份
1
1
100%

旅游系主任听课制度
份
1
1
100%

旅游系教学督导办法
份
1
1
100%

旅游系学生顶岗实习制度
项
11
17
154%
人才培养模式与培养方案
旅游管理专业调研
次
3
3
100%

旅游管理专业调研报告
份
3
3
100%

旅游管理人才培养方案论证报告
份
3
3
100%

旅游管理专业人才培养模式实施方案
份
3
3
100%

旅游管理专业人才培养方案
份
3
3
100%

旅游管理专业卓越人才培养方案
份
3
3
100%

技能大赛获奖
项
3
5
166%
课程体系构建与核心课程建设
编写课程标准
门
15
15
100%

建设优质专业核心课程 
门
5
5
100%

院级精品课程 
门
5
7
140%

校企合作开发出版教材 
门
3
4
133%

校本教材 
本
0
6
净增

专业教学资源库 
个
1
1
100%

课程教学资源库 
门
5
5
100%

教学模式改革报告
份
1
1
100%

考试评价方法改革报告 
份
1
1
100%

教学组织形式改革报告 
份
1
1
100%

学生社团建设 
个
3
15
500%
教学团队建设
培养校内专业带头人
名
1
1
100%

培养校外专业带头人 
名
1
1
100%

培养骨干教师 
名
5
5
100%

培养青年教师
名
2
4
200%

双师素质比例
%
67
100
149%

聘请兼职教师 
名
4
4
100%

兼职教师库 
名
20
21
105%

名导工作室
个
1
1
100%
实训条件建设
改造实训室 
个
1
1
100%

新建实训室 
个
4
8
200%

新增校外实训基地 
家
6
8
133%

厂中校(处)
处
2
4
200%
社会服务能力建设
横向课题或服务项目
个
6
11
233%

技术服务收入
万元
15
19.35
129%

社会培训
人时
12000
15750
131%

培训收入 
万元
30
52
173%

开展技能鉴定 
人
60
75
125%

开展中职学校对口支援 
所
1
1
100%

培训泰安市中职教师 
人
3
3
100%

担任行业技能大赛裁判 
人次
3
6
200%

A级景区评审 
个
3
11
366%

星级酒店审核 
个
3
28
933%
专业群建设
酒店管理专业调研 
次
3
3
100%

酒店管理专业人才培养方案 
份
3
3
100%

旅游专业群平台课程标准 
个
5
6
100%

酒店管理专业优质核心课程 
门
2
9
100%

完善实训室 
个
3
3
100%

新建实训室 
个
3
3
100%

培养专业带头人 
名
1
1
100%

培养骨干教师 
名
2
4
200%

新聘用兼职教师 
名
2
2
100%

专业教学资源库 
个
1
1
100%
中高职衔接和文化建设
人才培养方案
份
1
1
100%

企业文化报告 
次
3
3
100%
(二)建设成效
1.“政、行、企、校”融合,形成校企合作长效机制
成立校企合作理事会,推进校企互动。成立由泰安市旅游协会、高校旅游专家、行业企业管理和技术专家、学校专业带头人和骨干教师组成的旅游管理专业建设委员会,全面负责专业建设。制定并逐步完善了《旅游管理专业建设委员会工作制度》、《旅游系校企合作理事会章程》、《旅游管理专业校企合作工作实施方案》、《旅游管理专业订单班管理规定》等制度,形成了“人才共育、过程共管、成果共享、责任共担”的校企合作办学长效机制。
专业建设委员会和校企合作理事会职能明确,运转规范。校企共同对人才培养方案调整、教育教学模式改革、师资队伍建设、课程体系构建、实习实训基地建设、兼职教师聘任、实习管理、学生就业等专业建设关键环节进行深度分析并决策实施。召开校企理事会12次,校企共同论证人才培养方案3次,建立厂中校4处,与天成国际旅行社、花样年华景区等公司合作编写教材5本,正式出版4本,与山东金磐电子科技有限公司合作建设数字化课程资源和专业教学资源库5门,校企合作成效显著。
2.实施全面质量管理,对全过程进行管控
以学院评价、系部评价、第三方评价、学生评价相结合的教学质量评价体系,对教学质量进行全面的研究、比较和评价,切实保障人才培养质量。
成立由系主任任组长的教学工作督导小组,落实监督检查工作。对教学秩序和教学规范执行情况进行检查,依据学院《教学事故认定及处理办法》,及时处理出现的问题,通报处理教师3人,规范教师教学行为。制定并完善《系部教学例会制度》、《系主任听课制度》、《教学督导办法》等制度。通过听评课、教学检查、期末考核等措施提升有效课堂建设。设立教学信息员,反馈信息。每个教学班至少抽1人组成教学信息反馈小组,每周提报反馈意见,并及时进行处理。学生实习期间的质量监控由校企共同负责,教师和企业导师对学生进行实习指导、监督,实习结束后,学生顶岗实习考核由企业管理人员完成。对毕业生进行跟踪调查,接收来自企业、学生和家长的反馈意见,作为人才培养方案修订和完善的重要依据。
3.深化“旺入淡出、产学结合”特色人才培养模式改革
以三方共赢架构人才培养模式。突出“德技并重、理实一体”设计理念,以学校、企业和学生三赢为出发点,根据旅游企业淡、旺季交替生产周期规律调整教学要素布局,在旅游旺季安排学生进企业顶岗实习,淡季回校上课。既满足了企业周期性的用人需求,又使学生在真实的职场氛围中培养了职业技能。这种教学模式既符合“工学结合”的教学要求,培养了学生的职业技能,又符合国家创新人才培养的精神,将学生就业和创业进行了结合。以学生就业岗位综合能力和职业能力培养为目标,充分利用旅游企业实训资源与条件,落实知识、能力、素质等多种形式有机结合的教学内容。
健全制度,抓过程管理,确保实施效果。召集专业建设委员会论证方案,组织建设团队,精心制定活动方案,寻求优质企业,派驻指导教师,精心挑选企业指导教师,完善顶岗实习规定,制定完善了17项顶岗实习管理制度,落实考核方案,确保工学交替效果。加大顶岗实习信息化管理手段,开发顶岗实习信息管理平台,对实习计划、实习评价标准、学生实习日志、四方(企业、学校、教师、学生)联系沟通等进行信息化管理,实现对顶岗实习的全过程、即时性监督与管理。
改革成效显著。学生经过工学交替以后,职业道德、创新意识、合作能力、服务技能、服务心理和对客服务技巧都有显著提升,同时解决了惯常教学安排下的实践教学安排与企业生产淡季用工缩减之间的矛盾。在旅游旺季,学生顶岗实习,不仅解决了企业周期性的临时用工需求,而且使实践教学安排落到实处,学生带薪顶岗,真实体会职场环境,真实履行岗位职责,理论联系实际。在实践中学生发现自身不足,增强了回校上课的学习动力。
4.以职业能力培养为主线,构建“平台+模块”课程体系
实施课程体系优化工程,在专业建设委员会的指导下,按照“调研—工作任务和职业能力分析—确定毕业生面向岗位—整合提炼典型工作任务—行动领域的转换—学习领域”的转换工作流程,以“平台+模块”为建设思路,构建旅游管理专业课程体系。
该课程体系由公共基础平台、专业基础平台即专业群平台、专业核心模块、公共拓展模块、专业拓展模块组成。专业基础平台也就是专业群平台课程,主要培养学生从业基本素养和技能,由旅游概论、服务礼仪、普通话、职业道德、旅游服务心理、旅游英语等课程构成;专业核心模块主要培养学生核心职业能力,由导游实务、旅行社计调、旅行社经营管理、景区经营管理和旅游市场营销课程组成;通用素质拓展平台为学生素质多方面发展提供机会;专业拓展模块(专业方向模块)设置旅行社、景区和酒店三个方向,学生依据自身优势和兴趣,自主选择发展方向,实现个性化发展。
校企共同开发建设核心课程5门,完成了课程标准、项目化设计、教学课件、实训指导书、学习指导书、技能考核与评价标准、教学案例库、课程习题库、考试题库和拓展资源库建设,达到了省级精品课程的建设水平。与企业合作开发校本教材5本,正式出版教材4本。
融入创新创业能力培养。设置专业拓展课程《职业生涯规划》和《创新创业》,厚植创新创业文化,同时,在实践教学环节强化创新创业教育,培养学生创新创业意识和创新创业能力。在专业教师指导下,学生成立志愿者服务、礼仪服务、导游讲解等专业社团,开展创新创意比赛、技能比赛和各类兴趣社团活动等,培养学生的创新创业思维。每年举办一届大学生职业生涯规划大赛和大学生创业比赛。成立创业训练营,搭建创业能力培养平台。
5. 重视教学研究,进行项目化课程与有效课堂改革
进行项目化课程改造。所有教师每人至少承担一门课程的改革任务,参加学院组织的测评达标活动,全部合格通过。教师普遍采用案例分析、任务驱动、课堂讨论等教学方法,提高学生的抬头率、参与率,有效课堂建设取得新突破。
改革教学模式,以真实项目训练学生职业能力。校内利用泰山书院,组织学生为来校嘉宾进行讲解服务,承接学院各种接待工作的礼仪服务工作,平均每年完成11次讲解、13次礼仪服务;校外邀请全国知名导游带领学生深入景区,边学边练,提升职业能力;为全国优秀导游员韩兆君设立工作室,以师父带徒弟的形式培养优秀学生20人。
改革评价方法,突出过程评价。改变过去只有期末理论考试的评价方法,将学生出勤、课堂表现、课程项目完成情况、团队合作等考核评价要素纳入评价体系,加大过程性评价。同时,引入企业评价主体,学生的顶岗实习全部由企业专家进行考核评价,形成了比较科学的评价考核体系。
6.专兼教学团队形成
采取“走出去、请进来、进驻企业”的方式,培养“有高尚师德,能为人师表;有精湛教学能力,能教书育人;有娴熟动手能力,能指导实训;有较强技术研发能力,能服务社会”的四有型教学团队。增加教师派出学习机会,三年内共安排23人次国内培训,6人次境外培训,参加了国培、省培、专项技能、专业负责人、高级研修等培训;学校邀请专家学者进行全院性培训十余次;所有专业教师每年至少进行一个月的企业顶岗实践。这些培训活动,更新了教师理念,开拓了教育教学思路,使双师素质得到了提升。寻找优秀行业、企业专家加入外聘教师库,形成了21人在内的兼职教师资源库。
三年内教师申报省级课题1项,市级重点课题2项,与企业开展横向研究和服务项目11项,发表论文14篇,获得省微课比赛一等奖、信息化教学设计比赛三等奖、讲课比赛三等奖各1项,获得市级行业技能比赛一等奖、二等奖、三等奖各1项,市职业学校教师基本功比赛二等奖2项。
7.建设产学研结合的实训基地
建设生产性实训基地—育才旅行社,师生参与经营进行实践操作能力锻炼。建成集教学、培训和社会服务功能于一体的校内实训基地。改扩建形体礼仪实训室、3D导游模拟实训室,新建旅游电子商务、化妆、酒水、茶艺、中餐、西餐和客房等实训室7处,满足学生校内实训和教师对外社会服务需求。
新增了泰安市诚之旅国际旅行社、方特欢乐世界等校外实训基地8处,在泰安天成国际旅行社、泰山瀛泰国际旅行社等优质企业设立了厂中校4处。充分利用校外实训基地的资源,开展工学交替顶岗实训、实习和就业活动。
引进先进企业文化和管理理念,校企共同制定完善实训基地管理运行制度、学生顶岗实习管理制度,保障学生实训和实习效果。
8.拓宽社会服务领域,增强社会服务能力
鼓励教师积极开展课题研究,激发教师服务企业的主动性,提升教师科研能力。
开展技术服务,校企互赢。主动服务,承担多种技术服务任务。承担4次全国导游资格考试,担任全市行业协会技能大赛裁判,承担泰安市A级景区评审、星级酒店审核工作,承担泰安市旅游咨询系统、泰安市旅游集散中心建设方案制定工作。参与泰安市、泰山区、岱岳区、汶河新区、东平县的旅游总体规划的编制、评审、论证工作。承接泰安市旅游企业信息调研任务,调研经费每年2万元。承办泰安市职业学校技能比赛2次。
开展社会培训,服务区域产业发展。广泛联系,寻求社会服务项目,开展面向企业员工、乡村旅游员、酒店和景区管理人员的培训活动,共培训963人,服务区域产业发展。
辐射带动兄弟院校。对接泰安市岱岳区职教中心,为该校旅游专业培训专业教师3人,共同制定人才培养方案、共建旅游专业实训中心、指导学生技能大赛,起到了很好的辐射带动作用。
9.依靠重点建设专业,带动专业群发展
引领专业群课程建设。依据旅游管理专业调研的流程进行市场需求和人才培养规格调研,组织教研室教师对人才培养方案进行讨论,邀请校外专家进行论证,形成了2013级、2014级和2015级酒店管理专业人才培养方案。根据“平台+模块”设计思路,重构酒店管理专业课程体系。按照专业群平台建设要求,完成旅游概论、服务礼仪、普通话、职业道德、服务心理等专业群平台课程的课程建设。
辐射带动师资队伍建设。建设期内,酒店管理专业教学团队共完成2次酒店集中跟岗、顶岗实践,6人进行茶艺、调酒、插花专项技能培训,4人进行礼仪、播音、形体培训,大大提升了教师专业技能和职业素养。专业群教师积极参加各级各类比赛,共获得山东省微课比赛省级一等奖1项,省信息化教学比赛三等奖1项,市级行业技能比赛一等奖、二等奖、三等奖各1项,泰安市职业学校教学基本功比赛二等奖1项,3人次获“山东省职业院校技能大赛优秀指导教师”称号。
增设专业,资源共享。新增了空中乘务专业,共享专业群内的师资、实训等资源。结合旅游管理专业实训条件,完善了中餐、西餐、客房实训室建设,新上了化妆、酒水和航空模拟实训室,更好的满足学生实训需求。酒店管理专业参照旅游管理专业教学资源库建设标准,收集整理了包括课程标准、职业资格证书、网站、电子资源在内的教学资源库。
10.打造“诚信服务”专业文化
在学院“四位一体”校园文化大环境下,建设“诚信 服务”为主题的专业文化。立足专业,开展诚信、服务的专业文化建设工作。强化日常管理,将学生的日常表现量化。对学生的生活、学习、活动等各方面纳入量化管理,将诚信良品和服务意识自觉落实在日常行为和学习生活中,为将来成为旅游管理专业优秀人才奠定坚实的基础。
制作宣传企业文化展板。图文并茂地体现诚信、服务等专业文化精神,营造良好的专业化育人环境。邀请行业企业专家进校作报告,融入企业文化。
(三)特色与创新
1.形成“厂校一体、资源共享”的长效机制,校企合作不断深化
建设校中厂和场中校,共享资源,搭建校企合作育人平台。校内设立生产性实训基地—泰安市育才旅行社,教师和学生以双重身份参与企业经营,锻炼提升职业能力。校外在天成国际旅行社、泰山瀛泰国际旅行社等优质企业内部设立4处厂中校,由企业提供场所,专业教师和企业技术专家共同为企业员工和实习实训学生提供教学指导,双方资源共享,搭建合作育人平台,形成了长效机制。
校企合作从最初单一的顶岗实习向订单培养、人员互聘等深度融合方向发展。与山东省舜和酒店集团订单培养学生20名,并提供了28.8万元的学费。空中乘务专业与东方领航集团合作办学,集团投入51万元建成航空模拟仓,为专业培训教师5名。酒店管理专业与山东港航局合作开展酒店管理专业国际游轮服务人才订单培养。
2. 建成“旺入淡出、产学结合”分阶段人才培养模式
突出“德技并重、理实一体”设计理念,以学校、企业和学生三赢为出发点,根据旅游企业淡、旺季交替生产周期规律,调整教学要素布局,建成“旺入淡出、产学结合”分阶段人才培养模式。建设期内,安排3次工学交替实习,总结经验,完善17项管理规定,形成“校企合作办学、适合高技能人才培养”的方案。新的专业人才培养模式,更贴近企业和社会的需求,拓展了学生就业空间,提升了学生综合职业能力,提高了人才培养的质量。
3. 名师引领,探索现代学徒制模式
为全国优秀导游员、全国行业导游技能大赛三等奖获得者韩兆君设立名导工作室,探索现代学徒制。选拔20名优秀学生加入工作室,从基本素养、专业技能、职业综合能力等几个方面进行强化培养。充分利用泰山书院提供的机会,指导学生为来校嘉宾进行讲解服务,以真实项目训练学生职业能力。校外,带领学生踩线,深入景区,边学边练,提升职业能力。三年内,学生获得全国旅游院校职业技能大赛二等奖两项,培养泰安市精品导游员1人,优秀导游员4人。从工作室走出来的学生职业能力突出,很快成为了企业骨干。

项目五   计算机应用技术专业建设
(一)建设项目完成情况
项目立项建设以来,专业团队按照建设方案和任务书完成8个二级项目,25个子项目和327个验收要点,实现了预期建设目标。计划投入资金530.2万元,实际投入635.93万元。主要建设项目指标完成情况见表2-34。
表2-34  计算机应用技术专业建设项目指标完成情况一览表
序号
建设内容
关键指标
单位
预期目标
实际完成
完成度
1
体制机制建设
校企合作办公室
个
1
1
100%


校企合作制度
个
8
8
100%


内部管理制度
个
3
3
100%


合作行业数
个
1
2
200%


有合作协议的合作企业数
个
25
25
100%


年度质量报告
份
3
3
100%
2
人才培养方案优化与人才培养模式改革
专业建设委员会
个
1
1
100%


专业调研
次
3
3
100%


专业调研报告
份
3
3
100%


2015级人才培养方案
套
1
1
100%


人才培养模式改革方案
份
3
3
100%


人才培养模式改革总结报告
份
3
3
100%
3
课程体系构建与核心课程建设
人才需求分析报告
个
3
3
100%


构建课程体系
个
1
1
100%


专业优质课程
门
6
6
100%


院级精品课
门
2
9
450%


编写课程标准
门
15
28
187%


编写校本教材
部
4
4
100%


专业教学资源库
个
1
3
300%


数字教学资源库
门
6
7
117%


课改测评达标人数
人
0
22
超额
4
教学团队建设
聘任专业带头
人
1
1
100%


培养专业带头人
人
1
1
100%


培养骨干教师
人
5
5
100%


副教授以上职称人数
人
4
8
200%


培养青年教师
人
4
4
100%


兼职教师
人
22
22
100%
5
实训条件建设
新建校内实训室数量
个
6
6
100%


新增校外实习基地数量
个
10
10
100%
6
社会服务能力建设
为企业技术培训
人次
0
360
超额


职业技能鉴定
人次
200
409
205%


辐射带动中职学校
所
0
1
100%
7
专业群
人才培养方案
套
3
3
100%


核心课程
门
0
6
超额


骨干教师
人
6
12
200%


新建共享实训基地
个
2
2
100%
8
其他
信息技术工程系综合管理平台
个
0
1
超额


云中心实训室建设
个
0
1
超额
(二)项目建设成效
1.体制机制建设有新突破
(1)成立了计算机应用技术专业建设指导小组,制定了相关工作章程
在学院专业建设指导委员会的统一指导下,由系主任任组长,由企业技术专家、技术能手和骨干教师组成了计算机应用技术专业建设小组,制定了各项目标任务的落实方案,重新修订了多项系部管理规章制度,制定了岗位目标责任制,对规划目标和任务完成情况进行严格评价和考核,为名校建设项目组顺利完成各项任务进行全方位指导。
(2)完善“校企合作、人才共育”的校企合作制度建设,保障校企合作深度融合
在学院校企合作办公室的指导下,依托“专业建设委员会”,校企共同修订人才培养方案,共同进行课程体系开发、教师队伍组建、教育教学实施、管理队伍搭建和考核评价。在企业调研基础上,完成了与中视完美动力集团“动漫环游记——完美动力全国高校大型巡讲”活动,并与青岛完美动力集团完成了“宏志班”校企合作框架协议;邀请北京百科荣创、山东微分电子有限公司、亚伟速录培训的工程师到校内进行集中实训;建立了山东中动文化传媒有限公司校园人才培养基地,并安排103名学生前往校企合作单位进行顶岗实习。
(3)加强系部教学质量监控
在学院“432”教学质量监控与保障体系的统一框架下,制定《信息技术工程系教学质量监控及保障体系实施办法》,加强质量监控,确保教学秩序规范运行。探索建立第三方参与的教学质量评价与保障制度,构建多方参与的教学质量保障体系的处理机制。建立由用人单位、行业协会、学生及其家长、社会等多方参与的第三方人才培养质量评价机制。综合、全面地对毕业生就业率、就业质量、企业满意度、社会认可度及创业成效进行考核与评价,形成与产业对接的人才评价优化机制。
(4)全面推行“7S”管理
在学院办公室的统一领导下,全面推行“7S”管理。引入企业精细化管理理念,在实训室、教室、宿舍区推行“7S”管理,制定“7S”管理实施办法。
(5)加强毕业生顶岗实习管理
根据学院《学生顶岗实习管理办法》,完善计算机应用技术专业顶岗实习方案,制定专业教师巡回指导、企业兼职教师跟踪指导管理办法。依托学院“顶岗实习信息管理平台”,对计算机应用技术专业的学生进行实习计划、实习评价标准、学生实习日志等信息化管理,实现对顶岗实习的全过程、即时性监督与管理。
2.优化人才培养方案,深化人才培养模式改革
(1)召开计算机应用技术专业人才培养方案专家论证会,修订计算机应用技术专业人才培养方案
围绕省会城市群经济圈发展和泰安市电子信息产业的发展,根据专业人才培养方案修订意见和原则要求,邀请山东大学、泰山医学院、泰安市海华科技有限公司、山东中动传媒有限公司等院校专家到校进行计算机应用技术专业人才培养方案论证。结合专家建议,制定切实可行的计算机应用技术专业人才培养方案。
(2)进行专业调研,完成专业人才需求调研报告
在计算机应用技术专业建设委员会的指导下,定期安排人员深入企业开展专业人才需求调研。三年来走访调研泰安海华科技有限公司、青岛完美动力集团、北京百科荣创科技有限公司、山东微分电子有限公司、甲骨文(山东)OAEC人才产业基地、惠普软件(济宁)人才产业基地管理有限公司、北京蓝鸥科技有限公司、亚伟速录培训有限公司、山东中动文化传媒有限公司等30多家企业,赴台湾昆山科技大学、青岛职业技术学院软件与服务外包学院、浙江机电职业学院、山东科技大学等10余所职业院校进行调研。
(3)“阶段培养、能力递进”的人才培养模式初见成效
计算机应用技术专业依托“软件开发中心”及泰安、济南高新技术开发区,以服务为宗旨,以能力为导向,探索专业与产业联动,创新“阶段培养、能力递进”的人才培养模式。以培养学生的职业能力为主线,把学生的职业能力的培养分为基本素质与能力、岗位综合能力、职业能力三个阶段,按照从简单到复杂、从单一到综合、从低级到高级的知识进阶规律,强化学生职业能力的培养。
3.课程体系全面优化,教学效果显著
(1)以岗位技能培养为核心,构建基于工作过程的“平台+模块”的课程体系
计算机应用技术专业按照“阶段培养、能力递进”的“三段式”的人才培养模式,通过调查研究,收集分析社会实际需求、行业实际需求及学生实际需求信息,完善“平台+模块”的课程体系。行业、企业专家技术人员共同参与课程建设的全过程,对计算机应用技术专业课程体系的重构提供了重要帮助,使得课程体系更加合理,从而满足专业的实际需求。
(2)开发数字化教学资源,建立优质专业核心课程
计算机应用技术专业建设委员会按照基于工作过程的“平台+模块”的课程体系开发思想,根据人才培养目标,为满足社会岗位对知识能力的需求,进行科学的课程设置,动态管理。紧跟企业岗位能力的发展需要,以职业标准引导专业课程体系,企业岗位需求引导课程教学内容,把企业文化理念和职业情景直接融入教材,实现教学内容和企业生产实际的无缝对接。加强信息化建设,大力开发数字化教学资源,建成6门体现岗位技能要求、促进学生实践操作能力培养的优质核心课程。
(3)全面推进数字教学资源库建设和校本教材编写工作
根据专业建设方案要求,整合省内外校、企优质资源,通过系统化设计,建成计算机应用技术专业教学资源库。结合课程改革和建设的目标和规划,制定符合专业特色的教材建设规划,全面推进校本教材编写工作。现已经完成《C语言程序设计》《计算机文化基础》《汉字录入》《计算机专业英语》4门校本教材的编写。
(4)实施“有效课堂”建设,加强教学模式改革
根据学院教师职教能力测评活动要求,制定系部教师职教能力培训与测评活动实施方案,组织全系教师进行职教能力测评活动,专业教师充分发挥个人能力,采取多种方法进行职教能力展示,整体推进专业核心课项目式教学改革。全部专业教师通过学院整体教学改革,有一位教师被评为学院职教能力测评专家。围绕人才培养模式改革进行课题项目研究,李霞老师主持了1项山东省教学改革项目,宋晓玲老师主持了1项山东省职教所计算机应用技术专业项目,李倩老师主持了1项2015年度山东省教育科学“十二五”规划课题,院级立项教学改革项目8项。
(5)强化实践教学环节,构建“四层次三结合”的实践教学体系
计算机应用技术专业人才培养模式以岗位能力培养为中心,以强化实践教学环节为重点,突出学生职业能力、就业和创业能力培养,形成可持续发展的实践教学特色。构建了基础实践教学、专业实践教学、综合实践教学、社会实践四个环节循环递进,素质教育贯穿始终的“四层次三结合”的实践教学体系。
4.建立了“理实兼备、优势互补”的优秀双师教学团队
依据学院“名师递进培养工程”,通过聘请、引进、培养等多种途径,创建了一支由特聘专家、专业带头人、行业企业技术骨干、骨干教师为成员的“理实兼备、优势互补”的“双师结构”优秀教学团队。三年内培养专业带头人1名,培养骨干教师5名,青年教师4名;聘任(用)校外专业带头人1名,兼职教师7名,兼职教师承担专业课学时比例达到55%以上;高级职称教师比例明显提升,新增4名高级职称教师,现有高级职称教师8名;校内专业带头人李倩被评为“院级教学名师”,5名骨干教师黄志艳、宋晓玲、李长英、陈亮、张青被评为“院级教坛英才”;青年教师张小童老师被评为“院级教学新星”。
5.理实一体的实训条件不断完善
以“校企融合,实境育人”为切入点,根据专业建设的需要,加大校内外实验实训基地的投入力度,三年建设期内,专业完成软件开发技术研发与服务中心——“软件开发中心”、“数字媒体中心”两大中心的扩建工作,完成云计算中心的建设工作,完成“软件提升实训室”、“网站开发实训室”、“移动开发实训室”、“办公自动化综合实训室”、“网络管理实训室”、“软件研发实训室”6个实训室的新建工作,并与浪潮集团合作研发大数据综合实训教学管理平台。在原有15家校外实训基地的基础上新增10家校外实训基地,使校外实训基地总数达到25家,其中紧密结合性中等规模以上10家。与甲骨文(山东)OAEC人才产业基地、惠普软件(济宁)人才产业基地管理有限公司、北京蓝欧科技有限公司、北京智慧科技有限公司建立紧密型校企合作关系,探索与甲骨文(山东)OAEC人才产业基地进行企业订单培养。
6.专业服务区域经济能力不断加强
依托“软件开发中心”,与电子信息行业企业合作,整合现有的信息化资源和智力资源,通过以“自主研发为主,校企合作为辅”的方式,加强以信息技术研发创新服务为重点的社会服务能力建设,拓宽技术服务项目,加大技术研发力度,促进成果转化,增强专业服务区域经济能力。
三年来,本专业承担了泰山区法院信息技术培训,承担了对泰山中学参加计算机专业春季高考学生共计76人的培训,培训收入4万元;先后对3批次71名泰安市退役士兵进行了为期60天的职业教育和技能培训项目,培训人次700人次,收入34万元;利用人才资源优势,为全市的计算机专业技术人员从业资格认证考试提供服务和技术支持,2014年12月安排2014级学生203人前往山东泰盈科技有限公司进行技术服务,2015年11月安排234名学生前往创业大学进行技术服务,年均达200人次,更好的服务于泰安市信息技术产业;张青老师作为首批“企业访问工程师”到企业顶岗实习期间,参与开发了《泰安市工商行政管理办案通系统》《泰山东岳重工有限公司全国技术服务系统》等10个软件,实现技术研发和成果转让经费年均达20万元;圆满完成2011级、2012级、2013级学生的职业技能鉴定工作,职业技能鉴定人次达到409人,学生顺利通过了工业和信息化部电子行业职业技能鉴定指导中心的计算机装调员高级证书,证书获取率达到100%。
7.带动专业群全面发展
信息技术工程系专业群以“计算机应用技术”专业作为群主,引领“计算机多媒体技术”、“物联网应用技术”两大专业发展,重构课程体系,整合课程内容;校企合作,共建师资队伍;建设开放、共享的实训基地,实现资源共享。
通过联合培养学生、教师培训与交流等,辐射和带动泰安及周边职业院校教育发展,与宁阳职教中心、新泰职业中专和岱岳区职业中心建立长期的校校合作关系,与汶上县职教中心实施跨区域专业对口帮扶。无偿开放实验室并提供优质教学资源;对泰安市县市职教中心的师资进行培训;为受援学校提供专业建设、课程改革、教材开发等方面帮助;选派3名专业教师到县、市职教中心对口支援,不仅提升该地区的专业教育水平,而且促进当地经济的发展。
8.形成“技术改变世界,信息沟通心灵”的专业文化氛围
在专业建设中,重视专业文化建设,通过专业文化墙建设、学生作品展示、优秀毕业生事迹展示、职业规范与标准展示,将“进取、担当、包容、和谐”的泰山精神融合到专业文化或人才培养目标中,展现系部文化与企业文化,营造真实职场氛围,为学生提供学习标准和榜样,潜移默化提升职业素养,形成“技术改变世界,信息沟通心灵”的专业文化。
(三)特色与创新
1.云中心支持的实训室建设成效显著
秉承“精细培养、个性发展”的培养理念,拓宽视野,借鉴国内外科学的计算机教育教学理念,搭建及完善了信息技术工程系综合管理平台,全面落实现代精细化管理理念,实施实践精细化教学、管理、培养,记录师生教学和学生管理全过程,及时评价及改进,规范管理,落实“愉快高效协同”管理理念。
对实训室服务器进行整合,并增添存储阵列等存储设备,建立起了私有虚拟云服务器,组建起中心机房。对系部现有应用系统运行的硬件基础平台进行整合,将原来分别运行在单台物理服务器上的应用系统迁移到虚拟服务器上,形成了虚拟化的硬件共享资源池。通过构建开放、共享的信息化云服务平台,用信息技术为教师、学生等提供资源存储、消息通讯、内容聚合、精细化管理、开放式智能实验室管理等基础IT服务,提高教育教学管理水平。
2.发挥技能大赛优势,提高人才培养质量
本专业高度重视学生技能水平的提高,建立“四层次三结合”的实践教学体系,以强化实践教学环节为重点,突出学生职业能力、就业和创业能力培养。积极鼓励师生参加各级各类技能大赛,成绩有显著提高。先后获得全国第八届信息技术应用水平大赛国家级一等奖1项、国家级二等奖1项,第九届全国信息技术应用水平大赛省级奖项6项,专业教师也先后获得国家级最佳指导教师和优秀指导教师称号,教师参加山东省职业院校信息化教学设计大赛以第一名的成绩获得一等奖。
3.以“自主研发为主、校企合作为辅”,稳步提升社会服务能力
本专业合理整合现有的信息化资源和智力资源,通过以“自主研发为主,校企合作为辅”的方式,加强实践教学的内涵建设,强化项目育人主线;提升专业的社会服务能力。2014年2月完成泰山区法院信息技术培训;2014年、2015年连续两年完成泰山中学春季高考学生专业技能考试科目培训;2014年、2015年承接三次退役士兵职业技能培训培训。专业教师以“企业访问工程师”的形式到企业顶岗实习期间,先后开发布署了《泰山区社区居委会人口和计划生育管理服务系统》,升级《泰山东岳重工有限公司全国技术服务系统》,开发了《泰山职业技术学院公推竞岗选拔副县级干部报名系统》等10个应用软件,超额完成了社会服务培训任务。

项目六  园艺技术专业建设
	•	建设任务完成情况
根据《园艺技术专业建设方案》和《园艺技术专业建设任务书》,项目建设共分7个一级验收项目、30个二级验收项目、340个验收点,经过三年的不懈努力,各项建设预期目标全部实现,部分指标超额完成。计划投入资金517万元,实际投入616.56万元。主要建设任务指标完成情况见下表。
表2-35   园艺技术专业建设项目目标完成情况一览表
序号
建设内容
关键指标
单位
预期目标
实际完成
完成度
1
体制机制建设
成立专业建设委员会
个
1
1
100%


制定校企合作制度
项
5
5
100%


制定内部管理制度
项
5
5
100%


签订合作行业
个
1
1
100%


签订合作企业
个
6
8
133%


年度质量报告
份

3
新增项


形成毕业生跟踪调查报告
个
3
3
100%


订单班培养
个

2
新增项
2
人才培养模式与培养方案
专业调研报告
个
3
3
100%


专业人才培养方案
套
3
3
100%


人才培养模式实施方案
份
3
3
100%


人才培养模式改革总结报告
份
3
3
100%


职业资格证书获取比例
%
100%
100%
100%
3
课程体系构建与核心课程建设
人才需求分析报告
个
3
3
100%


构建课程体系
个
1
1
100%


优质课
门
7
9
128.5%


院级精品课
门
2
5
250%


制订课程标准
门
8
15
187%


编写校本教材
部
5
5
100%


公开出版教材
部
0
1
超额


专业教学资源库
个
5
5
100%


构建实践教学体系
个
1
1
100%


市级大学生科技创新项目
个

1
新增


院级大学生科技创新项目
个
3
4
133%


创新社团
个

6
新增


院级教改项目立项
项
3
6
200%
4
教学团队建设
聘任兼职专业带头人
名
1
1
100%


培养专业带头人
名
1
1
100%


培养骨干教师
名
4
4
100%


培养青年教师
名
2
2
100%


培养双师素质教师
名
15
15
100%


双师素质教师比例
%
100
100
100%


聘任兼职教师
名
5
5
108%


专兼职教师比例
%
50
50
100%


培训专业教师
人次
12
21
175%


教师企业顶岗锻炼
人天
1440
1820
126%


国家级横向课题
项

1
新增项


省、市横向课题
项
1
2
新增项


省级教科研课题
项
1
5
500%


市级教科研课题
项
4
10
250%


院级教科研课题
项
6
12
200%


国家级教学成果
项

1
新增项


省级教学成果
项

1
新增项


院级教学成果
项

9
新增项
5
实训条件建设
新建校内实训室
个
2
2
100%


扩建实训室
个
5
5
100%


新增校外实习基地
个
6
8
113%
6
社会服务能力建设
社会培训
人时
每年完成现代农业技术培训8000人时,农民培训16000人时
每年完成现代农业技术培训12000人时,农民培训18130人时
126%


技能鉴定
人次
600
606
101%


技术研发
项
9
28
311%


春季高考
人次

2615
净增项
7
专业群建设
人才培养方案
套
3
3
100%


制订课程标准
套
3
3
100%


院级精品课群
个
1
1
100%


骨干教师
人
10
10
100%


新扩建共享实训基地
个
2
2
100%
(二)项目建设成效
1.创新体制机制,校企合作有了新突破
(1)成立园艺技术专业建设指导小组,专业顶层设计科学化。
在学院专业建设指导委员会的统一指导下,成立有“政、行、校、企、所”多方参与的专业建设指导小组,由系主任、专业带头人和骨干教师参与,邀请泰安市农业局、园艺企业和科研院所专家代表、毕业生代表参加,专业顶层设计更科学。
(2)校企共建“农时互聘”长效机制
根据专业特点,调整教学计划,与企业形成“农时互聘” 长效机制。农忙时节,教师带领学生到企业进行工学交替,学习综合技能。农闲季节,聘请企业专家到校进行讲座和兼课,参与专业人才培养方案、课程开发和教学资源建设。
(3)形成校企“共培共研共享”资源整合机制
专业教师和企业技术人员,实行人才“共培”,技术“共研”,资源“共享”,使企业参与专业建设成为常态。
(4)“订单培养”,招生即招工,人才培养专项化。
与青岛海利尔药业有限公司和旺旺集团达成“海利尔班”和“旺旺班”的订单人才培养协议,双方共同组建校企合作委员会,共同开展招工招生、培养培训、专业建设、课程体系开发、教师队伍组建、教育教学实施、管理队伍搭建和考核评价,学院成为企业储备人才培养基地。
2.创新构建“两线四段三融合”人才培养模式,人才培养质量不断提升
(1)“两线四段三融合”人才培养初见成效
园艺技术专业经过多年的实践,秉承 “德育为先,能力本位”的教育教学理念,进行高职园艺专业人才培养模式的改革和探索,形成“两线四段三融合”新模式,以“岗位技能培养为核心,行业标准为依据”构建“平台+模块”课程体系,以“项目导向、任务驱动”进行教学模式改革,以“校企深度融合”进行校内外实训基地建设,建立与人才培养模式相适应的动态评价机制,全方位提高学生职业素养和职业能力,使毕业生顺利走向工作岗位。
(2)就业质量进一步提高,招生质量明显提升
园艺技术专业根据行业、企业需求,培养学生动手能力强,具有较强的吃苦耐劳精神,很受企业欢迎,就业率高达100%,专业对口率达75%以上,企业对学生满意度在95%以上。
通过重点专业建设,提高了专业的知名度和社会影响力。对专业报考率、录取率及生源质量都起了积极的推动作用,专业招生连续3年增长。
3.“平台+模块”课程体系构建完成,项目化课程改革全面推进
(1)对接行业职业标准,“平台+模块”课程体系构建完成。
根据园艺行业对从业人员岗位职业素养能力、专业基本能力、专业核心技能、专业拓展能力要求,确定课程体系能力培养的四条主线。根据能力培养所需要的知识、技能与职业素质,重构课程,构建平台,实施基于园艺专业岗位能力需求的“平台+模块”课程体系。
(2)项目化教学改革全员推行,课堂有效性显著提升
全面推行项目化教学改革,制定教师执教能力培训与测评实施方案,组织全系教师进行执教能力测评展示。全体专业教师通过了学院项目式整体教学改革测评,课堂有效性显著提升。
(3)课程建设水平明显提高
建设期内,园艺技术专业5门核心课建成院级精品课,《植物病虫害防治技术》建成泰安市市级精品课程,并完成5门校本教材的编写,5门专业核心课教学资源库的建设。
(4)构建完成“四段循环递进”实践教学体系
根据学生的职业成长特点,结合课程教学、配合能力培养主线,设置“单项技能-专项技能-综合技能-就业创业技能”四个层次循环递进、能力提升的实践教学过程,构建完成专业实践教学体系。
(5)大学生创新创业能力逐渐增强,技能大赛成绩逐年提高
三年来,立项泰安市大学生科技创新项目1项,院级大学生科技创新项目4项;成立6个专业学生社团。组织承办了山东省生态小空间比赛,获得一等奖、三等奖两项。连续两年参加山东省教育厅、农业厅举办“植物组织培养”技能大赛,均进入前三名。
(6)“尚德乐农,创新发展”的现代农业专业文化逐步形成
通过聘请企业专家进行讲座、优秀毕业生创业报告会、校企合作单位进行岗位培训等多种方式,进行专业文化渗透,提高学生学农爱农的职业情操,逐步形成“尚德乐农,创新发展”的现代农业专业文化。
4.教学团队建设卓有成效
(1)人才递进培养和对接行业产业方面成效显著
1名教师获得博士学位,1名教师晋升副教授职称。递进培养“教学新星、教坛英才、教学名师”8名。3年内完成教师境外培训4人次,国培2人次,省培8人次。1人被聘为全国农业生物技术专业职业教育教学指导委员会委员,1人被聘为山东省农林教指委专业指导委员会成员、1人被聘为山东省职业技术教育学会农村职业教育工作委员会。1名教师被聘为泰安市农业科技特派员。
(2)教学团队在质量工程项目方面取得丰硕成果
目前,专业教师已经获得国家科技横向研究课题1项、省市横向课题2项;省级科研项目5项,市级课题6项,院级课题12项,参与获得国家级教学成果二等奖1项,省教学成果1项,院级教学成果11项。发表论文23篇。团队建设大大提升了专业教师科研服务教学的能力。
5.实训条件明显改善,实践教学水平迅速提升
完成改扩建5个专业实训室,新建2个实训室。校外实训基地由10家增至18家,毕业生校外顶岗实训比例达到100%。与企业共建“厂中校”2处,校企共建农产品质量安全检测中心,搭建第三方检测平台,提高了专业及学校知名度,扩大了专业影响力。
6.专业服务区域经济能力凸显
(1)社会培训经济效益显著
对泰安市岱岳区200名新型农民进行农业职业技能培训(病虫害防治和乡村旅游员)4天,对500名农民进行为期12天的蔬菜专项技术培训,培训达6800人次。对305名泰安市基层农技人员开展了为期5天的能力提升培训项目,培训达4575人次。
(2)农业行业职业技能培训
对学生和企业员工进行花卉园艺工、农作物植保员、种子繁育员等职业技能鉴定培训,三年共完成鉴定606人次。
(3)教师积极进行科研成果转化,服务行业企业
专业教师参加省、市、院级科研课题研发,与企业合作进行技术服务。名校建设以来,为企业技术项目研发和横向课题研究3项;立项山东省科技厅项目2项、泰安市科技局科研项目5项、院级科研项目8项。
(4)成为山东省春季高考技能考试(农林果蔬类)主考院校
2014年,我院成为山东省春季高考技能考试农林果蔬类专业主考院校。3年来,承担了农林果蔬类专业类目的试题命制和考试工作,考核2615人次,受到山东省招生考试院和广大考生的认可。
7.实现资源共享,带动现代农业专业群发展
园艺技术专业的建设极大带动了现代农业群内其他专业的建设和发展,在实训条件、师资队伍、教学资源等方面专业群实现资源共享,优势互补。其中,食品加工技术和食品营养与检测专业与旺旺集团开展“订单培养”;2016年3月,园林技术专业学生“园林规划设计”项目技能大赛以山东省第二名的成绩进入国赛。食品营养与检测学生获得山东省农产品残留检测技能大赛优秀奖。畜牧兽医专业引进博士1名,孟秀彦老师被聘为全国高职院校技能大赛“鸡新城疫抗体测定”国赛裁判。学院现代农业专业群发展正以崭新的姿态迈入新的台阶。
(三)特色与创新
1.创新“农时互聘、共培共研共享”校企合作新体制
校企深度融合,实施教师与企业员工进行“农闲互聘”支持计划,既提高教师专业执教能力,也培养专业人才操作能力。通过学生“工学交替”、“订单培养”,做到人才“共培”;与企业合作共同研发科研项目,达到技术“共研”;专业与企业共享教育、实训、技术研发等资源,新的校企合作机制有力促进了专业发展。
2.创新“两线四段三融合”人才培养模式
园艺技术专业经过多年的实践,秉承“德育为先,能力本位”的教育教学理念,进行高职园艺专业人才培养模式的改革和探索,形成“两线四段三融合”工学结合人才培养的新模式,以“岗位技能培养为核心,行业标准为依据”构建“平台+模块”课程体系,以“项目导向、任务驱动”进行教学模式改革,以“校企深度融合”进行校内外实训基地建设,建立与人才培养模式相适应的动态评价机制,全方位提高学生职业素养和职业能力,使毕业生顺利走向工作岗位。开展订单培养,校企共同制定人才培养方案,共同参与学生综合能力培养,使人才培养专项化。
3.创新服务模式,专业服务行业企业能力凸显
密切关注“三农”政策,利用区域经济发展规划,通过专业教师主动联系企业,走到田间地头,进行新型农民职业培训、农技人员培训、与企业共同研发项目和为企业进行技术指导等多种方式,提高专业服务行业企业能力。三年期间,专业教师通过以上服务模式,社会服务能力逐步加强,专业在社会的知名度逐渐提升。
项目七 汽车电子技术专业建设
(一)建设目标完成情况
汽车电子技术专业二级项目8项,子项目35项,具体验收要点266项。自2013年项目开始建设以来,通过近三年的努力,圆满完成了项目任务的建设,部分项目超额完成,具体验收要点完成情况见下表:
表2-36  汽车电子技术专业验收要点完成情况统计表
二级指标
验收要点
计划数
实际完成数
完成率
体制机制建设
成立专业建设指导委员会
1
1
100%

召开汽车电子技术专业建设研讨会
3
3
100%

建立校企合作办公室
1
1
100%

校企合作办公室工作章程
1
1
100%

召开校企合作工作会议次数
5
6
120%

制定校企合作教学管理办法
1
1
100%

兼职教师聘任管理办法
1
1
100%

教师顶岗实习管理办法
1
1
100%

汽电系骨干教师和专业带头人培养方案
1
1
100%

汽车系教师绩效考核办法
1
1
100%

制定教学督导工作办法
1
1
100%

汽车电子技术专业质量评估报告
1
1
100%
人才培养模式与培养方案
汽车电子技术专业调研(次)
3
3
100%

汽车电子技术专业调研报告(份)
3
3
100%

优化汽车电子技术专业人才培养方案(次)
3
4
113%

人才培养方案实施情况分析报告
3
3
100%

专业指导委员会研究论证人才培养方案的会议
3
4
113%

汽车行业调研报告
3
3
100%

人才培养模式改革方案
3
3
100%

实践教学信息和反馈记录
6
6
100%

制定顶岗实习方案
2
2
100%

专业教师指导顶岗实习次数
12
14
117%

制定顶岗劳务或技术服务津贴协议制度
1
1
100%

制定教师顶岗实习跟踪指导管理办法
1
1
100%
课程体系构建与核心课程建设
修订核心课程教学标准数
6
10
166%

企业技术标准分析报告 
1
1
100%

课程体系使用效果反馈报告
1
1
100%

课程体系实施评价报告
1
1
100%

建设优质专业核心课程数
3
4
133%

院级精品课程
3
4
133%

开发校本教材数
3
6
200%

出版教材
3
6
200%

教材评价次数
3
6
200%

校企合作开发课程
3
4
133%

企业人才结构与需求状况调研报告 
1
1
100%

专业教学资源库
3
4
133%

优质网络核心课程
3
4
133%

专业教师课程改革数
10
22
220%
教学团队建设
聘请兼职教师人数
9
11
122%

双师型教师人数
15
22
147%

专任教师研究生数
11
15
136%

培养校内专业带头人
1
1
100%

名师递进培养教师人数
9
9
100%

培养骨干教师人数
5
8
160%

高级职称教师人数
8
11
137%

教师培训人次
22
65
295%
实验实训条件建设
卓越技师训练中心
1
1
100%

新建实训室数
1
4
400%

改扩建实训室数
3
4
133%

汽车检测中心
1
1
100%

汽车养护中心
1
1
100%

新增校外实训基地数
5
6
120%
社会服务能力建设
开展汽车保养培训次数
3
3
100%

开展汽车销售培训次数
3
3
100%

开展汽车专业专升本辅导次数
1
1
100%

完成社会培训人数
200
212
105%

开展技能鉴定人次
400
650
163%

社会培训收入(万元)
150
478.5
319%
专业群建设
专业群建设调研
12
14
116%

综合实训中心建设方案
1
1
100%

优化专业群人次培养方案次数
14
16
114%

建设专业群核心课程数
4
4
100%

撰写专业群年度质量报告数
5
7
140%

专业群共享实验室数
22
23
105%

培养专业群内专业带头人数
10
12
120%

培养专业群骨干教师人数
15
15
100%

聘用兼职教师人数
15
17
113%

建设专业群教学资源库数量
5
5
100%

专业群双师型教师人数
26
26
100%
专业文化建设
开展专业文化主题班会次数
30
90
300%

开展专业文化活动次数
6
12
200%

制定“7S”管理制度数量
1
3
300%

开展专业文化讲座次数
3
6
200%

开展学生企业参观次数
3
6
200%

开展学生素质拓展次数
3
3
100%

开展学生企业实践次数
3
3
100%
(二)项目建设成效
截止到2016年4月,汽车电子技术专业名校建设共支出建设经费791.39万元,人才培养效益及社会效益逐步显现,主要集中体现在以下几个方面:
1.建立健全体制机制,保证名校建设项目顺利进行
(1)在体制机制改革方面,成立汽车电子技术专业建设项目领导党小组,建立校企合作办公室,完善组织机构,高效的开展工作。包括完善办公室工作章程,制定各年度工作计划,定期召开专业建设研讨会和校企合作办公室会议,确定专业建设方向。定期企业调研,收集企业需求变化情况,定期进行工作总结。
(2)加入山东交通运输职教集团,山东汽车工程职教集团,发挥职教集团作用,深化校企合作、产教融合。与国内行业知名企业建立深度的校企合作关系,如与长城汽车制造有限公司、新疆广汇汽贸集团、航天特种车辆制造有限公司、吉利汽车制造有限公司、一嗨租车等多家企业建立了深度的校企合作关系,校企共建专业,提高人才培养质量。在校企合作基础上,建立了工学交替制度,我系汽车专业学生,先后到济南吉利汽车制造公司、天津长城汽车制造公司进行的工学交替。
(3)深化内部管理制度改革,完善质量保证体系建设。调整了汽车电子技术专业建设委员会、健全了教学质量督导领导小组体制,完善专业、课程、实践教学及顶岗实习管理等制度,全面实施内部管理考核激励办法,保证了人才培养质量的提高。
(4)加大社会化培训力度。一方面强化企业对专业教师的深造培训,比如在上海举办的萨塔喷涂培训、在无锡举办的汽车技术培训等;一面是强化企业对学生的就业岗前培训,比如上海中锐集团在无锡南洋职业技术学院针对我系12级和13级学生开展的为期三个月岗前技能强化训练,提高学生就业竞争力。
(5)加大教师企业顶岗锻炼,提高教师职教能力。在2015年初,我系18名专业课教师分别到星科集团、广汇汽车、一嗨租车、吉利汽车制造厂等多家企业进行企业顶岗锻炼,极大的提高了教师的实践技能水平和项目化课程改革能力。
2.优化人才培养模式与培养方案,切实提高人才培养质量
建立人才培养方案动态调整机制,优化人才培养方案,紧跟行业发展,重视学生实践操作能力培养。加强院校交流,济南技师学院、山东铝业学院、东营职业技术学院等学校的汽车类专业先后来校交流,共同探讨人才培养模式的改革与创新,完善实施“4+1+1” 校企共建共育的工学结合人才培养模式的办法,提高人才培养质量。
(1)通过行业调研,定期召开专业建设委员会会议完善“4+1+1”校企共建共育的工学结合人才培养模式,形成校企合作共同培养优质人才的办学特色,提高学生综合实践技能水平。在人才培养方案和培养模式的建设中,主要做了深入调研、深化完善、深度执行汽车专业4+1+1的人才培养模式。
(2)制定完善了汽车电子技术专业学生的顶岗实习方案、兼职教师跟踪指导管理办法、顶岗劳务或技术服务津贴协议制度以及学生顶岗实习风险代偿或补偿的相关办法。
(3)与北京中唐方德科技有限公司合作开发了数字化实习实训系统,实现了对学生顶岗实习远程监控的效果。
(4)加强学生专业技能素质培养,学生培养质量提高,就业率、就业对口率大幅度提高,招生规模快速增长,实习就业率达到了100%,对口率达到了96%以上。
3.整合优化“平台+模块”课程体系,促进技能型人才培养
(1)建专业建设指导委员会,进行社会调研,组织行业企业专家进行课程体系论证探讨,优化核心课程教学标准,完善“平台+模块”的专业教学课程体系,提升素质教育能力、实践教学能力,开设创新创业教育课程群,提高学生创新创业能力。
(2)校企共建专业核心课程,出版教材6本,建设网络优质资源课程4门。先后与上海交大、中锐集团联合开发出版了《汽车电器设备》、《汽车底盘构造》、《汽车发动机构造》、《汽车发动机电控》等4本教材。积极进行网络资源库建设,通过学院牵头组织,与中唐方德及金磐科技合作,共同建成优质资源精品课程平台,将《汽车发动机构造与检修》、《汽车发动机电控技术》、《汽车底盘构造与检修》、《汽车电器设备》等4门建设成为网络优质课程,将课程标准、教学课件、电子教案、视频、习题、实训指导书、参考资料、在线单元测试题等纳入资源库中,形成了网上优质教学资源库平台。
(3)实施“任务驱动、项目导向”的教学模式,通过项目教学法、案例教学法等教学方法,采用多媒体教学、开放式实践教学、网络自主学习等手段,改革教学组织形式。积极推进项目式教学,所有任课教师全部通过院职教能力达标测评深化项目化课程改革,在职专兼职教师共27人,全部完成了课程改革项目,通过了学院组织的教学能力水平达标测试。
(4)加强创新创业教育。泰山职业技术学院十分重视大学生的创新创业教育,关心学生就业情景发展,设立泰安市创业大学,在此基础上,汽电系大力发展学生创新创业教育,结合汽车电子技术专业建设,成立汽电系学生汽车协会,开展创新创业实践,结合专业课程,开展汽车保养维护、汽车美容、汽车网络可视设备安装、汽车电器设备检修以及汽车电路检修等创业实践活动,目前正在整合资源,筹备成立汽车养护服务中心,为学生为了创业提供前期实践准备。另外通过开设大学生创业指导、就业指导等课程进行全民创业教育,帮助学生提高创业素质,做好创新创业知识储备工作。
(5)注重学生实践能力提升,通过了一系列的课程建设,我系师生在名校建设期间,荣获全国信息化技术应用大赛国家级二等奖1项,三等奖3项;荣获全国机械行业高职院校汽车技能大赛三等奖两项;荣获山东省大学生机电产品创新设计大赛一等奖5项,二等奖20余项;在泰安市大学生科技创新活动中,立项10余项,占全院立项总数的1/3以上。
4.加强教学团队建设,打造高素质教师队伍
汽电系领导层,深入学习科学发展观,注重教师队伍的人性化发展,重视教师职业成长。以培养中青年骨干教师、“双师” 教师为重点,全面提高专任教师的综合职业素质和实践教学能力,确保所有专业教师参加专业培训1次以上,实现全体教师的整体提高,建设了一支校内外专业带头人引领、专兼结合、双师型高素质的教师队伍
(1)优化生师比,将比例调节在16:1的合理状态;
(2)师资队伍更加合理,现有高级职称教师11人,高级职称教师比例达到50%;研究生以上学位15人,研究生以上学位比例达到 68%,双师教师22人,双师教师比例达到100%,宋丽玲老师,陈元勇老师被聘请为泰安市企业科技特派员;
(3)根据专业带头人和骨干教师评选及管理办法,培养校内外专业带头人2名,培养骨干教师8名;
(4)专业教师参加国培、省培、教指委和企业等培训共65人次;
(5)聘请3名以上的企业技术人员,其中至少1人为技术总监,指导学生进行技术实训。聘任人员制定详细的实训计划并圆满完成;
(6)实施名师递进建设工程,培养市级教坛英才1名,院级教学名师3名,院级教坛英才4名,院级教学新星3名。
5.加大实验实训条件提升力度,促进学生技能成长
作为培养高素质技术技能型人才为己任的高职院校,更加重视学生的实践动手能力的培养,为了更好的锻炼学生的实践能力,提高学生的操作水平,构建基础技能、专业技能、综合技能、创新能力、就业创业能力五位一体的实践教学平台。三年来,汽车电子技术专业在实训室建设中预算投入272万元,实际投入791.39万元,超额完成任务。建设期内新建卓越技师训练中心、汽车虚拟仿真实训室2个,新建汽车检测中心1个,扩建校内实训室4个。与6家汽车行业中有影响力的知名企业建立密切的合作关系,新建5个校外实训基地,以满足学生的顶岗实习要求。
6.社会服务能力显著提高
依托自身优势,多形式、多渠道地开展社会服务,服务周边经济发展,为其他职业院校的建设和发展起到示范引领作用。面向泰安及周边地区汽车制造业和汽车服务业,积极开展社会培训及技能鉴定工作,开展社会培训7671人次,开展职业技能鉴定650人次,实现收入478.5万元。
(1)提高服务周边经济能力,建立了泰安市汽车技术服务平台,泰安市技术服务技术群,目前加入平台的有润之福大众、宝骏、东方经贸天津一汽、北方车辆等多家4S店及多家汽车修理厂。
(2)系部成立汽车协会,面向全院教职工开展汽车义务养护与维修保养。
(3)校企合作成立汽车美容培训班,面向社会开展汽车美容装饰培训。与广汽传祺合作,在校内实习基地开展汽车展销活动等。
(4)派出2名教师到新泰、新疆等地支教,输出课程4门。
(5)响应市政府号召,开展退伍士兵社会培训工作,并通过多元化的培训内容,增加社会培训项目,培训收入达到了400余万元。
7.带动相关专业群发展
通过专业调研、召开专业论证会,对各专业学生就业情况及服务泰安市区域经济发展需求等方面综合分析后,建立以汽车电子技术专业为核心,与汽车技术服务与营销专业、电气自动化专业、汽车检测与维修专业、应用电子技术专业组成的汽车与电气专业群,通过汽车电子技术专业建设,带动专业群的高速发展,其中汽车检测与维修技术专业是建设期内新增专业。共享校内汽车维修实训中心,培养专业群骨干教师15名,在实训基地、课程建设、教学团队等方面为专业群的建设提供更好的条件,实现教学资源共享。更好地服务泰安产业结构调整和转型升级后的经济发展。1名教师被聘为泰安市科技咨询协会理事,2名教师被泰安市科技局聘为企业科技特派员,2名教师参与山东省高职、五年一贯制应用电子技术教学指导方案制定。
8.专业文化建设
发挥课堂教学的主渠道作用, 秉承“修德、笃行、创新、奉献”的校训,开展融“厚德、包容”的泰山文化和“严谨决定成败、服务创造价值”的企业文化为一体的校园文化建设项目,突出“做事先做人”的文化理念,重视学生职业道德和职业操守的塑造与培养;构建“四育人”特色德育模式(环境育人、管理育人、活动育人、文化育人),将泰山文化教育渗透到教学之中,引入“7S”管理模式,开展各类企业文化讲座,丰富实训室内汽车行业相关展板,开展汽车企业一线参观、实践等学习活动,实训过程中引入相关企业管理理念,模拟企业工作环境对学生进行实践锻炼。通过一系列的文化建设,形成了“严谨务实、追求卓越”的汽车电子技术专业特有的专业文化。
(三)特色与创新
1.工学结合促发展——人才培养质量显著提高
2013年以来学生参加山东大学生机电产品创新设计竞赛获一等奖1项、二等奖8项目、三等奖6项,参加全国信息技术应用水平大赛获二等奖1项目、三等奖3项,参加全国汽车职业院校学生技能大赛获三等奖2项,学生科技创新活动在泰安市科技局获得科研课题立项3项。
2.校企共育促成长——校企合作进一步深化
汽车与电气工程系领导班子十分重视校企合作建设。通过校企合作充分利用企业的信息优势、技术优势和设备优势,把企业和学校教育紧密结合起来,让企业在学校的发展规划、专业建设、课程建设、师资建设、实习教学、教学评价、研究开发、招生就业和学生管理等方面发挥积极的作用,为专业课程改革作出应有的贡献。
自2013年以来,我系先后与上海中锐教育投资有限公司、泰安特种车制造有限公司、一嗨租车汽车租赁公司、泰安市北方车辆有限公司、长城汽车制造有限公司、吉利汽车制造有限公司等多家企业建立了合作关系,校企合作的深度与广度正在逐步增加,为系部的发展奠定了坚实的基础。与建立深度合作关系,先后有560人次分别进入天津长城汽车制造有限公司,济南吉利汽车制造厂进行短期工学交替。泰安市天工系统集成有限公司联合中国电信,投入47.5万在我系建立了泰安市公共安全技术实训培训基地,又投入5万元完成了系部无线网络全覆盖。
3.一线实践促技能——教师业务水平得到大幅度提高
通过名校建设,4名教师赴台湾进行培训学习,21名专业骨干教师到浙江机电职业技术学院进行培训,2015年18名专任教师赴济南广汇、吉利、星科集团等企业进行企业调研顶岗锻炼。上海中锐教育集团累计培养6人次。20人次教师参加国培、省陪,通过培训,更新了职业教育理念,提高了专业技能及教育教学水平。
4.提高硬件促发展——实训条件大幅度改善
新建了卓越技师训练中心、汽车仿真实训中心,扩建了汽车底盘实训室。购买设备全部安装完毕并投入正常使用,极大改善了汽车电子技术专业的实训条件。成立了汽车检测中心,以汽车养护中心为主要工作场所,对校内外车辆进行汽车性能检测及维护保养。
5.多措并举促服务——社会服务能力得到提升
承担了泰安市退役士兵汽车专业培训,学生及社会人员汽车电器装调工的技能等级等培训项目,得到参培人员的一致好评,三年来社会培训收入达478.5万元。面向全社会的消防公共安全培训即将开展,实训室建设以基本完成,春季后面向社会招生。
6.辐射带动促成效——专业辐射成效显著
通过对汽车电子技术专业的建设辐射带动了汽车检测与维修技术、汽车技术服务与营销、电气自动化技术、应用电子技术4个专业的发展与建设。
7.改革创新促成才——大力开展非专业素质教育
深入推行7s管理制度,提高宿舍、教室和实验室的管理水平,并与企业共同开发企业文化及专业文化建设,专业文化氛围增强。在课余时间,开展丰富多彩的第二课堂活动,对学生进行爱心教育,增强感恩意识,增强合作意识,并且开展了多元化的创业创新教育,提高学生的创业创新能力。



项目八    服装设计专业建设
(一)项目建设完成情况
服装设计专业项目建设以来,专业团队按照建设方案和任务书要求完成29个二级指标,379个验收要点。项目预算投入303万元,实际投入488万元。实现了预期建设目标。主要建设指标完成情况见表2-37。
表2-37   服装设计专业主要建设指标完成情况一览表
建设内容
建设任务
预期目标
实际完成
完成率
制机制建设
成立服装设计专业建设委员会(个)
1
1
100%

召开服装设计专业建设研讨会议(次)
3
7
234%

服装设计专业建设规划(份)
1
2
200%

服装设计专业建设委员会工作章程(份)
1
1
100%

服装设计专业校企合作工作实施办(项)
1
1
100%

服装设计专业学生顶岗实习管理办(项)
1
1
100%

服装专业“7S”管理建设方案(项)
1
1
100%

服装专业教学质量保障体系实施办(项)
1
1
100%

山东省级特色建设专业
1
1
100%
人才培养模式与培养方案
服装设计专业调研(次)
3
4
134%

服装设计专业调研报告(份)
3
4
134%

服装设计专业人才培养模式改革报(份)
1
1
100%

人才培养方案论证报告(份)
3
3
100%

服装设计专业人才培养方案(份)
3
3
100%

服装设计专业顶岗实习工作规范(项)
1
1
100%
课程体系构建与核心课程建设
建设优质专业核心课程(门)
4
7
175%

精品课程(门)
5
6
120%

开发校本教材(部)
2
5
250%

课程教学资源库(个)
4
4
100%

专业教学资源库(个)
1
1
100%

教学模式改革报告(份)
1
1
100%
教学团队建设
培养校内专业带头人(名)
1
1
100%

培养校外专业带头人(名)
1
1
100%

培养骨干教师(名)
5
6
120%

聘请兼职教师(名)
8
8
100%

兼职教师库(名)
8
12
150%

双师素质教师(名)
8
8
100%

名师递进培养(名)
3
7
267%

省级教学名师(名)
1
1
100%

省级优秀教学团队(个)
1
1
100%
教学实验实训条件建设
改建服装工作室(个)
5
5
100%

新建服装工作室(个)
5
5
100%

新增校外实训基地(家)
6
6
100%
社会服务能力建设
开发特色新产品(项)
2
3
150%

开展技能鉴定(人)
60
90
150%

开展中职学校对口支援(所)
1
2
200%

技术服务收入(万元)
50
107
210%
辐射带动专业群建设
专业调研(次)
3
3
100%

专业人才培养方案(份)
3
3
100%

建设优质核心课程(门)
2
2
100%

新建专业群共享实训室(个)
1
4
400%

培养专业带头人(名)
1
2
200%

培养骨干教师(名)
3
4
100%

新聘用兼职教师(名)
6
6
100%

专业教学资源库(个)
1
1
100%
总完成率
133%
(二)项目建设成效
1.创新体制机制,校企共同开创双向服务新局面
(1)建立健全组织机构,践行专业建设委员会职能
成立专业建设委员会,开展活动7次。1名委员被聘为“省纺织服装专业建设指导委员会委员”。委员会根据服装职业岗位对人才要求,开展活动7次,确立专业培养目标和规格,审定人才培养方案,商讨教学团队建设规划,开展教学质量评价等,践行了专业建设委员会职能。
(2)建立校企合作长效运行机制
在专业建设委员会指导下,修订服装设计专业建设规划、制定了专业建设委员会工作章程等18项制度(方案),构建了有利于技术型和高层次技能型人才培养的制度环境。发挥校企人力、物力和信息资源,在学生实习、就业、人才培养、员工培训、产品开发等方面展开深度合作,努力推进产学研合作对接,逐步完善校企合作长效机制。
(3)全面推行“7S”管理模式
引入“7S”管理模式优化服装教学岗位师生工作职责。制定完善“7S”管理标准及评审细则,各工作室内部管理采用了企业精细化管理理念,工作区进行细致划分,提高了工作效率。改善教学风气,实现了“人造环境,环境育人”。
(4)完善教学质量保障体系
建立专业教学“全过程监控”教学质量监控与保障体系。落实教学质量第一责任人制度、隔周教学例会制度、加强教学常规检查、听课及教学督导力度;实施岗位目标责任制,细化分工,明确责任;实施工作效能督察制,提高工作质量。完善人才递进培养实施方案和教师工作管理办法,实现校企人员互兼互聘。顶岗实习采用了信息化管理模式,毕业生质量跟踪调查方面联合社会、学生实习企业、专业共同展开,实现信息的及时交换、汇总、分析和反馈。
2.创新基于服装工作室的“工学交替、能力递进”的人才培养新模式,提高服装人才培养质量
(1)优化基于服装工作室的“工学交替、能力递进”的人才培养方案
校企共同商讨优化专业人才培养方案。先后赴全国9个省份,7个地区和6个县市区调研,调研企业22家,院校11家,撰写4份调研报告。形成了既能满足服装设计生产目标、岗位知识能力与职业素质需要,又注重学生全面发展的人才培养方案,优化了基于服装工作室的“工学交替、能力递进”的人才培养方案。
(2)确立基于服装工作室的“工学交替、能力递进”的人才培养模式
在专业建设委员会的指导下,按照“能力培养层次化,师生身份职业化,基地建设企业化,实践教学生产化”总体要求设计实施了人才培养方案,实现学生、专兼职教师在校企间的互动。与雷诺、如意等企业合作,确定服装设计生产职业岗位工作任务,按照学生职业生涯发展特点,实行专业认识实习、专业单项训练、专业综合实训、顶岗实训三年不间断的实践教学组织方式,最终构建了基于服装工作室的“工学交替、能力递进”的人才培养模式。
3.加大课程改革力度,构建适应“服装岗位能力需求”的课程体系
(1)贴合岗位实际的职业能力分析为课程改革把准脉向
深入纺织服装企业生产一线调研,最终确定服装设计、生产、管理、营销为四大关键岗位,服装产品设计、制作、品质检验及营销能力作为职业核心能力,并以此作为专业建设和培养技术型、高层次技能型人才的依据。
(2)校企共同制定人才培养目标,为课程改革明确方向
培养能从事设计、产品开发、生产技术管理、营销等职业岗位所必需的综合素质和完成服装业生产、服务、管理工作所必备的专业知识和职业素养,两者构成了服装职业能力。综合素质是基础,职业素养是核心,两方面相辅相成,缺一不可。
(3)构建适应“服装岗位能力需求”的课程体系
通过“确定岗位群—典型任务分析—行动领域分析—学习领域分析”,通过学科体系知识的解构与行动体系知识重构,构建了基于“服装岗位能力需求”的课程体系。
(4)全面推进以服装产品项目为载体、任务驱动、教学做一体化的课程开发
①推行基于工作过程系统化的“服装工作室+项目”教学模式改革。将“服装工作室”引入教学,详细分解工作室工作流程,明确了学习训练任务,在“教、学、做”合一的场景下,使学生独立、循序渐进地掌握工作的各个环节,将学习内容工作化,在工作过程中体验整体服装从设计到实现为产品的整个工作流程。
②积极推行有效课堂改革。有效课堂改革主要针对教师课堂教学开展,评价由教师教的成效转变为学生学的成效。推进了体现“学思结合、知行统一”的启发式教学、案例教学、讨论式教学等多样化的教学方法改革,并将道德修养、职业素养融入了教学过程。教师课程整体设计和单元设计全部达标。
③优质核心课程与精品课程建设。建成体现工学结合的优质核心课程7门,2门课程被评为省级精品课程,2门课程被评为“泰安市级精品课”。编写5本优质校本教材。4门课程建成了网络课程资源库。
(5)教学组织形式改革
①定期组织社会调查与讲座活动。通过专家讲座、企业参观等多种形式积极开展学生职业教育、职业规划活动,最大限度地激发了学生积极性。
②积极开展合作育人活动。建立社会、家、校联系制度,定期邀请校外专家来校作报告,开展共育合作教育活动。邀请优秀毕业生回校做报告,帮助学生规划职业。邀请企业技术能手、著名艺术家开展讲座,通过合作育人提高学生修养。
③探索实施“双讲师”制。推行了兼职教师与专任教师共同授课模式。
④加大顶岗实习改革力度。充分利用校外实习基地,增大生产性顶岗实习力度,顶岗实习不少于6个月。顶岗实习期间,采用现代化信息技术管理学生,由专业教师和企业技术人员全程跟踪指导,实习成绩由校企共同考核管理。
⑤推行双证书制度。学历证书和服装定制工职业资格证书相结合考核度,增强学生就业能力和竞争能力。
⑥开展大学生创新创业教育。以学院创建创业大学为契机,创设以创新创业为导向的创业环境,组建学生搭伙众创空间,创建大学生创新创业孵化基地。通过搭伙众创空间开展创客分享活动,初步建成汇聚创意的“创梦工坊”。
⑦通过大学生科技创新项目将学校课程与实践有机结合。3年申报7项科技创新课题。尝试将“大学生科技创新项目”引入教学。依托“C.U服装设计工作室”等学生社团,结合大学生科技创新课题进行服装综合设计。
⑧建立和完善校内学生技能竞赛制度。将大学生服装设计竞赛项目植入教学。“以赛促学、以赛促练”改变了传统服装专业课教学方式。定期开展技能与创新大赛。3名学生参加“山东省制板师”大赛,获优秀奖。连续3年参加泰安市技能大赛获一等奖6项。其他获奖20项。
⑨积极开展社会实践活动。发挥学生专长服务社会。成立了服务型C.U服装设计学生工作室,为全院学生提供服装制作、织补、清洗等技术服务;成立了10个艺术设计专业学生社团,开展社会服务。
⑩联合社区开展合作育人活动。积极组织学生开展志愿者活动,服务社区、服务社会。与多家幼儿园合作开展学生实践活动。与福利院合作开展“大手牵小手”志愿服务。1个项目被评为泰安市“最佳志愿服务项目”。1个团队被评为山东省大中专学生志愿者“三下乡”优秀团队。
因材施教的具体做法,最终实现“全方位育人和全过程育人”的目的。
(6)教学改革成效显著
依托服装工作室与合作企业,三年立项省级课题5项,市级课题3项,院级课题10项。获国家级奖项2项,省级奖项20项,市级奖21项。申请专利3项。
4.加大师资队伍建设力度,提升教师文化传承与社会服务能力
积极开展卓越技师培养和名师递进培养工程。培养“山东省级教学名师”1名,“山东省十佳制版师”1名,递进培养“教学新星、教坛英才、教学名师、功勋教师”8名。培育专业带头人2名、骨干教师6名;聘请兼职教师8名。8人参加访问工程师培养。教师境外培训6人次,国培、省培8人次。1人被聘为“省专业教学指导委员会委员”。1人被聘为“国家职业资格技能鉴定督导员”,8人考取“国家职业资格考评员”证书。6人考取“心理咨询师”、“就业指导师”证书。14人次获省市荣誉称号。通过项目建设支持教师开展应用性创新研究和实践,全面提升教师执教能力、文化传承能力、实践动手能力和社会服务能力。服装设计专业教学团队被评为“山东省级优秀教学团队”。
5.加大实训投入,切实增强专业服务造血功能
加大了投入,提高了装备水平。新建、改建工作室10个。企业投入合计61万元,建成了“基于开放经营型服装工作室”的教学做一体的实践教学平台。增加6处校外实训基地,深度合作企业3个,学生带薪顶岗实习率达100%。
(1)构建适合岗位能力需求的实践教学运行平台。建设特色型基础性实训工作室、拓展实训工作室、师生设计工作室,校企共建专题工作室等,搭建适应岗位能力需求的实践教学运行平台。依托平台发挥专业教师设计优势,借智兴业,服务泰安区域经济。
(2)与企业深度合作共建专题设计工作室。与企业联合成立“学生装设计研发工作室”、“中华泰山封禅大典演出服装研发工作室”、“圆方商贸服装研发工作室”。工作室的建立旨在开发设计特色服装。
(3)建立创新型学生工作室——“C.U服装设计工作室”。主要为师生提供服装修补服务,为学生创业实践提供平台。
(4)师生共建“C.U”服装设计工作室。发挥泰山名师示范、引领作用,促进青年教师快速成长。开展校内、校间、校企研讨与交流,为教学改革深入与成果推广提供空间。
(5)在特色工作室硬件建设的同时,重视实训教学体系的内涵建设。突出工学结合,利用实训基地的良好条件,开发设计新产品,进行生产性实训课题开发;提高教师实训指导业务水平;加强校外实习基地运行管理,完善顶岗实习管理制度,提高实习质量。
6.发挥实训中心优势,增强专业社会服务能力
依托服装实训中心,开发新产品4项;为泰山封禅大典设计研发表演服装,与泰安圆方商贸有限公司合作开发旅游服饰产品,与泰安市玉新服装厂合作开发校服。开展技能鉴定120人次,每年开展中职师生技能培训及师资培训2次,为服装企业进行职工培训2次,为全市出国劳务人员进行服装技能培训1次。面向农村开展技术服务2次。社会培训、技术服务总收入107.72万元。
与汶上职教中心和岱岳区职教中心建立对口支援,在人才培养模式创新、专业与课程建设、实践教学条件建设、师资培训等方面为对口支援学校提供指导和帮助。
7.发挥专业资源优势,带动艺术设计类专业群建设
服装设计专业发挥“省级特色专业”示范带动作用,围绕艺术专业岗位进行专业方向拓展,共同构建艺术设计专业群,新建珠宝首饰与鉴定专业,协助完善装饰艺术设计专业,目前,两个专业已建成为学院独具特色的专业。
通过服装设计专业课程体系建设,带动装饰艺术设计专业课程建设。通过服装设计专业实训条件建设,带动艺术类专业实训条件建设。通过服装设计专业师资队伍建设,带动装饰艺术专业师资队伍建设。围绕专业群积极搜集优质共享型专业教学资源,调动了广大教师的积极性。
(三)特色与创新
1.特色与创新
(1)推广发展“四位一体”专业文化教育
工作室教学项目设置均以泰山文化为主题,围绕泰山旅游产品开发开展工作,深入发掘泰安历史文化和产业要素,优化专业文化与企业文化交互平台,增进专业文化与经济社会的融合,充实发展融传统文化、泰山文化、企业文化、专业文化“四位一体”专业文化。工作室运行极大地提高了学生积极性,为学生的创业实践提供了平台。
(2)“三业”融通教育贯穿专业建设始终
专业建设中将职业教育、就业与创业教育贯通于课程改革中,通过“三业”融通教育提高学生的就业竞争力。在创新创业教育方面,积极响应国家关于大学生创新创业教育要求,以学院创建创业大学为契机,创设以创新创业为导向的创业环境,将教学管理与学生管理工作充分融合,发挥各工作室作用,引导学生建立社团,组建学生搭伙众创空间,初步建成了汇聚创意的“创梦工坊”,创建了大学生创新创业孵化基地。
(3)发挥辐射带动作用,推动泰安服装职业教育发展
依托服装工作室,开展全市服装师生技能培训,为服装企业进行职工培训,为全市出国劳务人员进行服装技能培训。与汶上职教中心、岱岳区职教中心开展对口帮扶支教,充分发挥了示范院校建设成果的辐射带动作用。
2.主要标志性成果
建设期内,获国家、省教学成果奖各3项。1项教改课题通过省级鉴定。立项省级创新课题1项。市级优秀重点社科课题1项。培养“山东省级教学名师”1名;1名教师被聘为山东省纺织服装专业建设指导委员会委员。建成“山东省级优秀教学团队”1个。“山东省级特色建设专业”1个。基本实现建成具有区域特色品牌专业建设目标。

项目九 珠宝首饰工艺及鉴定专业建设
(一)建设任务完成情况
项目立项建设以来,专业团队按照建设方案和任务书完成8个二级项目,63个子项目和169个验收要点,计划投入资金321.2万元,实际投入732万元,其中地方财政实际投入105万元,行业企业实际投入362万元,学校自筹实际投入265万元。主要建设任务指标完成情况见表2-38。
表2-38  珠宝首饰工艺及鉴定专业建设项目目标完成情况一览表
序号
建设内容
关键指标
单位
预期目标
实际完成
完成度
1
体制机制建设
专业建设指导委员会
个
1
1
100%


专业研讨会议
次
3
3
100%


专业指导委员会专家建议
份
3
3
100%


珠宝首饰工艺及鉴定专业指导委员会章程
个
1
1
100%


校企合作运行管理办法
项
3
3
100%
2
人才培养方案优化与人才培养模式改革
珠宝首饰工艺及鉴定专业调研报告
份
3
3
100%


珠宝首饰工艺及鉴定专业人才培养模式改革报告
份
3
3
100%


人才培养方案论证报告
份
3
3
100%


人才培养方案
份
3
3
100%


教学质量保障管理制度
份
3
3
100%
3
课程体系构建与核心课程建设
建设优质核心课程
门
3
3
100%


开发教材
部
2
4
200%


教学资源库
个
3
3
100%
4
教学团队建设
校外专业带头人
名
1
1
100%


校内专业带头人
名
1
1
100%


培养骨干教师
名
4
4
100%


培养双师素质教师
名
4
5
125%


双师素质比例
%
100%
100%
100%


引进教师
名
1
1
100%


院级教学名师
名
3
3
100%


聘请国家级工艺美术大师
名
2
3
130%


聘用兼职教师
名
4
5
125%


兼职教师库人数
名
4
10
250%
5
实训条件建设
新建实训室
个
6
10
190%


新增校外实训基地
家
7
8
110%


制定相关规章制度
个
3
3
100%
6
社会服务能力建设
科研与科技服务(项)
项
5
5
100%


申请专利(项)
项

2
200%


与企业设计开发产品
件
6
6
100%


完成各类社会培训
人时
320
570
170%


开展技能鉴定
人次

21
300%
7
专业群
培养专业带头人
人
1
1
100%


建设优质核心课程
门
2
5
250%


专业人才培养方案
份
3
3
100%


校内实训室
个
1
2
200%


校外实训基地
家
7
8
130%
(二)项目建设成效
1.体制机制建设彰显活力
(1)建立健全组织机构,践行专业建设指导委员会职能
珠宝首饰工艺及鉴定专业成立了工艺美术大师、行业协会专家和骨干教师等组成的专业建设指导委员会,审议本专业各年级人才培养方案,提出合理建议。并就专业课程体系改革与课程建设,师资队伍建设、实践条件建设、专业共享资源库、社会服务能力建设、专业群建设等方面给出了具体的指导和建议。
表2-39   珠宝首饰工艺及鉴定专业建设委员会组织结构

姓名
工作单位
职称/职务
主  任
谭毅
泰山职业技术学院
教授/系主任
副主任
张 莹
泰山职业技术学院工艺美术系副主任
副教授
顾  问
高毅进
江苏省扬州玉器厂总设计师
国家级工艺美术大师

宋建国
河北玉雕玉石研究所
国家级工艺美术大师

李博生
北京李博生造型艺术有限公司
国家级工艺美术大师

马培安
泰山职业技术学院副院长
副教授

张清民
泰山职业技术学院
副书记

王元成
泰山职业技术学院院长助理
研究员

李志强
泰山职业技术学院教务处主任
教授
企  业
专  家
冯  浩
泰安市泰山玉雕人才培养中心
副主任

申传禄
泰安逸致玉雕设计有限公司
总经理

魏伟
泰安先锋数控自动化设计中心
总经理

师秀娜
泰安遇石记玉石制品有限公司
总经理

许  新
泰安大观旅游商品开发有限公司
总经理
教  师
李厚清
泰山职业技术学院
副教授/教学副科长

李利鲁
泰山职业技术学院
高级工艺美术师

张林
泰山职业技术学院
讲师

张群
泰山职业技术学院
教研室主任

王传斌
泰山职业技术学院
讲师
(2)建立校企合作长效运行机制 
通过校院两级管理机制,在专业建设委员会指导下,建立了校企优势互补、资源共享机制、竞争激励机制、基于文化融合的校企沟通机制。制定各项规章制度,构建有利于技术型和高层次技能型人才培养的制度环境。充分利用校企双方人力、物力和信息资源,在学生实习、就业、人才培养、员工培训、产品开发等方面开展深度合作,逐步完善校企合作长效机制。
(3)完善教学质量保障体系
在学院教学质量保障体系框架下,在珠宝首饰工艺及鉴定专业建设委员会指导下建立了“多元化”的教学质量监控与保障体系,制定了《第三方人才培养质量评价办法》教学质量保障体系实施办法。落实了教学质量第一责任人制度,定期召开由企业、行业协会、学生等共同参与的教学质量评价会议,对创业成效进行考核与评价,并作为衡量专业人才培养质量的重要指标,完善了人才递进培养实施方案和教师工作管理办法,实现了校企人员互兼互聘。针对教学质量保障体系反馈问题,健全了处理机制,保证了及时有效的解决问题。
2.创新“校企共育,三位一体”人才培养模式
(1)完成专业调研报告,专业定位瞄准产业发展
珠宝首饰工艺及鉴定专业邀请企业专家为专业发展把脉,共商人才培养方案;走访兄弟校院,开展专业调研,寻找专业特色发展方向;发展专业人才培养课题研究,探索专业人才培养模式。适合泰安地区社会经济发展,3年内组织10次专业调研,通过走访包括北京李博生造型艺术有限公司在内的18家区域内外企业,充分把握企业用人需求,对接宝玉石珠宝设计、鉴定、营销等岗位,确定专业人才培养目标和定位。
(2)以“诚信”为核心,实施人格塑造工程
以“诚信”为核心,突出德育培养,通过《思想政治》、《国学与人生》课程,提高学生的人文素养,引导学生养成健全的人格。定期开展自信心训练、素质拓展训练等形式多样的心理健康教育;与企业共同制定学生职业规划,利用与企业建立的e+创客工作室,通过项目让学生参与企业产品营销活动,通过入学教育、就业指导、养成教育等方式强化职业素质训练,提高学生的职业素养。
(3)以“工匠精神”为引领,创新能力培养工程
结合专业特点,以“工匠精神“为引领,实施有效课程改革,通过专业基础课程、专业核心课程、综合实践课程、顶岗实习,培养设计能力和工艺制作能力;引导学生积极参与课题研究与技能大赛, 2项市级大学生创新课题申报成功,3年内学生参加全国职业院校技术技能创新交流赛获2项二等奖,5项三等奖,参加泰安市职业学校技能大赛获2项一等奖、3项二等奖、5项三等奖。学生利用专业优势组建了墙艺社团、陶艺社团、木雕社团、211设计工作室、搭伙众创工作室5个社团,进行创新产品研发,提高创新和社会服务能力。
(4)以专业理念为核心,培养学生人文素质
“切磋琢磨,精雕细刻”是珠宝首饰工艺及鉴定专业的核心理念,围绕这一理念,通过《人文素质教育》《中国工艺美术史》《玉石材料学》《玉器鉴赏》课程,提高审美情操和文化品位。利用图书馆藏书和电子书库,积极鼓励学生进行课外阅读,增大学生知识面,与企业开展“名人进校园”系列活动,先后有中国工艺美术大师李博生、宋建国,泰安珠宝首饰行业协会秘书长冯浩来我院进行讲座活动,同时积极组织学生参观2015年中国天工奖、2015年中国九届美展,2015中国山水画大展(泰安)等展览,进行观摩学习,开阔学生视野,提高审美水平。


		图2-28   “校企共育,三位一体”人才培养实施结构图
图2-28   “校企共育,三位一体”人才培养实施结构图
专业基础课程、专业核心课程、综合实践课程、顶岗实习
专业基础课程、专业核心课程、综合实践课程、顶岗实习
参加科研课题、开展设计大赛
参加科研课题、开展设计大赛
组建大学生创新工作室

组建大学生创新工作室

学生参观展览、组织展览
学生参观展览、组织展览
入学教育、养成教育
入学教育、养成教育
文化素养课程
文化素养课程
  开展“名人进校园”系列活动
  开展“名人进校园”系列活动
人文课程和艺术鉴赏课程
人文课程和艺术鉴赏课程
开展“读书月”活动
开展“读书月”活动
开展自信心训练、素质拓展训练
开展自信心训练、素质拓展训练
心理健康教育
心理健康教育

校企共育

校企共育
三位一体
 
三位一体
 
德育首位
德育首位
文化本位
文化本位
能力本位
能力本位
人文素质教育工程
人文素质教育工程
健全人格塑造工程
健全人格塑造工程
创新制作能力培养工程
创新制作能力培养工程
三大工程
 
三大工程
 
具备优良的诚信品质和较强的法律意识
具备优良的诚信品质和较强的法律意识
培养内容
 
培养内容
 
具备较好的人文修养和较高的文化品位
具备较好的人文修养和较高的文化品位
具备扎实的专业技能、较强的创新制作能力
具备扎实的专业技能、较强的创新制作能力
培养重点
 
培养重点
 
以人文素质和传统文化教育为重点
以人文素质和传统文化教育为重点
以培养学生的创意和创新制作能力为重点
以培养学生的创意和创新制作能力为重点
以培养学生诚信品格为重点
以培养学生诚信品格为重点
健全人格
健全人格
扎实的专业技能、较强的创新制作能力

扎实的专业技能、较强的创新制作能力

培养结果

培养结果

珠宝首饰工艺及鉴定专业技术技能型人才
珠宝首饰工艺及鉴定专业技术技能型人才
较好的人文素养
较好的人文素养
培养措施

培养措施


  3.构建了以“工艺制作、创意设计、鉴定营销”为核心能力的专业课程体系
 经过三年建设,构建以职业能力培养为核心,充分体现两个“密切”(专业发展与行业需求密切相关、专业能力与岗位核心技能密切相连)、三个“融合”(学习方式与工作过程融合、专业技能与岗位要求融合,实训目标与职业资格融合),以“工艺制作、创意设计、鉴定营销”为核心能力的课程体系。采用与人才培养模式改革相适应的教学方法和手段,全面提高教学效果,提升学生的职业核心竞争力。
在专业建设指导委员会的指导下,完善校企共同开发课程制度和课程建设负责人制度等建设,加强课程建设管理。以真实工作任务驱动和项目实施为基础,进行了课程体系重构,修订了人才培养方案,并积极进行课程改革,三年内顺利完成《线描实训》、《泥塑实训》、《玉雕工艺实训》3门优质核心课程建设工作,课程资源量满足专业教学需求。各门课程已在校本资源平台使用,同时增添专业建设等资源,为下一步构建专业资源库打下基础。
表2-40  珠宝首饰工艺及鉴定专业核心课程资源分布表
课程
课程标准
总体设计
单元设计
实训指导书
案例库
微课视频
线描实训
√
√
√
√
√
√
泥塑实训
√
√
√
√
√
√
玉雕工艺实训
√
√
√
√
√
√


图2-29   以“工艺制作、创意设计、鉴定营销”为核心能力的课程体系

目  标
目  标
核心职业能力
核心职业能力
课程
课程
技能要求
技能要求
岗   位
岗   位

平面设计、包装设计
平面设计、包装设计
专业技能拓展能力
专业技能拓展能力
培养产品展示、包装设计能力
培养产品展示、包装设计能力
培养法律意识、社会责任意识、协作和沟通能力、较好的人文素养。
培养法律意识、社会责任意识、协作和沟通能力、较好的人文素养。
培养珠宝玉石作品的鉴别能力,客户沟通能力
培养珠宝玉石作品的鉴别能力,客户沟通能力
培养绘图造型、创意能力,.运用计算机创意能力
培养绘图造型、创意能力,.运用计算机创意能力
培养雕刻、打磨、抛光的处理能力,三维软件、雕刻机操作能力

培养雕刻、打磨、抛光的处理能力,三维软件、雕刻机操作能力

通用素养
通用素养
中国玉器概论

中国玉器概论

创意设计能力
创意设计能力


工艺制作能力


工艺制作能力

鉴定营销能 力

鉴定营销能 力
能识别、鉴定玉石的类别材料、具备一定的宝玉石鉴定常识
熟悉市场营销知识
了解营销流程、途径和方式
能识别、鉴定玉石的类别材料、具备一定的宝玉石鉴定常识
熟悉市场营销知识
了解营销流程、途径和方式
通过市场调查,确立设计定位,创作款式新颖的造型,产品效果图绘制
通过市场调查,确立设计定位,创作款式新颖的造型,产品效果图绘制
能运用电脑设计制作模型、能按工艺流程完成三维扫描、打坯、立体雕刻等工艺制作
能运用电脑设计制作模型、能按工艺流程完成三维扫描、打坯、立体雕刻等工艺制作
能完成宝石造型,确定材料的尺寸规格
能完成雕、镂、打磨、抛光等工艺
能完成宝石造型,确定材料的尺寸规格
能完成雕、镂、打磨、抛光等工艺
泥塑实训

泥塑实训

思想道德修养与法律基础
思想道德修养与法律基础
就业指导、心理健康教育
就业指导、心理健康教育
计算机文化基础

计算机文化基础

大学英语、人文素质教育
大学英语、人文素质教育
玉雕器皿设计与制作
玉雕器皿设计与制作
小型挂件设计与制作
小型挂件设计与制作
玉石材料与工艺
玉石材料与工艺
玉器市场营销与管理
玉器市场营销与管理
素描、图案
素描、图案
设计线描实训

设计线描实训

电脑设计基础
电脑设计基础
玉石鉴赏
玉石鉴赏
中国工艺美术史、泰山文化


中国工艺美术史、泰山文化


电脑三维造型设计与制作

电脑三维造型设计与制作

玉雕动物、人物设计与制作
玉雕动物、人物设计与制作
珠宝鉴定与销售岗位
珠宝鉴定与销售岗位
珠宝造型设计岗位
珠宝造型设计岗位
珠宝工艺制作岗位
珠宝工艺制作岗位
4.专业教学团队素质优良
通过聘请、培养、引进、聘用等途径,引进专任教师1名,培养校内专业带头人1名,聘请校外专业带头人1名,培养骨干教师4名,双师素质教师比例达到100%。聘请企业业务骨干和能工巧匠兼职教师5名,聘请建设顾问中国工艺美术大师3人(中国工艺美术大师),建设了一支由工艺美术大师、专任教师组成的双师型教学团队。
实施“名师递进培养工程”, 递进培养院级“教学新星、教坛英才、教学名师” 3人,教师境外培训1人次; 5名教师参加各类培训12人次。文化厅科研课题3项,市科研项目1项,院级教改项目1项,山东省现代学徒制试点改革项目1个,团队教师获得学院讲课比赛一等奖1人,山东省工艺美术雕刻比赛二等获奖1人。泰安市职业学校技能比赛指导教师2人,外观专利2项。山东省艺术职业教育专业指导委员会委员1名。
5.实训基地功能趋于完善
(1)校内实训条件建设
三年建设期内,新建校内实训室10个,总投资612万元。
表2-41  新扩建校内实训基地及其功能
序号
实训室名称
建设时间
主要功能
1
泥塑实训室
2015
立体造型
2
珠宝鉴定实训室
2016
宝玉石鉴定
3
玉雕实训中心
2014
雕刻工艺
4
大师工作室
2015
雕刻工艺
5
数码雕刻实训室
2015
数码雕刻
6
珠宝设计实训室
2015
造型设计
7
宝玉石展室
2015
展示研讨
8
产品造型工作室
2015
学生课外实训
9
E+创客工作室
2015
学生课外实训
10
CAD实训室
2016
电脑造型设计
(2)校外实训条件建设
三年建设期内,新增校外实训基地5个,使校外实训基地增至8家。
表2-42   珠宝首饰工艺及鉴定专业校外实训基地一览表
序号
合作企业
建设时间
针对岗位群
1
泰安市遇石记玉石制品有限公司
2014
珠宝营销
2
泰安鸿芳玉雕设计工作室
2015
玉雕工艺
3
北京李博生造型艺术有限公司
2014
玉雕工艺
4
泰安逸致玉雕设计有限公司
2013
数码雕刻
5
泰安玉传琦玉雕艺术中心
2014
珠宝营销
6
泰安大观旅游商品开发有限公司
2014
旅游工艺品
7
泰安先锋数控自动化设计中心
2013
数码雕刻
8
泰安泰山玉研发有限公司
2013
玉雕工艺
6.社会服务能力显著提升
(1)技术服务,校企互惠共赢
三年建设期内,与泰安逸致玉雕设计有限公司泰、安先锋数控自动化设计中心、泰安与玉遇石记玉石制品有限公司等提供5项技术服务,合作开发6种新产品,其中申请了两项外观专利。与逸致玉雕有限公司合作申报的泰安市科技计划课题《泰山玉数字雕刻技术研究》1项
(2)社会培训,服务区域需求
三年建设期内,联合泰安市珠宝玉石首饰行业协会鉴定玉石雕刻技术21人,玉文化讲座培训从业人员50人,手工雕刻培训人员10人,数码雕刻人员培训5人,共计培训学员65人、570人时。
7.专业群建设持续发展
(1)带动师资队伍建设
装饰艺术设计专业积极组织骨干教师参加各类技术培训,孟祥勇参加山东省职业院校艺术设计专业教师培训,阴楠楠老师参加国家职业院校艺术设计专业教师培训,3年建设期内培养专业带头人1人,骨干教师4人。服装设计专业被评为省级教学团队,培养省级教学名师1名,建立了一支专业技能强,有丰富实践经验的艺术设计专业群教师队伍。
(2)完善实训室建设
通过专业群建设,装饰艺术设计专业在原有的实训条件的基础上,购置了激光雕刻机、写真机、条幅机、3D打印机等设备,完善了产品造型工作室,新建一个CAD实训室,大大改善了教学实训环境,为装饰艺术设计的人才培养创造了有利条件。服装设计专业经过名校建设,改扩建实训室10个,投入资金150万元,教学实训条件进一步完善。
(三)特色与创新
1.国家级工艺美术大师参与专业建设
珠宝首饰工艺及鉴定专业属于传统手工业范畴,要想办好专业,离不开传统工艺美术大师参与,为此我们先后聘请了国家级工艺美术大师高毅进、宋建国、李博生作为我们的客座教授与专业建设顾问,李博生大师2014年来我院面向全市进行了两场玉文化讲座,反响强烈, 2015年,2013级珠宝首饰工艺及鉴定专业4名学生被送往李博生大师工作室学习研修,初高级专业人才培养的“学院+工作室”人才培养的模式初步形成。为未来高级玉雕人才的培养奠定了基础。
2.学生创新创业能力增强
珠宝首饰工艺及鉴定专业把搭建科技创新学生社团,作为教学工作的组成部分和一种特殊的教学组织形式,经过多年运行,现有墙艺社团、陶艺社团、木雕社团、211设计工作室、搭伙众创工作室等5个,通过社团活动,教师提高了自身的专业实践能力和教学水平,改革了实践教学模式,激发了学生创造思维和学习兴趣,培养了学生的创新意识和创新能力,提高了学生运用所学知识解决实际问题的能力,学生创新成果显著,创客作品曾在全国职业院校创新技能交流赛、泰安市创客嘉年华展演中屡次获奖,受到社会的好评。
3.省级“玉石雕刻”现代学徒制试点项目
2016年3月,经山东省教育厅批准,珠宝首饰工艺及鉴定专业“玉石雕刻”现代学徒制项目被列为省级试点专业,在专业建设成果的基础上进一步加强校企合作,与企业共同确定人才目标的定位、联合招生招工方式、深化教学改革,构建满足学徒培养的教学方案和标准以及课程体系的开发等,进而形成现代学徒制管理制度体系、现代学徒制人才培养方案、适合现代学徒制的课程等。本项目形成的适应行业企业发展需求、产教深度融合的具有引领和示范作用的经验和成果,对珠宝首饰工艺及鉴定专业及相关专业实施现代学徒制具有较好的推广应用价值,不仅可以在本校推广,而且也为其他高职院校实施现代学徒制提供案例。

四、特色项目建设成效
项目十    校企合作体制机制建设
(一)建设任务完成情况
项目立项建设以来,按照建设方案和任务书完成3个二级项目,42个验收要点。主要建设任务指标完成情况见表2-46。
表2-43  创新校企合作体制机制建设目标完成情况一览表
建设内容
建设任务
单位
预期目标
实际完成
完成度
健全校企合作组织,完善校企合作制度

市政府《关于加快泰山职业技术学院省级特色名校建设的决定》文件
份
1
1
100%

成立校企合作委员会及组织机构、校企合作委员会会议制度
个
3
3
100%

泰山职业技术学院校企合作管理办法
个
1
1
100%

校企合作委员会年度工作计划、总结
份
6
6
100%

校企委员会会议纪要、活动记录
份
6
6
100%

校企合作理事会组成文件
份
2
2
100%

教师企业顶岗锻炼制度、学生顶岗实习管理办法、校外实训基地建设制度
份
3
3
100%

“厂校一体,资源共享”的校企合作总结
份
2
2
100%

兼职教师参与人才培养的相关材料
份
2
2
100%

教师企业顶岗锻炼考核鉴定表
份
1
1
100%

顶岗实习计划、总结、记录
份
1
1
100%

校外实训基地一览表
个
1
1
100%

教师服务企业记录
个
1
1
100%
拓展校企合作空间,促进校企深度融合

校企合作协议
份
8
103
1200%

校企合作总结
份
3
3
100%

校企合作材料
份
200
400
200%

学院新建教学做一体化楼
座
1
1
100%

校企合作实习基地
处
30
68
227%
完善“厂校一体,资源共享”的校企合作模式
“校中厂”、“厂中校”、“教学型工厂”的项目计划书和验收材料
份
17
34
200%

订单班专业
个
6
10
167%

校区合作
个
5
5
100%

菜单式培训
名
5000
5000
100%
(二)项目建设成效
1.健全校企合作组织,完善校企合作制度
(1)搭建起了校企合作平台,形成了“政、行、企、所、校”联动的合作体制机制
依托市政府牵头、多方参与,泰安市发改委、经信委、国资委、财政局、教育局、科技局、人社局等部门,以及有关行业协会、企业单位和职业院校,按照“政府主导、行业指导、企业参与、共同发展”的基本要求,组成泰安市校企合作联席会议,成立泰山职业技术学院校企合作理事会和泰安市机电职教集团,构建了人才共育、过程共管、成果共享、责任共担的紧密型合作办学体制机制,推动合作办学、合作育人、合作就业、合作发展。
(2)健全校企合作组织机构
学院成立由院长担任主任的校企合作委员会,负责校企合作发展规划、专业建设、人才培养改革和发展等重大事项的决策和部署。委员会下设校企合作办公室,负责健全和完善校企合作制度,推进实施校企合作具体事宜;系部成立相应组织,具体落实学院校企合作工作部署。
(3)健全校企合作制度,完善激励和约束机制
修订了《泰山职业技术学院校企合作管理办法》,制定《校企合作工作考核指标体系》、《校企合作工作委员会会议制度》等一系列制度,建立专业技术职务聘任、考核制度,深化各类人员的人事、分配制度改革,形成教师密切联系企业的机制;按照“工作项目化、管理精细化、服务人性化”的原则,进一步细化各岗位工作职责;各系根据专业制定校企合作方案、实施细则等;建立精细化管理考核体系,实现全过程监控,保障校企合作保持长期、稳定、健康发展,完善了校企合作的推动、激励和约束机制。
表2-44 学院校企合作制度一览表
序号
制度名称
1
关于调整校企合作工作委员会的通知(含《校企合作委员会会议制度》
2
校企合作委员会会议制度(见《关于调整校企合作工作委员会的通知》)
3
泰山职业技术学院校企合作管理办法(含《泰山职业技术学院校企合作考核指标体系》
4
泰山职业技术学院校企合作考核办法(考核指标体系)
5
泰山职业技术学院校企合作实施方案
6
校企合作理事会章程
7
关于进一步加强校企合作开展社会服务工作的实施办法
8
泰山职业技术学院校中厂、厂中校运行管理办法
9
兼职教师聘任制度
10
教师企业顶岗锻炼制度
11
学生顶岗实习管理办法
12
校外实训基地建设制度
13
教师服务企业制度
2.拓展校企合作空间,促进校企深度融合
(1)拓展校企合作空间
学院根据《泰安市国民经济和社会发展第十二个五年规划纲要》和名校建设方案,结合特色专业和重点项目,加强与行业内龙头企业的有效沟通,加强与国际、国内大中型企业的合作力度,进一步提高校企合作的质量,扩大合作领域,深化合作层次,更好地满足“校企融合,实境育人”的要求,名校建设期间,三年内新增73家校企合作基地,覆盖学院所有专业,每个专业达2个以上。
表2-45  校企合作基地一览表
序号
校企合作基地
1
中铁十四局集团电气化工程有限公司
2
新泰和康源牧康饲料有限公司
3
武汉华中数控股份有限公司实习基地
4
沃尔重工
5
潍坊瑞麦食品有限公司
6
潍坊戈尔声学实习基地
7
万达电子
8
泰盈科技
9
泰山学院机械与工程学院
10
泰山蒙牛集团
11
泰山集团
12
泰安正大园林科技有限公司
13
泰安遇石记实训基地
14
泰安玉之源实训基地
15
泰安玉传琦实训基地
16
泰安逸致玉雕实训基地
17
泰安一开电器有限公司实习基地
18
泰安鑫民科技农业有限公司
19
泰安先锋数控实训基地
20
泰安泰和电器电力设备有限公司实习基地
21
泰安泰丰园艺有限公司
22
泰安市中天算量有限公司
23
泰安市建筑技术协会
24
泰安市东城建筑工程有限公司
25
泰安鸿芳玉雕实训基地
26
双驰汽车有限公司
27
寿光市蔬菜病虫害防治协会
28
山东众成项目管理有限公司
29
山东兴润建设集团有限公司
30
山东沃尔重工科技有限公司实习基地
31
山东泰山普惠建工有限公司
32
山东森宇重工机械有限公司实习基地
33
山东华龙建筑安装有限公司
34
山东禾宜生物科技有限公司
35
清华斯维尔科技有限公司
36
山东岱银雷诺制衣公司
37
青岛完美动力
38
青岛海利尔药业有限公司
39
青岛DND集团实习基地
40
诺顿网络
41
济南禾立达种子有限公司
42
华鲁锻压
43
荷兰bejo(潍坊)种子公司
44
浩天网络
45
广州南方测绘仪器有限公司
46
东方格兰(北京)种业有限责任公司
47
北京李博生大师玉雕实训基地
48
中科招商
49
丽泰金控
50
中国重汽技术实习基地
51
泰达长城汽车实习基地
52
泰安沃通汽车销售服务实习基地
53
泰安市金江汽车贸易实习基地
54
泰安市华君汽车销售服务实习基地
55
泰安胜力汽车销售服务实习基地
56
泰安青年汽车制造实习基地
57
泰安交通汽车制造实习基地
58
泰安航天特种车实习基地
59
上海中锐教育实习基地
60
卡帕尔汽车养护中心实习基地
61
东营吉奥汽车实训基地
62
泰安市东尊华美达实训基地
63
泰安市岱庙景区实训基地
64
泰安市诚之旅国际旅行社实训基地
65
泰安正元名饮酒店集团实训基地
66
泰安新闻国旅
67
易游天下实训基地
68
泰山方特欢乐世界实训基地
69
泰山宾馆实训基地
70
泰安联合假期国际旅行社
71
泰安康辉国旅
72
泰安巨鼎餐饮有限公司实训基地
73
泰安花样年华景区
74
青岛威斯汀大酒店实训基地
75
青岛海情大酒店实训基地
76
青岛德泰大酒店实训基地
77
济南舜和酒店集团实训基地
78
北京泰山御座宾馆实训基地
79
G伊诺特(烟台)有限公司
80
青岛蜗牛影视动画公司
(2)促进校企深度融合
构建开放办学格局,加强校企智力资源、人力资源、设备资源、生产资源、市场资源的整合,坚持产学研相结合,合作建设专业实验室,学院新建“教学做一体化”大楼,同时根据合作企业生产经营的季节性特征和周期性特点,灵活安排工学交替,理论与实践结合更紧密。如建筑企业冬天不能施工,实习安排在暑假前后;园林园艺专业依据作物生长周期安排实习;食品专业、旅游实习则考虑安排在旺季,软件企业与学校合作更倾向采取弹性安排,因其委托学校利用学生实习完成的服务外包项目来单通常是随机的。
实施访问工程师项目,敦聘客座教授。专业教师每年利用寒暑假进行4周以上的生产实践锻炼;每年选派至少20名专业教师到企业进行为期半年的实践锻炼,培养一支融理论教学、实训指导、项目开发为一体的“一专多能”的师资队伍。聘请泰山科学院原院长杨鸿问等知名企业家、专家学者为学院客座教授。
表2-46  泰山职业技术学院客座教授一览表
姓名
工作单位职务
聘任系别
杨鸿问
泰安市科学技术协会原副主席,泰山科学院原院长,中国机械工程学会高级会员,研究员
学院
苏立华
中国保丽洁集团董事长,高级经济师
学院
姜润喜
泰安鲁普耐特塑料有限公司副总经理,教授级高级工程师
成教处
王志利
山东泰盈科技有限公司董事长
财经系
陈文涛
泰安致信联合会计师事务所主任会计师,高级会计师
财经系
赵德政
山东沃尔重工科技有限公司副总经理  高级工程师
机电系
赵令印
山东能源重装集团泰装工程制造有限公司总工程师,高工
机电系
刘桂军
泰丰园艺有限公司总经理,高级农艺师
生物系
刘庆山
山东泰医医疗器械有限公司总经理
汽电系
纪恩峰
山东岱银进出口有限公司技术部部长
工美系
王文革
泰安如意科技时尚产业有限公司常务副总经理,
工美系
殷耀静
青岛英特软件有限公司总经理,济南工程职业学院工程系副理事长高级工程师
建筑系
郭敬元
泰安市城市建设设计院雕塑创建室主任 ,高级工程师
工美系
谢立平
北大青鸟泰安校区校长 高级工程师
信息系
彭峙
泰安市东城建筑工程有限公司副总经理,高级工程师
建筑系
展鹏
泰安银行董事长,高级经济师
财经系
杜欣涛
泰安巨鼎餐饮有限公司董事长
旅游系
(3)成立创新创业教育中心,搭建泰安创业大学、泰安市第一公共实训中心平台,“定点培训机构”通过验收,为大学生创新创业就业、科技孵化服务。
名校建设期间,学院进行就业创业培训资源整合,成立创新创业教育中心,泰安创业大学在我院筹建挂牌,泰安市第一公共实训中心在学院批复成立,成功中标泰安市和泰山区“加强就业培训提高就业与创业能力五年规划(2014-2018)培训项目”定点培训机构,通过市政府政策资金支持,打造成为本地区综合、开放、共享的学生实训中心、技能培训中心、技术研发中心和技术服务中心,形成大学生创新创业教育培训、创业政策咨询、创业科技孵化等多功能于一体的创新创业人才服务体系,
3.完善并创新“厂校一体,资源共享”的校企合作模式
(1)“厂中校”“校中厂”建设
学院高度重视校企深度融合,目前学院建有7个 “厂中校”、16个“厂中校”、4个教学型工厂,订单班专业9个、22个订单班。通过“厂校一体,资源共享”的校企合作,学生参与企业生产的全过程,在真实的环境中提高和掌握技能,并为企业提供技术和劳动力支持,取得生产与教学、实习就业多赢效益。 
表2-47  “厂校一体,资源共享”建设成果一览表
序号
建设内容
建设任务
单位
预期目标
实际完成
完成度
1

校中厂

济南达内科技有限公司
个

6


7


117%



甲古文(山东省)OAEC人才产业基地
个





泰安市泰山区先锋自动化数控设计中心
个





泰安市遇石记玉石制品有限公司
个





泰盈科技
个





山东泰山科技股份有限公司
个





上海中锐教育投资有限公司
个



2
厂中校

北京李博生造型艺术有限公司
个
10

16



160%




华泰种业有限公司
个





蓝鸥科技
个





青岛DND集团胶州分公司
个





山东沃尔重工科技有限公司
个





泰安泰丰园艺有限公司
个





潍坊戈尔声学有限公司
个





济南吉利汽车制造厂
个





长城汽车制造有限公司
个





上海平安保险公司
个





泰安宝龙新时代商贸有限公司
个





泰安市永信会计教育培训学校
个





泰山天成国际旅行社
个





泰山瀛泰国际旅行社
个





泰山封禅大典
个





方特欢乐世界
个



3
订单班
玉雕造型班
个


6



22



383%



旺旺班
个





统筹E算特色培养班
个





数码雕刻艺术班
个





甲古文班
个





航天特车班
个





达内班
个





汽车电子班
13个





舜和酒店管理精英班
个





宝龙新时代商贸订单班
个



4
教学工厂
e+创客工作室
个
3

4

130%


惠普济宁-泰职
个





汽车养护中心
个





数码雕刻工作室
个



(4)创新校区、校行、校县合作模式
根据学院专业综合优势,结合区域经济发展和产业结构状况,联系行业、社区、产业集群等区域群体,进行高层次、全方位、多元化、规模化的区域合作。
学院与省级经济开发区、山东岱岳工业园区—“青春创业园”深化合作,与泰安市高新技术开发区签署战略合作协议,并与所属泰开集团、青年莲花汽车、航天特车、太和集团、尤洛卡股份等泰安市骨干企业培植工程进行全方位、多元化合作。
与泰安市行业管理办公室及其行业协会—泰安市传统食品协会、泰安市工艺美术行业协会、泰安市冷链物流协会、泰安市包装印刷行业协会、寿光市蔬菜病虫害防治协会,共同签署了校行战略合作协议,共建基地,互认挂牌。
与县市区政府及所属中职学校进行战略合作,与东平县人民政府战略合作成立泰山职业技术学院东平移民学院揭牌。







图2-30 成立泰山职业技术学院东平移民学院揭牌
5.加强菜单式服务  
加强社会服务能力,按企业要求进行菜单式培训。学院利用师资、教学做一体化实训楼、多媒体教室、公共实训中心、技能鉴定站,承担一定的企业员工培训、鉴定工作。学院承担泰安市退伍军人培训,汽电系汽车电子技术专业充分利用汽车行业特有工种鉴定培训,旅游系导游证、营销师、策划师、助理计调师职业培训,信息系为合作企业提供员工岗前培训服务,生物系农技人员、新型农民阳光培训,机电系承担武汉华中数控华东地区人员培训。通过菜单式服务,建立稳定可靠的校企合作网络,获得众多的企业支持,拓宽了就业渠道,服务与地方经济发展。
(三)特色与创新
1.政、行、企、所、校 “多方联动、协同发展”的校企合作体制机制日臻成熟
泰安市政府牵头政府部门、企业、学校组织校企合作联席会议,学院泰山职业技术学院校企合作理事会和泰安市机电职教集团;与中锐集团、东方领航、沃尔重工等80多家企业实现校企合作规模化;与济宁汶上县、泰安市5个县市区加强合作,签订学院与职教中心合作办学协议,成立泰山职业技术学院东平移民学院;与泰安市行业管理办公室及其行业协会—泰安市传统食品协会、泰安市工艺美术行业协会、泰安市冷链物流协会、泰安市包装印刷行业协会、寿光市蔬菜病虫害防治协会等19个行业协会建立紧密联系,共同签署了校行战略合作协议,共建基地,互认挂牌。
2.校企合作硕果果累累,亮点纷呈  
新建一座2万平米的“教学做一体化”实训楼(行知楼),校园设立校企合作园景观,泰安市泰山玉雕人才培养中心在学院成立,泰安创业大学挂牌成立,泰安市第一公共实训中心批复成立,学院成功中标“泰安市加强就业培训提高就业与创业能力五年规划(2014-2018)培训项目”和“泰山区加强就业培训提高就业与创业能力五年规划(2014-2018)培训项目”定点培训机构,珠宝首饰工艺及鉴定专业现代学徒制试点获2016年山东省职业院校现代学徒制试点项目,引入企业精细化管理理念,全面推行“7S”管理,学院育才旅行社、东方驾校成立运营。
项目十一    教学质量监控与保障体系建设
(一)建设任务完成情况
项目立项建设以来,教学质量监控与保障体系建设团队按照建设方案和任务书完成6个二级项目,23个子项目和54个验收要点,全部按照项目建设任务要求完成任务,达到预期建设目标。
表2-48  教学质量监控与保障体系建设目标完成情况一览表
序号
建设内容
关键指标
单位
预期目标
实际完成
完成度
1
构建监控主体
校企合作理事会
个
1
1
100%


校企合作委员会
个
1
1
100%


专业建设委员会
个
1
1
100%


专业人才培养状况报告
年
0
1
超额


年度人才培养质量报告
年
3
3
100%


应届毕业生社会需求与培养质量跟踪评价报告——麦可思
年
3
3
100%
2
目标与标准体系建设
教学管理制度文件
个
25
25
100%


专业标准
个
29
29
100%


课程标准
个
410
410
100%


专业人才培养方案
年级
3
3
100%
3
监控与评价体系建设
教师职教能力培训与测评
人次

296
超额


新增设专业
个

5
超额


各系部教学考核结果
年

3
超额


年度学院讲课比赛
年

3
超额


院级教材评选
年

1
超额


院级教科研奖励
年

2
超额


师生员工奖励
年

3
超额


年度考核实施意见
年

3
超额


专业技术职务评聘办法、晋级聘任办法
年

2
超额
4
反馈与调控体系建设
各系部分学期教学常规整改报告
学期
5
5
超额


学院教学监控评价分析报告
学期
5
5
100%


人才培养评估专家意见
份
1
1
100%


人才培养评估整改报告
份
1
1
100%
5
完善人才培养工作数据采集平台

人才培养工作状态数据采集与上报工作实施方案
份
3
3
100%


人才培养工作状态数据采集与管理平台
个
1
1
100%
6
教学管理信息平台建设
教务管理信息系统
个
1
1
100%


专业教学资源库平台
个
1
1
100%


专业教学资源库
个
0
29
100%


专业核心课程
门
0
48
100%


顶岗实习信息管理平台
个
1
1
100%
按照“完善平台、健全制度、过程管理、反馈改进”的指导思想,以提高人才培养质量为根本,遵循高职教育规律,坚持“学院评价为核心、教育主管部门引导、行业企业广泛参与”的原则,以政府(教育主管部门)、行业企业、学校和社会公众为4个监控主体,健全目标与标准、监控与评价和反馈与调控3个体系,完善人才培养工作状态数据和教学管理信息2个平台,构建“432”教学质量监控与保障体系,全面提高人才培养质量。










图2-31 教学质量监控与保障体系结构图
(二)项目建设成效
1.构建四个监控主体
(1)发挥市职教联席
“4”个监控主体,即政府(教育主管部门)、行业企业、学院、社会公众。市政府牵头,由发改委、经信委、国资委、财政局、教育局、科技局、人社局等部门,以及有关行业协会、企业单位和学院共同组成泰安市职教联席会议,发挥在政策出台、资源整合、规划指导、资金筹措、基础建设等方面的决策、咨询、协调、监督和推动作用,推进学院与行业企业在在人才培养、产学研结合、毕业生实习就业、科技开发等方面的全方位合作。
(2)健全组织机构
学院成立了校企合作委员会,创新“政、行、企、所、校”联动的合作体制机制,制定委员会的议事规则,明确了职能,全面推进学院真正实现产教融合。
学院健全了专业建设指导委员会,进一步明确了院系两级专业管理权责,通过定期召开委员会议,进一步扩大了与行业企业、理事会成员单位的合作力度,从专业设置的论证、人才培养目标的定位、人才培养模式的实施、人才培养方案和课程标准的制定、毕业生跟踪调研等多个方面开展深入合作,成效显著。 
(3)完善内部保证机构
学院完善了内部两级四层次的教学质量保障组织机构,形成了院系(部)二级管理机构,学院、系(部)、教研室、学生4个层面的组织与管理机构。完善了学生信息员教学信息反馈制度。实行了年度人才培养质量报告、专业人才培养状况报告制度,2013-2015年度人才培养质量年度报告在学院网站上发布,向社会展示了学院的精神风貌和办学特色,办学质量不断提升。
(4)重视第三方评价工作
学院非常重视对毕业生的跟踪调查,请麦可思公司进行了第三方社会需求与培养质量跟踪评价,及时全面的获得了毕业生就业能力的真实数据,如毕业生社会满意度、就业率、就业对口率、初次就业起薪值、半年内离职率、自主创业率与三年内毕业生职位晋升率等,全面客观地反映了学院的人才培养质量。
2.目标与标准体系建设
学院坚持实施“德技并重,理实一体”的人才培养模式,构建了“平台+模块”的课程体系,全面推行“分阶段、项目化、协同式”实践教学模式,人才培养质量不断提升。引进企业技术标准和职业资格标准,根据国家和省有关专业规范要求修订了专业标准。实施了课程体系优化工程,把创新创业教育纳入人才培养方案,树立“以学生发展为中心”核心价值理念,突出“求实”“求新”“求高”品质的“服务”价值取向,按照规范流程修订完善了2013级、2014级、2015级专业人才培养方案,修订了410门课程标准。
修订、完善了教学管理制度、文件等25个,进一步明确、规范了常规教学要求、标准和程序,并将教学系部常规教学管理质量检查评价和监控教育教学质量的重要依据。
3.监控与评价体系建设
(1)办学定位
学院制定了《关于进一步明确学院办学目标、定位与人才培养模式的意见》(泰职院字〔2013〕77号)和《关于实施“德技并重、理实一体”人才培养模式的意见(试行)》(泰职院字〔2014〕15号),形成人才培养模式创新长效机制。
办学目标定位:建成涵盖旅游文化服务、经贸类、建筑类、制造类、农业类等专业,特色鲜明、在全国有一定知名度的优质专科高等职业院校。
服务面向定位:立足泰安、面向全省、辐射全国,重点服务以旅游文化为重点的现代服务业、先进制造业和现代农业等产业。
专业发展定位:以制造类和财经类专业为优势,以旅游文化为特色,旅游文化、机电装备与制造、经贸服务、建筑工程、艺术设计与加工、现代农业、信息技术、汽车与电气技术、教育9类专业协调发展。
人才培养目标定位:秉承包容、和谐、进取、担当的泰山精神,实施“德技并重,理实一体”的人才培养模式,培养“德高技高”的技术技能型人才。
(2)人才培养工作评估
2013年10月,学院顺利通过“高等职业院校人才培养工作评估”,根据教育部、省教育厅文件精神和评估专家组反馈的意见,结合学院“十二五”事业发展规划要求,在全面总结评建工作的基础上,形成评估整改报告,积极进行整改,于2014年12月圆满通过专家回访,学院的人才培养质量和办学特色进一步提升。
(3)形成常态化的专业调研和岗位分析机制
由学院专业建设指导委员会总体部署,各系专业建设委员会制订调研制度、明确职责、强化考核,各专业建设团队确保每个专业每年进行一次专业调研与职业岗位分析。通过现场走访、召开座谈会、问卷调查、电话网络调研等方式,对政府主管部门及相关机构、行业企业负责人、技术骨干、毕业生等,对行业、企业发展现状及趋势,职业岗位、工作任务、工作内容,人才需求及知识、能力、素质要求等内容进行调研,形成专业调研报告。根据专业调研结果,对现有招生专业进行研判,对连续三年招生形势不好的,逐步调整招生计划,直至停招退出招生。
(4)调整优化专业结构与布局
对接泰安经济社会发展和市场需求,对所有招生专业进行科学论证,提出专业增、留、改、并、撤意见,确定专业调整方案。学院服务区域经济社会发展,对接泰安重点产业,到2015年,学院现设有中央财政重点支持专业、省特色专业9个,2015年学院招生专业29个,涵盖了农林牧渔、土建、制造、电子信息、财经、艺术设计、旅游、轻纺食品、交通运输等9个大类,淘汰专业10个,调整专业2个,新增专业5个。
2013年与澳大利亚启思蒙学院联合举办的建筑工程技术和机电一体化技术两个专业;2015年与山东科技大学开展“3+2”专本对口贯通培养机械设计及其自动化专业;2016年首饰设计与工艺专业开展山东省现代学徒制专业试点。与宁阳、东平、新泰、岱岳等职业中专合作申报举办8个“3+2”专业,多种形式办学,促进专业更快发展。
(5)建立了专业评估机制
健全了常态化专业评估制度。2014年,在调研分析的基础上,参考省级品牌专业评估标准,完善专业评估的指标体系,由各专业建设委员会充分讨论,制定了《专业评估实施方案》。2015年,由学院专业建设指导委员会组织,通过查阅资料、数据分析、访谈、听课、实地查看等形式,对技能型特色名校重点专业实施了评估。每个专业每年都形成了年度质量报告,有效监测了专业建设质量,提高了人才培养水平。
每年接受山东省职业教育评估所的专业质量评估,按照评估指标体系,对各专业建设情况进行梳理总结,提交相关数据及佐证材料,由省职业教育评估所进行全省性专业评估。在2014年省高职专业评估中,农林牧渔、轻纺食品、艺术设计、财经、电子信息、制造等专业大类,分列第5、6、10、11、14、21位。2015年,在山东省专业质量评估中,各专业排名稳中有升。
(6)系统优化人才培养方案 
各重点专业在专业调研的基础上,根据区域产业发展方向,校企深度研讨,聚焦专业服务面向。通过召开专业研讨会和岗位分析会,按照市场调研→职业和岗位分析→人才培养目标和规格→构建课程体系→制定人才培养方案→运行评估→市场调研……流程,对2013级、2014级、2015级专业人才培养方案进行了修订。引进企业技术标准和职业资格标准,校企共同设计了基于工作过程、平台+模块”的课程体系,“分阶段、项目化、协同式”实践教学模式和“双证书”制度,优化培养过程,实现了专业教学要求与职业岗位任职要求的逐步对接。
(7)以项目为引领,深入推进有效教学改革
突出“学生主体,能力本位”,进一步加强内涵建设,提高教师的职教能力,实施了以项目为引领的课程与有效课堂教学改革,不断推进课堂有效教学。
2014年4月份,学院制定下发了《关于开展教师职教能力培训与测评活动的通知》《关于实施有效课堂教学改革的实施意见》和有效课堂评价表,修订了《主要教学环节质量标准》,实施“有效课堂”建设,对教师课堂教学评价的重点,由教师教的成效转变为学生的学习成效。所有专业都进行体现“理实一体”理念的“教学做一体化”教学模式改革,制定并实施符合本专业特点的教学做一体化教学模式改革方案,灵活采用项目导向、任务驱动等教学模式。推进体现“学思结合、知行统一”的启发式教学、案例教学、讨论式教学等多样化的教学方法改革,注重将道德修养、职业素养融入教学过程。
2013年以来,学院建设省级精品课程7门,市级精品课程10门,院级精品课程59门,建设专业核心课程53门,校企合作开发课程40门,自编校本教材(讲义)25种,公开出版教材38种,评选学院正式出版项目化课改教材28种,2015年获省优秀教材一等奖1种,三等奖3种。
教师积极参加省级以上各类职业院校教学比赛,青年教师、信息化教学设计、信息化课堂教学、信息化实训教学、多媒体课件制作和微课等比赛,三年来获省级以上教师教学奖项总计24项,其中国家级二等奖1项,三等奖3项,省级一等奖3项,二等奖11项,三等奖6项。
学院不断探索研究教育教学改革,完善了《教学改革研究项目管理办法》,使每名专任教师每年都至少参与一项教学改革研究项目。设立专项资金,对人才培养模式、课程改革、教学模式、教学方法、考试方式等改革项目给予重点支持。2014年荣获国家教学成果奖二等奖1项,省级教学成果奖一等奖1项、三等奖3项。在2015年度全省职业教育教学改革项目立项中,我院共有7个项目获准立项,共获自助经费19万元,其中重点项目2项,一般项目5项。
(8)实行信息化管理,强化顶岗实习
完善了《学生顶岗实习管理办法》,明确了学院、实习单位、学生三方的权利和义务,学院指导教师和企业指导教师职责,规范学生顶岗实习的管理与考核,将考核成绩作为学生毕业成绩的重要组成部分,提高学生顶岗实习的积极性,确保学生不少于半年的顶岗实习。
和中唐方德共同开发应用顶岗实习信息管理平台,加强信息化管理水平。2016届毕业生全面启用了实习实训管理系统,对实习计划、实习评价标准、学生实习日志、四方(企业、学校、教师、学生)联系沟通等进行信息化管理,实现对顶岗实习的全过程、即时性监督与管理。
(9)人才培养质量全面提高
学院以技能大赛为抓手,“以赛促练、以赛促改”,深化教学内容、教学方法和考核评价制度改革,提升学生专业技能。建设了卓越技师培养基地,每个专业都组建了专业核心技能训练队,由企业技术能手和学校专业带头人、骨干教师重点训练,培养预备技师。实施“技能竞赛月”活动,每年开展一次面向全体学生的技能竞赛。师生们积极参加各级技能大赛,各专业屡创佳绩。三年来,师生在省级以上技能大赛中,获奖297项,其中,全国一等奖9项、二等奖19项、三等奖45项,省特等奖、一等奖48项,二等奖71项、三等奖105项。
三年来毕业生就业率和企业满意度逐年提高,2015年就业率达99.9%,在134所高校中排名第2,获省高校就业工作先进集体。麦可思调查显示,学院培养毕业生就业现状满意度和职业期待吻合度明显高于全国高职院校。
4.反馈与调控体系
学院通过每个学期的教学检查、督导、各类评估和来自4个监控主体的反馈意见,形成每个学期的学院教学监控评价分析报告,各系部每个学期形成各系部的教学常规整改报告。
5.完善人才培养工作状态数据采集与管理平台
学院形成人才培养工作状态数据采集与管理常态化工作机制,健全组织机构,实施动态采集数据,即时分析。自2015年始全面启动网络版数据平台采集,逐步实现数据的动态管理。
每年制定详细的实施方案,落实数据采集工作责任制,根据数据采集平台中数据的特征分别归口到相关部门负责。数据采集完成后,学院数据采集与管理办公室对“人才培养工作状态数据采集与管理平台”信息进行认真全面的分析思考,查找问题,探究成因,撰写状态数据分析报告,为学院决策提供数据支持。
6.教学管理信息化建设
完成了教务管理信息系统升级改造,使用新教务管理系统,完善了管理功能。与北京中唐方德科技有限公司合作构建了技术先进、功能全面的专业教学资源库平台,与超星尔雅、中国知网等企业合作,共享共建先进的信息化教学资源,“互联网+”背景下的数字化校园建设成效初显。建设了9个专业教学资源库,带动专业群启动建设专业教学资源库18个,建设了48门专业核心课程并用于教学实践,全面提高我院信息化管理与应用水平。学院被省教育厅确立为山东省首批教学信息化试点单位。
(三)特色与创新
1.形成监控保证机制
学院完善了内部两级四层次的教学质量保障组织机构,形成了院系(部)二级管理机构,学院、系(部)、教研室、学生4个层面的组织与管理机构。完善了学生信息员教学信息反馈制度。修订编辑《教科研制度文件汇编》,进一步明确、规范了常规教学要求、标准和程序。制定了《院系两级管理办法(试行)》。完善了激励机制,继续实施师生员工奖励办法,技能大赛、教科研等奖励180万元。
2.健全了质量标准体系
修订论证的专业人才培养方案。根据专业调研和岗位需求分析,对2013级、2014级、2015级学生的人才培养方案进行了修订,使教学要求与职业岗位任职要求对接,制定出符合人才成长规律和教育教学规律,突出实践技能、素质培养和服务区域经济社会发展特点的人才培养方案。修订了410门课程标准,29个专业标准,修订编辑《教科研制度文件汇编》,进一步明确、规范了常规教学要求、标准和程序。
3.形成了年度质量报告公布制度
实行了人才培养质量报告和专业人才培养状况年度报告制度。按照教育部、教育厅要求,每年都认真撰写人才培养质量报告和专业人才培养状况年度报告,通过网站等形式向社会发布,完善了人才培养监测体系,实现了学校自治、政府监控与社会(行业企业、公众)监督的有效制度。
4.形成了人才培养工作状态数据网络和常态工作机制。
完善建立了人才培养工作状态数据采集与管理制度,形成了常态工作机制,确保源头采集、动态管理。在对数据科学分析的基础上形成每年度的状态数据分析报告,发现学院在办学方面存在的问题,更有针对性进行整改建设,促进了学院人才培养质量的持续提高。
5.开展了有效课堂和项目化课程教学改革。通过实施顶层设计、整体推进、分步测评、全面实施等系列措施,对全院教师进行了三期以课程整体教学设计、单元教学设计为主要内容的培训与测评工作,累计有296名教师测评合格,编制《整体教改课程教学设计案例集》两册。坚持“以学生的学习成效”为评价标准,突出“能力本位、素质渗透、项目载体、成果检验”,以学生到课率、参与率、达标率等六项内容为具体指标,以随机听课为评价方式,推进课堂教学改革。
6. 提高教学管理信息化水平。
完成了教务管理信息系统升级改造,使用新教务管理系统,完善了管理功能。与北京中唐方德科技有限公司合作构建了技术先进、功能全面的专业教学资源库平台,与超星尔雅、中国知网等企业合作,共享共建先进的信息化教学资源。建设了9个专业教学资源库,带动专业群启动建设专业教学资源库18个,建设了48门专业核心课程并用于教学实践。
和中唐方德共同开发应用顶岗实习信息管理平台, 2016届毕业生全面启用了实习实训管理系统,对实习计划、实习评价标准、学生实习日志、四方(企业、学校、教师、学生)联系沟通等进行信息化管理,实现对顶岗实习的全过程、即时性监督与管理。
教师参加各级信息化教学比赛,三年来获省级以上教师教学奖项总计24项,其中国家级二等奖1项,三等奖3项,省级一等奖3项,二等奖11项,三等奖6项。截至2016年3月,学院建有省级精品课程17门,市级精品课程10门,院级精品课程95门,校企合作开发课程41门。学院被省教育厅确立为山东省首批教学信息化试点单位。
项目十二    泰山特色校园文化建设
(一)建设任务完成情况
项目立项建设以来,特色团队按照建设方案和任务书完成5个二级项目,25个子项目和51个验收要点。计划投入资金210万元,实际投入283.15万元,主要建设任务指标完成情况见表2-51。
表2-49  “泰山特色”校园文化建设项目目标完成情况一览表
建设内容
关键指标
单位
预期目标
实际完成
完成度
泰山特色
校园文化建设
泰山书院
个
8
8
100%

环境文化
个
3
3
100%

制度文化
个
3
3
100%

行为文化
个
5
5
100%

精神文化
个
6
6
100%
(二)项目建设成效
1.泰山书院建设
建成书院,丰富馆藏;注册、出版《泰山书院》;成为省、市社科普及教育基地;创作完成泰山百景诗书画印作品;开设讲书堂,举办文化教育讲座;开展校内外文化交流。取得如下亮点性、标志性成果如下:
(1)完善基础建设
名校建设期间,多次召开泰山书院筹建专题会议,拟定建设方案,逐步完成了博物馆、墨宝斋、文渊阁等书院场馆建设。
(2)多途径丰富馆藏
学院通过购买、接受捐赠、编印泰山文化读本、交流创作等方式充实了书院馆藏。
(3)注册出版刊物
成功申请注册并出版刊物《泰山书院》,截至目前刊物已印发至第5期。
(4)设立讲书堂
邀请泰山文化专家、齐鲁文化专家、国内教育学者、行业大师、优秀毕业生等开展有关泰山文化、齐鲁文化、科普教育、创新创业、健康文明等主题讲座40余次,如“中国工艺美术大师”李博生先生,海洋研究领域著名专家赵炳来教授,现任教育部职业技术教育中心研究所研究员的资深职教专家姜大源教授等等。
(5)开展国内外文化交流,取得系列显著交流成果
先后迎接中国水墨艺术研究院、全国政协文史与学习委员会、全国职业教育校长联谊会、省内外高校、省市文联、省市社科联、合作企业等单位和台湾、马来西亚、韩国、美国、澳大利亚、新西兰等境外院校参观约百次。接待了著名海外华人作家严歌苓、玉雕大师李博生、著名画家杨宪金等文化人士。2013年12月接受中央四台“远方的家”栏目组采访并被报道。2014年9月,承办了山东省泰山文化论坛,与泰安市委宣传部、泰安市社科联携手编写了《文化泰山——山东社科论坛.泰山文化研讨会论文集》。此外,书院多次组织举办书画展,推荐优秀作品参加各级比赛并多次获奖。
(6)完成泰山百景诗书画印作品创作
以本院教师为主要创作人员,邀请当地知名书画家参与,组成创作团队。在泰山文化传承上,首次创造性地对泰山一百个著名景点以诗、书、画、印的形式进行了系统创作,受到了泰安市社科联和书画界的高度评价。并且,加强了成果转化力度,将本系列创作成果作为文化元素,应用于文化旅游产品设计当中,实效显著:2015年,李军祥的篆刻作品“对松叠翠”与李莉老师的参赛服饰设计等相结合,荣获15年中国泰安(泰山)平安文化旅游商品大赛金奖;泰山百景篆刻在茶具设计中也得到了开发应用。
(7)获批省市级社科普及基地
2014年,泰山书院获批为山东省社科普及教育基地和泰安市社科普及教育基地,承办2014年山东省文化泰山社科论坛,并被评选为省优秀论坛。
(8)探索书院加学院的职业教育模式,积极开展文化研究
书院申报的《泰山旅游文化策划—以泰山百景诗书画印系统创作为载体的泰山旅游品牌的推广研究》课题,被泰安市社科联立项为2015年社科类重大课题;另一课题《“书院+学院”高职教育模式研究探析》获批一般课题。学院“人文素质教育中心”在书院挂牌,为更好地发挥人文教育作用奠定了基础。 
2.环境文化建设
持续对校园环境进行规划、净化、绿化与美化,增加了“泰山特色”“职业特色”景观设施,完善学院形象标识系统,自然与人文结合,浓厚了环境育人氛围。取得如下亮点性标志性成果:
(1)加强宣传阵地建设
对校报、校园网、校内广播、宣传栏等宣传阵地加强了管理、补充、更新,在教学、实训区增设企业文化及7S管理宣传阵地,积极推进7S管理制度。
(2)增设、维护文化主题景点
名校建设期间,学院建设完成了泰山特色校企合作园、国际交流园、创业广场等主题园区,对休读亭、文化长廊、望岳园、生态园、青春广场等主题景点和场地持续净化美化绿化。
(3)增设、修护历史人物等雕塑及泰山石刻
为充分体现泰山精神、大学文化、传统文化特和企业文化,增设了泰山书院历史人物、日晷等四个雕塑,新增石刻13处,对原有70余处石刻进行了修护、翻新。
(4)完善学院文化形象识别系统
重新命名学院主要场馆,设计美化办公室门牌、文件袋等牌匾、办公用品等,整体提升学院形象标识系统。
(5)净化绿化美化校园
加强学院环境卫生管理,不断增植绿化面积,提高环境规划科学性,2013年学院被表彰为“山东省高校校园绿化管理工作先进单位”,学院为山东省花园式单位。
3.制度文化建设
制定7S管理、宣传管理、校企合作、师生评价等系列制度,推动学院健全完善规章制度体系,形成学院制度管理汇编,编印了《泰山职业学院制度汇编》《教学制度汇编》《学生管理手册》等。
4.行为文化建设
开展大量丰富多彩校内外实践活动,通过实践使文化育人落地,促进了校风、教风、学风的全面优化。取得如下标志性成果:
(1)书香校园建设
每年制定书香校园建设活动方案,征订、发放了中国传统文化读本,按照方案开展读书月活动。丰富藏书,创新图书馆服务,开设歌德电子借阅机服务,为师生免费提供数字信息化文化;开展好书推荐活动、读书笔记展评、“书韵凝香”主题读书活动、“早说晚写”活动、“书香四季智慧女性”活动、“书签制作”、书画作品大赛等系列活动,编辑了《读书留香》案例、《大美泰山》征文、《书香四季智慧女性》征文等集锦,浓厚了书香氛围,取得了系列成果,如:侯加阳老师在山东省“传承文明 共筑梦想”经典诵读活动中荣获三等奖;推荐书画作品参加全国、省、市大赛,李元生、李军祥等多人获奖。
(2)届次化大型活动
制定了届次化活动年度方案,按照计划指导各系部开展活动近百次。活动中,依托雷锋月、植树节、五四青年节、国庆节、抗战胜利纪念日、国家宪法日、清明节、端午节、重阳节、全国助残日、世界环境日、世界读书日等主题节日,适时开展丰富多彩的活动,不断增强学生法治、卫生、环境保护等方面意识,增强师生历史责任感与社会服务意识。
注重形式创新,强化活动育人效果,如两季常规运动会外开展各类趣味运动会。固化成效较好的大型活动,打造学院特色活动品牌,如金秋宿舍文化艺术节、大学生心理健康节等。2014年,作品《朋辈的力量》荣获山东省大学生心理健康节DV大赛三等奖;侯玲老师的《高职大学生心理健康教育个案分析》获山东省高校心理健康优秀论文三等奖。15年学院大学生心理健康节获得山东省高校大学生心理健康节优秀组织奖。
(3)社团活动
规范了社团管理,成立了如望岳文学社、摄影、书法、绘画、武术、科技创新、专业服务等院级、系级文体类、专业类主要社团50个以上。丰富多彩的学生社团自发组织了形式多样、健康有益的学生活动,张扬了学生个性,提升了学生的综合素质。参加各类文体赛事,多次荣获国家级、省级、市级荣誉。 
(4)志愿服务活动
建立志愿服务者名单,加入泰山文明志愿者服务协会,设置20多个泰山文明服务岗,定期组织文明服务活动。三年来,学院荣获“泰安市青年志愿服务贡献单位”、“泰安市无偿献血突出贡献奖单位”光荣称号;学院青年志愿服务队荣获“泰安市青年志愿服务优秀集体”,学院化马湾中心小学心理辅导项目被列入2014年度“泰安市青年志愿服务项目”,多人荣获“泰安市青年志愿服务先进个人”;学院被中共泰安市委宣传部等四部门表彰为文化科技卫生“三下乡”志愿者服务先进集体,有2人荣获先进个人。
(5)系部活动
制定了系部文化建设指导意见。各系部完善了文化活动提升计划或方案,开展了大量富有专业特色的文化活动,传承了传统文化和泰山精神,促进了企业文化与大学文化的融合,促进了学生专业能力与学习兴趣的提高。
特别是通过深化校企合作、筹建创业大学、举办职业规划大赛、举办或参加各级各类技能比赛、邀请企业专家作报告等活动,各系部不断提升学生职业素养,在国家级、省级、市级各类专业赛事中频频获奖。
5.精神文化建设
凝炼“进取、担当、包容、和谐”的泰山精神。开展精神文明建设、社会主义核心价值观教育,浓厚氛围,选树典型,发挥思想文化引领作用,取得如下标志性、亮点性成果:
(1)加强社会主义核心价值观教育和省文明单位创建
制定了学院培育和践行社会主义核心价值观实施意见。与学院党风廉政建设、师德师风建设、大学生思想政治教育相结合,发挥党团组织的引领作用,将社会主义核心价值体系建设融入教育全过程。
制定泰山特色校园文化建设规划、年度计划,把校园文化建设作为精神文明建设的重要内容纳入省级文明单位年度建设计划,结合实际灵活规定活动主题,广泛发动,持续狠抓落实,
名校建设期间,学院先后获得全国就业竞争力示范校、山东省高校思想政治工作先进集体、山东省2015年度最具特色高职院校、山东省2015 年度最受网民欢迎高职院校等荣誉,连续五年保持省级文明单位光荣称号。8个部门分别获得山东省工人先锋号、泰安市工人先锋号等称号。
(2)推进学院理念系统建设
制定了学院理念系统设计草案、完善了学院章程。确立了“进取、担当、包容、和谐”的泰山特色学院精神, 修订、完善了校训、校徽、校旗、校歌的设计和创作,优化了学风、教风和校风,使学院精神成为激励、指导全院师生员工攀登进取的重要精神旗帜。
(3)评优树先,发挥先模引领作用
利用教师节、五四青年节等时间节点,开展先模评选和典型事迹宣传;开展“最美泰职人”、“三八红旗手”、“功勋教师”评选等,选树楷模。挖掘先进事迹,编辑了《奋斗之歌》《修德笃行创新奉献》《修德 笃行 进取 担当》等系列先进事迹材料。推荐泰安好人30名,大力弘扬社会主义核心价值观和进取、担当的泰山精神。
(4)举办泰山特色校园文化建设成果展,获得社会各界肯定
制定泰山特色校园文化建设年度方案,编辑出版史志性材料《梦想的华章》,出版院报25期,组织了60年校庆学院教育成果展等一系列文化活动。积极通过《中国教育报》、《中国青年报》、省教育厅网站、市电视台等校内外各级各类媒体宣传学院特色校园文化建设成果。组织的“向林鸾老师学习”的师德教育活动获山东省高校文化建设活动评比三等奖;学院被评为泰安市校园文化建设特色学校、泰安市校园文化建设优秀单位。 
(5)泰山精神凝聚正能量,媒体宣传力度再创新高
大力弘扬“进取、担当、包容、和谐”的泰山精神,加强了在中国教育报、中国青年报、大众日报、教育厅网站、高职高专网、中国文明网、大众网、泰安文明网、泰安电视台、泰安日报、学院网站、院报、院刊校内外各级各类媒体的宣传力度,仅校外各级各类主流媒体宣传就近800次。其中,在中国教育报发表文章2篇、中国青年报发表1篇,大众日报2篇;学院建设案例入选了《中国教育大全》《山东省年鉴大全》。
(6)完成2项提升项目
省级课题“地方优秀文化在高职校园文化中的应用研究——以泰山职业技术学院泰山特色校园文化建设为例”和“泰山特色传统文化在高职教育中的继承和发展”顺利结题,为校园文化建设的可持续发展与下一步更好地发挥文化的创新和引领作用奠定了基础。
三年来,泰山特色校园文化建设开展扎实,成效显著。该项目资金预算为210万元,实际投入和支出280余万元。建成泰山书院,如期开放场馆,通过文化活动、出版刊物、开设讲书堂等形式,充分发挥了优秀地方特色传统文化在新时期的育人功能。通过环境文化建设,浓厚了泰山特色四位一体校园文化氛围,彰显泰山精神,凸显职业特色。通过制度文化建设,推进了学院各方面规章制度的完善,为泰山特色四位一体校园文化建设提供了保障,促进了学院各项工作的有序协调开展。通过行为文化建设,开展丰富多彩的实践活动,全面加强了校风、教风、学风建设,实现了文化育人的落地。通过精神文化建设,将“进取、担当、包容、和谐”泰山精神凝华为学院精神,在学院发展中起到了文化引领作用。
(三)特色与创新
1. 本项目创新高职院校与地方文化对接的校园文化建设思路,对同类院校起到了示范带头作用。用泰山书院、泰山松、泰山石刻等泰山元素打造学院文化环境,凝练“进取、担当、包容、和谐”的泰山精神为学院精神,泰山特色四位一体校园文化独具一格,既彰显泰山精神,又凸显职业特色。
2.重建千年泰山书院,探索职业院校与书院教育相结合的文化育人新模式,在职业院校开创“学院+书院”模式先河。2014年,泰山书院获批山东省社科普及教育基地和泰安市社科普及教育基地,成为文化传承与创新的辐射源,省内外、国内外许多高校与团体组织前来参观和借鉴。
3.泰山百景诗书画印系列作品创作,是泰安文化史上首次以诗书画印形式系统展现泰安百景的作品,具有形式上的创新性。

项目十三    数字化校园建设
(一)建设任务完成情况
项目立项以来,通过三年的努力,相关全部工作已相继完成。专业团队按照建设方案和任务书完成5个二级项目,8个子项目和22个验收要点。主要建设任务指标完成情况见表2-50。 
表2-50  数字化校园建设项目目标完成情况一览表
项目名称
建设内容
预期目标
实际完成
完成度
校园网软硬件平台建设
无纸化标准考场(个)
1
3
超额

建成万兆核心骨干校园网
1
1
超额

重点区域WLAN覆盖
1
1
超额

服务器、存储虚拟化、备份(套)
1
1
100%

建成统一身份认证系统
1
1
100%
数字化教学服务平台

建设全自动多媒体课堂录播室(间)
1
1
100%

建设网络教学系统(套)
1
1
100%

建设教学资源库(门)
0
18
超额

顶岗实习系统(套)
0
1
超额
校园一卡通平台拓展
系统平台(套)
1
1
100%

校园卡业务中心(个)
0
1
超额

电子消费系统(套)
1
1
100%

身份认证系统(套)
1
1
100%

圈存系统(套)
1
2
200%
信息化管理平台建设
建设数字化校园平台(套)
1
1
100%

教务管理系统
1
1
100%

办公OA系统(套)
1
1
100%

招生系统(套)
0
1
超额

站群系统(套)
1
1
100%

迎新系统(套)
0
1
超额
文献信息服务平台建设服务平台
知网电子期刊招标续费(年)
3
3
100%

汇文图书管理系统(套)
0
1
超额

电子资源库试用(个)
0
5
超额

电子书借阅机(台)
0
1
超额
	•	项目建设成效
通过数字化校园建设,基本建成完整统一、技术先进、覆盖全面、应用深入、高效稳定、安全可靠的数字化校园软硬件系统,以有线及无线网络为基础环境,实现网络增值产品、教学应用产品、信息管理产品有效整合集成。
1.构建高速校园网络,保障系统平稳运行
通过三年的建设,实现了教学区、实训区、办公区和学生宿舍区的网络全覆盖,“核心一汇聚一接入”三层架构使网络拓扑更加稳健,核心层配置了高性能万兆路由交换机,汇聚层交换设备配置了万兆接口模块,接入层采用全千兆智能交换机,实现了“万兆核心,万兆主干,千兆到机房,百兆到桌面”的高速网络架构,办公区域实现全光网络接入,安装最先进的802.11n无线局域网,为师生提供高速、无缝的网络覆盖,满足了用户使用平板、智能手机等移动终端联接互联网的需求。
通过配置四路八核256G内存高性能服务器、40TB高端网络存储设备和成熟的虚拟化产品,完成了核心交换机、60KVAUPS等设备安装,构建了高效的校内云计算数据中心,其中服务器、存储虚拟化满足了数字化教学服务平台、信息管理平台对运算和存储空间的需求;新建的校园网络中心机房,在布局规划、走线设计、简洁实用、易于拓展方面达到全省一流水准,为信息化工作提供了强有力的保障。
2.搭建在线教学环境,助推网络学习应用
搭建中唐方德数字化教学服务平台。该平台是由中唐方德公司开发的数字化学习系统,提供了大量的国内专家学者视频授课教程等教学参考资料。教师通过该平台能够建设网络课程,完成作业、资料、测验、答疑、讨论等互动教学活动,完成基本的教学统计。为推进以MOOC为代表的网络教学模式和教学方法改革,提高教学质量提供了很好的支撑。自系统投入使用以来,已经有两百余名教师利用数字化教学服务平台进行备课、网络辅助教学、网上答疑、在线考试等,全院学生都可以在网上进行辅助学习、在线答疑、交流、在线作业及测试等;涉及的课程有基础课程、专业课程、选修课程以及学生到企业顶岗后的学习等。
建成高标准全自动多媒体课堂录播室。采用云课堂、电动翻转屏、触摸屏一体机等最新主流技术,教学区、讨论区、实训区一体化设计,实现教学资源库教学视频、微课视频、课程群教学视频的三分屏全网全自动导播、直播、录播、点播。在功能设计、缺陷规避上达到了全省一流水准。
新建1个云桌面电子阅览室,2个无纸化标准考场,共400台电脑,具备电子阅览、上机实训、无纸化考试等一室多用功能,可以满足学生进行专业实训、模拟仿真实训等需要,也可以满足学生进行网络学习的需要,同时利用业余时间承担各种社会无纸化考试4万余人次创收50余万元,为学院进行翻转课堂教学改革提供了基础保障,提升了社会服务能力。
3.建成一卡通平台,一卡在手走校园
以学校校园网为载体进行校园一卡通系统建设,建成集身份识别、校内消费、校务管理、图书借阅、金融服务为一体的新型数字化校园核心应用项目,实现校园卡数据采集、制作、发放、充值、人工挂失功能,提供卡系统管理、设备管理、数据管理、对账等一体化服务。采用最新银校卡技术,实现银行卡与校园卡绑定,提供资金自助圈存、空中圈存、卡挂失、密码更改、数据查询、补助发放等服务。采用自助服务平台和一卡通APP客户端,涵盖餐饮、洗浴、开水、超市、图书借还等系统应用, 实现电子消费管理及灵活的数据统计功能,支持有条件的断电消费、脱机消费及限额密码消费,保证交易安全。系统全面实现了“一卡在手,走遍校园,一卡通用,一卡多用”。目前一卡通系统在最新技术应用上走在了全省前列。
据统计,校园“一卡通”系统共安装10台圈存触摸一体机,126台售饭机,50台开水控制器, 5台商户POS机,现有正式卡用户9000余人、临时卡用户700余人,用户月均圈存1.8万余次,日均消费2万余笔。补助实现集中发放、自助领取的模式,既保证了账务清晰,又方便用户自助管理。一卡通网站的综合查询功能,实现交易流水对商户、个人透明化管理,保证了每一笔交易都有踪可循。
4.实现系统整合集成,助力现代教育治理 
在原有教务管理系统、图书管理系统等业务系统基础上,定制符合学院管理要求的OA办公自动化软件系统、站群系统、单独招生系统、迎新系统各一套,通过统一数据标准、对接接口等手段,实现了全部数据的互联互通与共享,通过统一信息门户、统一身份认证、数据清洗与整合,集成了各类应用系统,师生通过单点登录,访问各应用管理系统。
通过管理系统的高度集成,师生在享有信息化带来的便捷的同时,也提升了各部门的工作效率,到目前为止,通过各管理系统建立的流程有53个,其中包括公文流转、经济事项审批、公务出差审批、教职工请假审批等,每天流转的流程有100多个。
5.广泛应用电子化资源,紧跟信息化潮流
以系统化、数字化的学术信息资源为基础,以先进的数字图书馆技术为手段,量力而行,分步实施,积极试用,整合资源,共享数据,不断充实和丰富本地镜像资源和远程资源,建设文献信息服务平台, 提供高效率、全方位的文献信息保障与服务,开通校外VPN,任何读者可以在任何时间,任何地点获取图书馆的任何电子资源。 
购买超星电子图书10.5万册,中国知网资源数据量7T,建立了本地镜像站点。经积极对接,先后在学院内网开通了重庆维普、读秀学术搜索等13个数字资源库的试用,仅中国知网检索次数就达到了34万余次,下载1.8万余篇。基本满足文献检索、在线浏览、正文下载需求。
(三)特色与创新
1.搭建校园云平台,提升信息化层次
整合系统各类资源,组建虚拟计算资源池,搭建网络教学云平台、教育治理云平台,为信息化进入云计算时代做好铺垫。网络教学云平台提供丰富的教育教学资源,开通互动课堂、在线测试、在线论坛、虚拟实训室等服务,采用全新的电子教学方式,以Web技术为基础,辅以诸如Flash、音频、视频等,学生可以在教室、宿舍、家里登录访问,根据自己的进度学习课程、观看视频或其它可视化技术讲解,使用在线考试系统来检查理解掌握知识的情况,实现个性化的学习。教学云平台逐步构建覆盖全校工作流程、校企协同的管理信息系统,通过管理信息的同步与共享,畅通政、校、企的信息流,实现管理的科学化、自动化、精细化。通过云平台建设构建数字化的教学环境、数字化的管理手段和工作环境,实现传统教材到多媒体教学资源的转变,传统管理到信息化管理的转变。
2.建设数字化场室,助推信息化应用
数字化场室并不是简单等同于多媒体教室,一体化实训室,它是以网络为基础,从环境(包括校园、场室等)、资源(如图书、教案、课件、音视频等)、到活动(包括教学、管理、服务等)的全部数字化而形成的一个数字空间,方便地实现教学做一体化实训、多媒体教学、电子教室、自主学习及管理、分组讨论、实践教学管理功能,从而提高信息化应用水平。
学院新建多媒体课堂录播室,采用云课堂、电动翻转屏、触摸屏一体机等最新主流技术,教学区、讨论区、实训区一体化设计,实现教学视频的全自动导播、直播、录播、点播。
学院新建云桌面电子阅览室,应用信息共享空间的理念和服务模式重新整和空间、资源和服务,在一个相对集中的空间内多室合一,用户提供综合的资源和服务,具备电子阅览、上机实训、无纸化考试等一室多用功能,可以满足学生进行专业实训、模拟仿真实训等需要,也可以满足学生进行网络学习的需要,为翻转课堂教学改革提供了基础保障。
3.打造移动应用,随时随地访问
顺应 “云、物、移、大、智” 信息技术发展方向,在各应用系统及整合平台的选型上,始终把是否支持移动应用作为重要的技术标准。一卡通系统支持手机圈存、挂失等业务;办公OA系统支持手机办理审批流程、即时通讯等业务;数字化教学服务平台支持手机移动学习、在线交流等业务;系统整合平台有功能强大的手机版应用,全面实现随时、随地、方便、快捷的移动应用接入访问。



第三部分   项目建设质量与效益

经过三年名校建设,学院严格按照建设方案和任务书,圆满完成了各项建设任务,实现了建成省技能型特色名校的目标。学院基本办学条件进一步改善,办学特色更加鲜明,内涵建设不断深化,管理水平、人才培养质量明显提高,社会服务能力和示范带动作用显著增强,办学综合实力显著提升,得到了社会广泛认可,形成了高职教育中的泰山品牌,取得了显著的效益。
表3-1  部分新增标志性成果一览表
成果名称
时间
描述
《高职“分阶段、项目化、协同式”实践教学模式研究与实践》获国家教学成果二等奖
2014
第二批省名校唯一获此等级院校,全省最高奖之一
省级教学成果一等奖1、三等奖3项
2014
第二批省名校仅3所学校获一等奖。
立项省教学改革项目7项,其中重点项目2项,争取资金19万元
2015
立项项目数、重点项目数在所有省名校中均列第二位
省教学名师1人
2015
第二批省名校10所各1人
省教学团队1个
2015
第二批省名校均有
省精品课程7门
2013
第二批10所有新增精品课的省名校中,数量列第六位
省特色专业1个
2013
第二批省名校共10所院校有新增
省级以上技能大赛
2015
2015年园林景观设计赛项获省一等奖,进入国赛
省优秀教材一等奖1种,三等奖3种。
2015
第二批省名校中仅2所院校获一等奖
全国高职院校教学诊断与改进试点院校
2016
省名校唯一入选院校。全省仅3校:1所国家示范校、1所国家骨干校、1所省名校
省人社厅高校就业率在举办专科层次的134所院校中排名第二
2015
全省高职院校排名第一位
全国职业院校就业竞争力示范校
2013
省名校唯一入选院校。全省共3所:2所国家示范校,1所省名校。
山东省首批教育信息化试点学校
2014
全省25所高职院校试点之一,省名校共15所。
省现代学徒制试点院校
2016
第二批省名校共8所
与本科“3+2”对口贯通培养
2015
第二批省名校共11所

2016
第二批省名校共  所
学前教育专业批准招生
2016
第二批省名校共2所,全省共9所院校
定向培养士官
2016
省名校唯一。全省共5所:1所民办本科,1所国家示范校、2所国家骨干校,1所省名校
接收新疆“普通高校未就业毕业生培养”三批79人
2013
省名校共2所院校承担此任务。全省共4所:1所国家示范校,1所国家骨干校,2所省名校
泰安创业大学
2015
职业院校办市级创业大学,省名校有3所
省厅确定结对帮扶中职学校(济宁汶上县职业中专)
2014
第二批省名校中仅3所院校承担此项任务
市委市政府科学发展综合考核先进单位
2015
驻泰高校唯一
泰山书院被列为省级社科普及教育基地
2014
全市唯一
学院25名专业技术人员被为泰安市科技特派员领导小组被聘为“泰安市第一批企业科技特派员”、 “1人被聘为农业科技特派员”。
2015
学院企业科技特派员占全市特派员总人数的61%

一、学院办学综合实力大幅提升
(一)学院办学特色更加鲜明
通过“4433”名校建设工程,学院以“服务”价值取向的办学理念、“多方联动、协同发展”的多元化办学模式、“德技并重、理实一体”的人才培养模式和融传统文化、泰山文化、企业文化、大学文化“四位一体”的校园文化已经形成鲜明的办学特色。
1.以“服务”价值取向办学理念得以彰显
学院历经六十年职业教育办学历程,经过名校建设的砥砺磨炼,全院教职工教育教学理念得到极大转变和提升。由管理角色向服务角色转变,由传授知识技能向服务成长成才转变,由封闭式独立办学向开放式合作办学转变;在服务教师专业成长中彰显学院价值,在服务学生成长成才中彰显教师价值,在服务社会发展和文化创新中体现学院对区域经济发展的贡献度,名校建设过程成为一场深刻教育理念的革命和实践创新。秉承“进取、担当、包容、和谐”的泰山精神已经深入到泰职人心中,融入到泰职人的血液,“服务是责任,服务是修养,服务创造价值,实践改变命运”的价值观成为泰职人的共识,成为引领学院专业建设和教学改革的指南,服务意识强、服务能力高、服务品质优的服务标准和要求成为规范师生员工行动的标尺。
在服务理念引领下,学院紧紧围绕人才培养这个核心,不断加大师资队伍建设力度,提高教师社会服务能力和水平。学院扎根泰安、面向齐鲁,全心全意服务学生成长成才,服务教师专业成长,服务社会发展、服务传承传统文化,奉献社会,勇于探索创新,整体办学水平和综合实力全面提升,服务价值取向成为学院独特的办学理念和价值追求。
服务理念落实到办学实践中,取得了较好的经济和社会效益。三年来,通过名校建设,教师理念得到大幅提升,学生个性得到张扬、能力得到提高,实现全面可持续发展;教师团队实现了专业化成长,一批专业带头人和骨干教师成为专家、名师和技术能手,教学、科研和社会服务能力全面提高;学院面向地方经济社会发展主战场提供种类社会服务,做出突出贡献。每年开展各类培训1052622人时,开展职业技能鉴定10309人,培训全市退役士兵、基层农技人员、中职教师等各类人员5000余人;开展技术研发、技术改造等服务项目306项;开展横向课题研究33项,取得国家级专9项,成人教育、远程教育培训学员1800余人,面向社区开展讲座、报告20余场。25名科技特派员充分发挥个人专业优势深入企业开展社会服务,10人被聘为市科技咨询协会理事、市社科专家,积极参加相关活动。各种社会服务收入2128万元。
2.“多方联动、协同发展”的多元化办学模式日臻成熟
学院积极探索合作办学体制,着力打造“人才共育、过程共管、成果共享、责任共担”的紧密型合作办学体制机制,“政行企所校”合作向纵深发展,建成了充满活力、富有效率、有利于科学发展的体制机制。坚持政府主导,行业指导,学校与企业主体原则,成立了校企合作理事会和校企合作委员会,制定了校企合作理事会章程;组建了机电职教集团和教育联盟,积极探索现代学徒制试点,创新了“政行企所校”多元合作办学体制,完善了多方参与、共建共评的管理运行机制,制定了人才共育、资源共享等系列校企合作制度,保障了校企合作高效运行,促进了产教深度融合,增强了办学活力。
3.“德技并重、理实一体”的人才培养模式更加完善
遵循教育规律和人才成长规律,创新实施“德技并重、理实一体”的人才培养模式,各重点建设结合专业不同特点,以“德技并重、理实一体”人才培养模式为核心,分别探索各具特色的人才培养模式。机电一体化专业探索出“行业主导,双业贯通,双证融合,项目推进”人才培养模式;建筑工程技术专业深化了“岗位引导,以岗定教”人才培养模式;电算会计专业创新实施“虚实结合,四段递进”的人才培养模式;旅游管理根据旅游市场变化,实施“淡入旺出,产学结合”人才培养模式;计算机应用技术专业“阶段培养、能力递进”人才培养模式;园艺技术专业实施了“两线四段三融合”人才培养模式;服装设计专业构建基于服装工作室的“工学交替、能力递进”人才培养模式,实现“学生能力培养层次化,师生身份职业化,基地建设企业化,实践教学生产化”的培养目标;汽车电子技术专业实施了“4+1+1”人才培养模式;珠宝首饰工艺与鉴定专业结合人才成长规律,创新“校企共育,三位一体”人才培养模式。
 “德技并重,理实一体”人才培养模式的创新探索与多样化实施,使学院人才培养更加符合专业不同特点,更加紧密对接企业用人标准,满足了社会需求,使学生在提高了理论知识和实践能力,学会做事的同时,也增强了社会能力,学会了做人,实现了可持续发展。
4.融传统文化、泰山文化、企业文化、大学文化“四位一体”的校园文化特色鲜明
以培养大学生职业精神和持续发展能力为目标,学院着力打造融传统文化、泰山文化、企业文化、大学文化于一体的四位一体校园文化品牌。
建成、充实了千年泰山书院,开设书院讲书堂,出版《泰山书院》专刊;接待国内外友人参观,举办文化教育讲座,成立人文素质教育中心,探索实施“学院+书院”职业教育新模式,使泰山书院成为人文素质教育基地,从事科学研究和社会服务的重要平台,文化传承创新的重要阵地和学院对外交流的窗口;突出“泰山特色”“职业特色”,持续对校园环境进行规划、净化、绿化与美化,实现自然与人文结合,产业文化和企业元素与校园结合,达到了校园文化建设物质环境的无形熏陶感染作用;在教室、实训室、学生宿舍、食堂以及办公场所全面推行“7s”管理,落实了企业文化进校园,培养了学生的职业习惯和职业精神,缩短了学生与社会的适应期,促进了学院内部管理向精细化、规范化管理转变;大力开展书香进校园活动、体育艺术节等届次化活动、社团主题活动、志愿服务活动及各具专业特色的系部活动,营造出浓厚书香氛围;发挥思想文化引领作用,开展精神文明建设、社会主义核心价值观教育,将校园文化建设与精神文明建设相结合,建立、完善学院文化形象识别系统,完善了校徽、校训、校歌和校旗,凝练学风、教风和校风,使学院精神成为激励、指导全院师生员工攀登进取的重要精神旗帜。
表3-2  办学特色标志性成果
序号
成果名称
1
与19个行业协会建立紧密联系,新增合作企业68家
2
我院牵头,联合省市内中、高职院校,大、中型企业40余家,共同组建泰安市机电职教集团
3
与泰安市高新区政府、东平县政府分别建立战略合作关系,成立东平移民学院
4
与山东科技大学、泰山学院联合开展高职本科“3+2”分段贯通培养。形成以专科层次高等职业教育为主,与中等职业教育、应用型本科教育相贯通的现代职业教育体系
5
与泰安市域内6所中职学校、平阴职教中心和汶上县职业中专建立了合作关系,与4所中职开展3+2中高职人才贯通培养
6
与泰安市泰山玉研发有限公司、北京李博生造型艺术有限公司等合作的首饰设计与工艺鉴定专业开展山东省现代学徒制试点,联合培养企业需求的人才
7
与澳大利亚启思蒙学院、台湾昆山科技大学等7家国(境)外大学或教育机构合作办学,招生人数达353人;12名教师参加了澳大利亚TAFE四级证书培训;84名教师国(境)进修;4批56名交流生赴台学习;引进原版教材26门
8
学院连续三年保持省级文明单位光荣称号,党委宣传处、财经系等荣获全省高校思想政治教育工作先进集体,8个部门先后荣获山东省工人先锋号、泰安市工人先锋号等
(二)专业建设成效明显
对接产业,建立了专业设置与动态调整机制,调整优化了专业结构与布局,重点建设了旅游管理、园艺技术、建筑工程技术、机电一体化技术等9个与泰安产业发展结合紧密的优势专业,带动全院各专业办学实力整体提升。成立了专业指导委员会和9个专业建设委员会;形成并完善了制度化、常态化的专业调研和岗位分析机制,通过调研和岗位分析,对行业企业人才需求状况进行统计、监测和职业岗位分析,进一步明确专业服务面向、人才培养目标、数量和规格,并形成专业建设调研报告,为学院专业设置和调整优化、专业人才培养方案修订、课程结构优化、教学内容调整、学生就业指导等提供依据,使学院专业人才培养紧密对接岗位标准和要求。完善了专业标准,健全了专业评估制度,修订完善了会计电算化、机电一体化等所有31个招生专业标准,并达到省内同类专业的先进水平。对所有重点专业实施了评估。
建设期间,招生专业数稳定在30个左右,累计建成4个省级特色专业,1个省级示范专业,2个国家财政支持的提升服务产业能力专业,9个省级特色名校重点建设专业。 实践教学条件和职业能力训练体系进一步完善,校内生产性实训基地专业教学、职业培训、技能鉴定和技术服务等功能全面提升。重点建设专业实验实训开出率达到100%,满足了教学要求。

表3-3  专业建设标志性成果
序号
成果名称
1
形成了具有地方特色的专业群布局。淘汰专业10个,调整专业2个,新增专业5个。
2
重视专业评估工作。制定泰山职业技术学院专业评估实施方案,对9个重点建设专业进行评估,每个专业每年都形成了年度质量报告。在2014年省高职专业评估中,农林牧、轻纺食品、艺术设计、财经、电子信息、机电等专业大类,分列第5、6、10、11、14、21位。每年专业排名稳中有升。
3
2015年、2016年先后与山东科技大学、泰山学院开展“3+2”对口贯通培养,2015年新增工业机器人技术、学前教育专业。首饰设计与工艺专业成为山东省现代学徒制试点专业。与宁阳、东平、新泰、岱岳等职业中专合作申报举办8个“3+2”专业。
4
2014年与澳大利亚启思蒙学院联合举办建筑工程技术和机电一体化技术等专业,招生289人。
5
建成4个省级特色专业,2个中央财政支持的提升服务产业能力专业,9个重点专业建设任务全部完成。
6
校企共建实训基地。新建2万平米的“教学做一体化”实训楼(行知楼)并投入使用,行业企业投入仪器设备1262万元,校企共建技术研发中心,建成泰安市第一公共实训中心,建有3个中央财政支持的校内实训基地。
7
建成了9个重点建设专业教学资源库,带动专业群启动建设专业教学资源库21个。开展了两期48门核心课程的数字化资源建设工作,前后举办了6期中唐方德教学资源库平台、实习实训管理系统等的使用培训会。
8
学院被省教育厅确立为山东省首批教学信息化试点单位。
(三)师资队伍整体水平显著提升
通过名校建设“名师递进培养工程”,采取境外培训、国培、省培项目以及学院各种培训学习,递进培养教学新星、教坛英才、教学名师、功勋教师;实施双带头人制度,9个重点建设专业培养1名校内专业带头人的同时,聘任1名行业专家任专业带头人,重点培养骨干教师162人,学院师资队伍水平得到整体提升。到2015年,师资队伍的双师素质比例、年龄、学历、学位、职称等结构更加优化,具有硕士以上学位的教师由54%提高到60.17%,高级职称教师由26.8%提高到31.64%,双师素质教师达到69.21%。建立了331人的兼职教师数据库;聘任兼职教师达到199人,兼职教师承担专业课时比例达到50%。生师比15.31:1。
通过名校建设,教师的职业教育理念得到提升,执教能力、科研能力和社会服务能力普遍提高,专业带头人能够准确把握专业发展,在行业企业产生了一定影响力,进入专家人才库,成为市科技特派员,为企业提供咨询与服务;骨干教师的理论水平、职业技能、课程开发能力得到明显提升;青年教师的教学水平进一步提高,参加技能大赛获得较好成绩。一支由名师带动、骨干支撑、专兼结合的高水平师资队伍已经形成,为完成教育教学和下一步深化改革打下坚实的基础。
表3-4  师资队伍建设标志性成果
序号
成果名称
1
通过引进、培养和聘用,不断优化师资队伍结构。学院引进了3名高层次人才,其中有2人具有博士研究生学历,有2人是泰安市拔尖人才。培养了1名省级名师,1个省级教学团队。调入、招聘了13名硕士研究生。
2
实施“名师递进培养工程”,按照“教学新星、教坛英才、教学名师、功勋教师”四个层次,递进培养校内专业带头人35人、骨干教师162人。
3
完善《专业带头人选拔培养管理办法》,重点专业实施了双带头人制度,培养校内专业带头人9人,从行业企业中引进9名熟悉相关专业的前沿技术、较好地把握专业发展方向、具有一定影响力的管理专家和技术骨干为校外专业带头人。
4
成立教师发展中心,健全了教师培训体系。继续实施教师素质提升计划,健全国家、省、市、校四级培训体系,三年内将所有专任教师轮训一遍,培训教师883人次。
5
实施访问工程师项目,完善专业教师每年利用寒暑假进行4周以上的生产实践锻炼。
6
先后邀请姜大源、马树超、匡奕珍、王汉忠、申培轩等35名国内知名专家来学院讲学、做专题报告。
7
以赛促教,鼓励教师参加各级各类讲课比赛、职业技能大赛,对每年参加技能大赛和质量工程建设获奖的集体和个人进行了表彰奖励,三年奖励金额超过180万元。
8
建立331人的兼职教师库,设立了100个兼职教师岗。
(四)教学改革成效突出
全院所有专业人才培养方案进行了优化;健全了“平台+模块”的课程体系,9个重点建设专业全部完成工作过程系统化的课程改革,所有专业全面实行“教学做一体化”教学模式,人才培养方案更加符合职业岗位需求和学生成长规律,以能力为本位的工作过程系统化课程改革更加符合技术型和高层次技能型人才培养要求,教学过程的实践性、开放性和职业性进一步展现。实践教学体系进一步完善,顶岗实习组织管理机构健全,顶岗实习管理制度规范,顶岗实习过程管理得到加强,学生顶岗实习的生活、安全、实习效果得到保障。
表3-5  教学改革标志性成果
序号
成果名称
1
修订完善了2013级、2014级、2015级专业人才培养方案;引进企业技术标准和职业资格标准,校企共同设计了基于工作过程的“平台+模块”的课程体系,实施了“分阶段、项目化、协同式”实践教学模式和“双证书”制度。
2
形成人才培养模式创新长效机制,制定了《关于进一步明确学院办学定位与人才培养模式的意见》(泰职院字〔2013〕77号)《关于实施“德技并重、理实一体”人才培养模式的意见(试行)》(泰职院字〔2014〕15号)。9个重点建设专业结合专业特点分别探索形成各具特色的人才培养模式。
3
实施课程体系优化工程。制定专业人才培养方案原则意见,校企共同修订人才培养方案,印制各专业人才培养方案合订本、专业课程体系合订本。
4
开展全员职教能力达标活动,推进整体课改和有效课堂教学改革。296名教师通过课改测评,并将课改成果应用到实际教学中,印制整体课程改革案例集。
5
加强教学研究,促进教学模式改革。2014年荣获国家教学成果奖二等奖1项,省级教学成果奖一等奖1项、三等奖3项。在2015年度全省职业教育教学改革项目立项中,我院共有7个项目获准立项,共获资助经费19万元,其中重点项目2项,一般项目5项。
6
建成省级精品课程7门,市级精品课程10门,院级精品课程59门,校企合作开发课程40门,自编校本教材(讲义)25门,公开出版教材38种。评选学院正式出版项目化课改教材28种,获省优秀教材一等奖1种,三等奖3种。
7
教师积极参加省级以上各类职业院校教学比赛,青年教师、信息化教学设计、信息化课堂教学、信息化实训教学、多媒体课件制作和微课等比赛,三年来获省级以上教师教学奖项总计24项,其中国家级二等奖1项,三等奖3项,省级一等奖3项,二等奖11项,三等奖6项。
8
和中唐方德共同开发应用顶岗实习信息管理平台,2016届毕业生全面启用了实习实训管理系统,实现对顶岗实习的全过程、即时性监督与管理。
9
将创新创业能力培养纳入人才培养方案,开设《职业生涯规划与就业指导》《大学生创新创业教育》等创新创业课程,举办创新创意比赛和第二课堂社团活动,搭建“大学生科技创新项目”平台,近三年,市级立项大学生科技创新项目20项,获得市级资金支持总计7.4万元;院级立项大学生科技创新项目共75项,院级资助资金6.5万元。
10
建设了卓越技师培养基地,每个专业都组建了专业核心技能训练队,实施“技能竞赛月”活动,各专业屡创佳绩。三年来,师生在省级以上技能大赛中,获奖297项,其中,全国一等奖9项、二等奖19项、三等奖45项,省特等奖、一等奖48项,二等奖71项、三等奖105项。
(五)体制机制进一步完善
1.多元化校企合作体制机制进一步完善。学院成立了校企合作理事会和校企合作委员会,校企合作委员会下设校企合作办公室。各系分别成立校企合作办公室。制定完善了校企合作系列制度,校企共同搭建人才培养平台、科技服务平台,实现学院与行业企业优势互补、资源共享,人才共育、成果共享,办学活力不断增强。
 2.内部管理进一步深化。按照“党委领导、校长负责、教授治学、民主管理”的原则,加快推进现代大学制度建设。制定了学院章程,建立健全了教职工代表大会制度,行政权力、学术权力、民主权力良性互动、有机协调、有效制衡、和谐高效的运行机制开始形成,依法治校、民主管理进一步深化。推行院系两级管理制度,重点改革了学校内部治理和组织框架,完善院系两级管理体制,建立和完善有利于激发办学活力的人、财、物有效管理运行机制,学校内部关系进一步优化,组织效率进一步提高;人事分配制度和考核激励机制进一步健全,在全市教育系统率先完成了专技人员晋级工作,充分调动起职工改革与发展的积极性和创造性。
3. 教学质量标准体系和评价体系健全,教育教学管理更加规范、高效。
构建了先进的教学质量监控与保障信息的管理平台;完善了教学管理制度,规范了教学行为;增强了教师的责任意识,提高了教师的教学水平;加大了教学改革力度,提高了人才培养质量。区域知名企业主动参与人才培养、监督评价,社会机构、用人单位对毕业生满意度提高,毕业生就业质量明显提高;各项教学工作得以持续改进,形成不断自我完善、保障质量提高的内生机制,社会、企业对学院人才培养工作高度认可。
4.“7S”管理全面实施,实现精细化管理。
创造了整洁的物质环境,提高了学生职业素养,树立了良好的学校形象,打造学校品牌形象,形成了良好的校风、学风。
三年来,学院积极探索合作办学体制,着力打造“人才共育、过程共管、成果共享、责任共担”的紧密型合作办学体制机制,“政行企所校”合作向纵深发展,建成了充满活力、富有效率、有利于科学发展的体制机制,最大限度地调动、发挥出学校、政府、企业和其他社会组织的积极性和创造性,增强了学校办学活力,提高了人才培养质量,满足了企业对高质量的技术技能型人才的需要。学院荣获2015年泰安市科学发展综合考核先进单位称号,为驻泰高校唯一获此殊荣的高校。
表3-6 体制机制标志性成果
序号
成果名称
1
市政府成立了市职业教育联席会;学院成立了校企合作理事会及委员会;系部成立了校企合作工作办公室。
2
学院牵头组建泰安市机电类专业职教集团,筹建泰安市现代农业、旅游职教集团;泰山职业技术学院东平移民学院成立。
3
泰安市创业大学在学院挂牌成立
4
泰安市第一公共实训中心在学院建立
5
泰安市泰山玉雕人才培养中心在学院成立
6
泰安市工艺美术学会在学院成立
7
学院首饰设计与工艺专业现代学徒制试点获2016年山东省职业院校现代学徒制试点项目
8
制定了学院《章程》
9
完善了《学院管理制度文件汇编》
10
制定了《学院十三五规划》等15个规划
11
实施了名师递进培养工程
12
制定并实施了“7S”管理系列制度
13
健全了“432”教学质量监控与保障体系
14
成为全国教学质量诊改试点院校
15
麦可思数据公司进行第三方评价
16
通过教育部人才培养工作评估、评估回访等
19
通过教育部人才培养工作评估、评估回访等
(六)招生就业更加畅通
根据不同生源特点,实行春季高考、夏季高考、单独招生、注册入学和技能考试、对口贯通培养、国际合作办学、订单班等多种形式,为各类学生提供多样化的成长成才路径,生源吸引力大大增强。招生形式的多样化,优质生源聚集,生源质量逐步提高,三年来,招生人数年年递增。名校建设已形成良好的职教品牌示范效应,扩大了学院的影响力,增强了社会吸引力。
以优质就业为导向,以充分就业为目标,建立学校、行业、社会三方共同参与的就业工作及评价机制。依托泰安市就业创业促进会学院工作站优势,搭建“校企战略、双向选择、优质企业推荐会、毕业生联谊会”等就业平台。每年举办两次校内大型就业双选会,参会企业累计达400家以上,提供就业岗位10000个,3000多学生达成就业意向。建立了毕业生信息库,完善毕业生就业信息跟踪和服务网络平台。建立创业大学,整合就业资源,全面开展就业指导和创业教育,开设就业创新课程,举办职业规划讲座及报告20场。以市创业孵化基地为依托,推动“以创业带动就业”战略,实现创业带动就业。建立创业项目库,已有不少于100个创业项目进入创业项目库。利用校友资源主动寻求就业机会,指导学生网上签约等方式,增加学生就业机会。就业指导体系不断完善,就业渠道更加宽广,学生就业率和就业质量明显提高,形成招生就业更加顺畅的良好局面。学院近三届毕业生就业率均在98%以上,全省就业率排名一直名列前茅,2013年就业率在134所有同类高校中排名第7。2015年就业率达到99.9%,在全省同类院校中排名第1。2013年被评为“全国职业院校就业竞争力示范校”。
表3-7 招生就业标志性成果
序号
成果名称
1
招生规模不断扩大,首次突破3000人,2015年三年制专科录取超计划31.9%,实际报到超计划14.9%,高职在校生8763人。
2
招生专业稳定在30个。录取分数线大幅提高,生源质量逐年提升,文理科最低分数线分别是292分、250分。文科最低分比资格线提高112分,理科提高70分。生源结构进一步巩固。
3
2013届—2015届毕业生就业率均在95%以上,全省就业率排名名列前茅。2013年就业率达98.39%,在专科毕业生中就业率排名第7位,在同类院校中排名第5位。2015届毕业生就业率达99.9%,在全省同类院校就业率排名第一。
4
依托“麦可思数据有限公司”对毕业生的社会需求与培养质量进行跟踪。根据调研情况统计,平均工资达到3334元,毕业生月收入较高的院系是汽车与电气工程技术系(4394元),毕业后月收入较低的院系是财经系(2543元)、旅游系(2592元)。收入高于同类院校就业于同行业类的收入。
5
培养13名国家高级职业指导师、32名国家中级职业指导师,15名国家创业咨询师,4名国家心理咨询师。大学生创业促进会创业导师5人,有创业实践的教职工40多人。
6
学院每年度组织一届创新创业大赛,对优秀获奖项目进行扶植、孵化。同时,积极组织师生参加“互联网+”大学生创新创业大赛、青年创业大赛、泰安创新创业大赛等各级各类创新创业大赛;每年组织一次职业规划大赛。
7
制定了《泰山职业技术学院就业工作考评暂行办法》,每年进行量化考核。
8
与企业共建实践教学体系,建设“数码雕刻工作室”、“遇石记e+创客空间”等多个工作室,形成了全真实践教学体系。在教师和企业师傅的共同指导下,深化形成产品,依托产品进行创业实践。在教育部主办的2015全国职业院校学生技术技能创新成果交流赛中仅工美系学生就获奖7项。
9
搭建创新创业实践和就业指导服务平台,建设运营泰安创业大学。学院与泰安市创业促进会合作,成立了“泰安市创业促进会泰山职技术学院工作站”,与泰安市人力资源和社会保障局(市人才交流服务中心)合作共建“大学生创业孵化培育中心”。
(七)国际交流与合作成效初显
通过名校建设,学院国际互动交流不断增加,国际化办学空间得到拓展。国际化办学的开展,丰富了学院多元化办学形式,加快了教育国际化步伐,取得了显著的效益。
1. 引进了境外教育机构的办学理念,拓宽了视野,更新了观念,确立了学院面向世界培养国际化人才的培养目标。
2.引进了境外的优质教育资源,促进了教学改革。通过合作办学,学院引进了国外优秀的原版教材,在专业建设、课程体系改革、教学内容更新、人才培养模式创新等方面带来了先进理念和经验, 加快了我院教材建设的更新速度,促进了学院的教育教学改革;同时,外籍教师先进的教学方法,通过案例教学、课堂讨论、课堂答辩、演示教学使学生接触和了解了境外教育机构的互动和启发式教学方法,丰富了学生的知识,拓宽了思路,提高了学生理论联系实际的能力。
3.加快了师资队伍的建设。学院选派骨干教师赴境外合作学校学习与进修,了解、学习和掌握国外先进的教学与管理理念与方法,加快了师资队伍建设。
4.引进了先进的管理模式。合作办学带来了国外先进的教学评估、课程评估和教学管理体系,促进了学院先进管理模式的借鉴,进一步加强教学管理,完善教学质量保障体系,使教学管理更加规范化、制度化。
5.合作办学项目适应了国际化人才培养形势和社会需求,促进了招生工作, 吸引了优质生源,推动了学院可持续发展。近年来学院招生持续火爆,中外合作办学项目的开展更进一步带动和促进相关专业的招生。
表3-8  国际交流与合作标志性成果
序号
成果名称
1
健全了国际交流合作委员会,加入中国教育国际交流协会,取得聘请外国专家资格,国际化办学取得显著成果。
2
建设友好学校7所:马来西亚成功礼待大学、台湾昆山科技大学、台湾朝阳科技大学、台湾万能科技大学、泰国斯巴顿大学、韩国世明大学、加拿大卡纳多文理应用学院。
3
完成合作办学3所:与澳大利亚启思蒙学院在会计、建筑与建造、机电一体化、电子工程等4个专业上签署合作办学,2014年9月实现招生。与泰国斯巴顿大学在工商管理、物流管理、航空服务等三个专业进行联合招生办学。与加拿大荷兰学院在会计电算化专业签订合作办学协议。现有中澳、中加班学生353名,学费收入180万元。
4
组织各类境外专家、教师培训17次,211名教师参加。其中境外培训师资84人;举办各类境外学术交流5次;与台湾昆山科技大学和台湾朝阳科技大学互派留学生56名,提升了广大教师的国际教育阅历和职业教育理念。
二、人才培养效益明显提高
通过名校建设,人才培养模式进一步完善,教育教学改革进一步深化,人才培养质量进一步提高,毕业生素养、知识、能力全面提升,持续发展能力显著增强,更加适应区域经济社会转型对技术技能型人才的要求。
(一)毕业生职业素养明显提高
通过人文素养课程和泰山特色校园文化建设,学生秉承“进取、担当、包容、和谐”的泰山精神,严守道德准则、道德情操,道德品质不断提高,具备良好的职业道德;通过推行“7S”管理,引入企业精细化管理理念,在教学管理中全面实施,学生时刻保持爱岗敬业、积极向上的职业作风;通过仿真企业实训、顶岗实习,培养学生主人翁意识,不断提升学生积极认真负责的工作态度,具备良好的职业意识,毕业生职业素养明显提高。通过第二课堂广泛开展社团活动、体育活动和创新活动,充分利用重大节庆日和纪念日,开展主题教育活动,突出职业特点,大力开展志愿者服务、暑期社会实践活动和“三下乡”活动。培养了学生良好的团队合作精神、协作能力,毕业生综合素质明显提高。全院共建设79个学生社团。成立20个志愿者服务队,学生参加志愿者服务1万多人次。每年327名学生参与封禅大典演出。
根据麦可思调查报告显示:本校2013-2015届工程类专业毕业生中,分别有62%、62%、63%的人认为大学帮助自己在“人生的乐观态度”、“乐于助人、参与公益”方面得到提升较多;92%的工程类毕业生表示在校期间素养存在提升,此类专业素养培养成效明显。本校毕业生创新能力连续三年总体满足度有上升趋势,比全国高职院校高2个百分点。

(二)毕业生核心知识水平明显提高
按照“德技并重、理实一体”的人才培养模式,依据区域经济社会发展需求,不断优化人才培养方案,通过对职业岗位能力的分析,重构课程体系,调整教学内容,扎实开展有效课堂建设,强化学生对理论常识、专业知识的掌握,毕业生理论知识水平明显提高。
(三)毕业生专业技术能力明显提高
按照校企合作、工学结合的要求和人才培养方案,学院不断强化学生实践技能培养,在各专业实施“分阶段、项目化、协同式”实践教学模式。以技能大赛为抓手,“以赛促练、以赛促改”,每个专业都组建了专业核心技能训练队,由企业技术能手和学校专业带头人、骨干教师重点训练,培养预备技师。三年来,师生在省级以上技能大赛中,获奖304项,其中,全国一等奖9项、二等奖19项、三等奖45项,省一等奖46项,二等奖71项、三等奖105项。通过麦可思调查,本校近三届毕业生的基本工作能力总体满足度呈上升趋势。

	•	基本工作能力培养效果变化趋势
表3-9  人才培养效益标志性成果
序号
成果名称
1
全院共建设79个学生社团
2
成立20个志愿者服务队,学生参加志愿者服务1万多人次
3
每年327名学生参与泰山封禅大典演出
4
根据麦可思调查报告显示:本校2013-2015届工程类专业毕业生中,分别有62%、62%、63%的人认为大学帮助自己在“人生的乐观态度”、“乐于助人、参与公益”方面得到提升较多;92%的工程类毕业生表示在校期间素养存在提升,此类专业素养培养成效明显。本校毕业生创新能力连续三年总体满足度有上升趋势,比全国高职院校高2个百分点
5
三年来,师生在省级以上技能大赛中,获奖304项,其中,全国一等奖9项、二等奖19项、三等奖45项,省一等奖46项,二等奖71项、三等奖105项
6
通过麦可思调查,本校近三届毕业生的基本工作能力总体满足度呈上升趋势
三、社会服务能力显著增强
学院立足泰安,辐射省会城市经济圈,为区域经济社会发展提供了强大的人才支撑、智力支持和技术服务;搭建起以社区文化服务、行业企业技术服务、各类人员培训服务和对口支援服务为重点的综合性公共服务平台,社会服务能力显著增强,对区域经济社会发展贡献度大大提高。
(一)为地方经济建设提供高素质人才支持
名校建设期间,学院为累计为地方区域经济社会发展培养了4561名高素质技术技能型人才,调查显示:学院大多数毕业生选择在省内就业(2013届-2015届分别为90.4%、84.2%、87.2%),毕业生就业量较大的为泰安、济南、潍坊等地区,为区域经济的发展贡献了较多专业人才。
(二)专业紧密对接地方优势产业
紧紧围绕省会城市群经济圈和泰安经济社会发展需要,根据泰安市十二五发展规划中泰安市重点发展的高压输变电设备、旅游文化、汽车、电子信息、金融、房地产等产业,不断调整优化专业布局结构,建成机电一体化、旅游管理、建筑工程等9个重点专业,形成专业建设紧密对接产业发展的专业建设思路,为地方经济发展作出突出贡献。
(三)提供多元化社会培训与服务
充分利用学院教育教学资源,面向社会开展横向课题研究、职业技能培训与鉴定、技术研发、社会考试等各种社会服务。为社会提供各类培训1052622人时,面向区域内行业企业开展职业资格鉴定,年均3000人次,技术服务能力显著增强,为400余家企业提供培训服务和技术支持,年均科技服务到款额500万元以上。
表3-10  社会服务标志性成果
序号
成果名称
1
制定了《泰山职业技术学院科研工作管理办法》《泰山职业技术学院科研成果奖励办法》《泰山职业技术学院科研工作量化考核办法》《泰山职业技术学院教科研经费资助办法》《泰山职业技术学院科技特派员管理办法》等制度。
2
成立了“泰安市现代职业教育研究院”“泰安市工艺美术学会”等研究机构,建设了软件开发、建筑工程质量检测、汽车检测等技术研发与服务中心。
3
牵头组织了省市旅游产业对经济、财政和就业贡献的研究,全域旅游开展、乡村旅游开发研究;参与了市旅游总体规划、市旅游营销策划的设计、论证、评审等,共计34个项目。为各级各部门建言献策152条,其中进入领导决策135条。
4
成人教育、远程教育、自考等学历教育,年均招生规模1000人。
5
发挥社会培训与职业技能鉴定平台作用,多渠道、多层次经济服务社会发展。依托学院职业技能鉴定站(点),完成技能鉴定10309人。成立驾校并完成培训约2800人、旅行社成立并正常运营。
6
政行企所校深度合作,开展各类培训。完成各类社会培训58479人次,年均折合34万人时,完成社会服务收入2128.68万元。
7
积极承担全市退役士兵职业与技能培训500人,政府采购代理机构人员培训300人,全市基层农技人员培训305人,农村劳动力阳光培训1200人。
8
承担汶上职业中专结对帮扶任务,完成市域内6所职业学校结对帮扶工作。
9
牵头组织泰安市“3311”品牌项目,学院的泰山名师和部分专业负责人参加全市名师送课活动。
10
与政府职能部门联动,每年承担包括全省春季高考、公务员、研究生等各级各类社会考试18项,服务考生30多万人次。
11
承担山东省援疆任务,接收培训新疆喀什地区岳普湖县的大学毕业生培养工作,累计三批共80余人;2名教师赴新疆喀什岳普湖县支教
12
选派3名党员干部到新泰、东平农村基层党组织任第一书记。
13
助力旅游发展,我院327名学生承担泰山封禅大典群众演员演出活动。
四、示范带动能力进一步增强
通过对口支援、对外交流、社会培训等,学院在办学特色、人才培养改革、校园文化建设、社会服务等方面对区域内外其它院校产生辐射带动作用。
(一)以点带面,带动其它专业共同发展
通过名校建设,各重点建设专业在体制机制创新、“德技并重、理实一体”人才培养模式改革、师资队伍建设、课程建设与教学改革、社会服务能力建设等方面形成了各自的特色与优势,带动相关专业群共同发展。以9个重点建设专业为龙头,辐射带动其它21个专业共同发展。借鉴重点建设专业经验,对专业群内专业人才定位、培养目标、人才培养质量标准、人才培养方案等方面进行了优化,实现了专业群内师资、实习实训条件、课程资源共享。通过建设,形成了以优势特色专业为龙头、相关专业为支撑的专业集群优势、品牌优势,促进了专业群与区域产业体系对接的良性互动,使学院专业综合实力得到明显提升。在各重点专业的辐射下,带动专业21个,惠及 余名学生。各专业新增校企合作企业 家。建成专业优质核心课程53门、校企合作开发教材78种,专业教学资源库30个,新建改建扩建实验实训室70个,教学型工厂3个,新增省级精品课程7门,省级教学团队1个,省级教学名师1人。1个专业列为山东省现代学徒制试点,通过重点专业的带动,有力促进了专业内涵建设水平整体提升。
(二)对口支援,成果辐射其它职业院校
充分利用学校优质教育教学资源和名校建设成果,对口支援汶上职业中专等7所中等职业学校,学院多次派管理干部和专业带头人到这些学校,对其专业建设、师资队伍建设、课程建设及教学改革进行指导。
名校建设经验被其它多所职业院校学习和借鉴。建设期间,先后有厦门技师学院、山东城建职业学院等50多所学院前来考察学习交流。学院先进办学经验在中国教育报、中国青年报、中国职业教育杂志、大众日报等媒体宣传报道。
(三)承办大赛,搭建技能交流平台
学院承办了“全省大学生小空间生态设计等比赛和泰安市“泰航杯”全市技能人才职业技能大赛等比赛。在承办大赛过程中,将比赛中最新行业标准和企业技术发展水平,以及新理念、新技术、新工艺、新流程引入到教学中,推进了学院课程改革和工学结合实践教学体系的重构。通过承办大赛,促进了院校、企业、行业的交流,为职业教育发展做出了贡献。
通过山东省技能型特色名校建设,学院在发展、管理、改革和现代职教体系构建等方面取得一系列标志性成果,形成办学特色、人才培养模式等方面的先进经验,引领同行整体管理水平和办学水平的提高,学院的名校建设示范带动作用进一步增强。
表3-11示范带动能力标志性成果
序号
成果名称
1
重点专业辐射带动专业21个共同发展
2
厦门技师学院、山东城建职业学院等50多所学院前来考察学习交流
3
承办“全省大学生小空间生态设计等比赛、泰安市“泰航杯”全市技能人才职业技能大赛、泰安市职业院校技能大赛等比赛
4
学院先进办学经验在中国教育报、中国青年报、中国职业教育杂志、大众日报等媒体宣传报道










第四部分 专项资金使用与管理情况
一、专项资金到位情况
泰安市政府和主管部门泰安市教育局全力以赴支持学院建设省级特色名校,在特色名校建设过程中保证建设资金及时足额到位,学院也多方面积极筹集资金,保证了建设资金的按期到位。项目建设预算资金5540万元,建设期间共投入资金11841万元,投入资金占预算资金的214 %。其中山东省财政厅拨款1000万元,到位率100%;泰安市财政局拨入名校建设专项资金3000万元,到位率100%;行业企业投入到位资金1262万元,到位率252%;学院自筹资金2762万元,到位率266%。此外,学院自筹资金3817万元,建设一栋19191平方米的实训一体化大楼和新接建4栋8466平方米学生公寓,极大的提高了学生的实训实习条件和生活条件,为学院的特色名校的建设打下了坚实的基础。
表4-1学院特色名校建设资金到位情况一览表
                                                 单位:万元
资金来源
总预算
年度到位资金
到位总资金
资金到位率(%)


2013年
2014年
2015年
2016年


省财政
1000
300
300
400

1000
100
市财政
3000

600
2400

3000
100
行业投入
500
248
458
210
346
1262
252
学院自筹
1040
244
2084
3220
1031
6579
632
合计
5540
792
3442
6230
1377
11841
145
二、专项资金的执行情况
(一)专项资金的执行情况
按照《山东省高等教育名校建设工程专项资金管理办法》(鲁财教〔2012〕84号)和教育省厅论证通过的《建设方案》《建设任务书》以及《泰山职业技术学院省级特色名校建设项目管理办法》要求,学院严格规范使用专项资金。
表4-2学院特色名校建设资金执行情况一览表
                                                     单位:万元
支出项目
预算投入
实际执行

总额
省财政
市财政
行业投入
学院自筹
总额
省财政
市财政
行业投入
学院自筹
省财政重点支出专业
3378.19
1000
1901
300
177.19
4443
1000
1901
388
1154
非省财政重点支出专业
1151.2

430
200
521.2
2010

430
607
973
其他建设项目
1010.48

669

341.48
1304

669

635
总额
5540
1000
3000
500
1040
7757
1000
3000
995
2762
执行率%

140
100
100
199
266
截止目前,名校建设预算总资金5540万元,其中省财政支持资金1000万元,市财政支持资金3000万元,行业企业投入500万元,学院自筹1040万元。完成建设资金7757万元,省财政支持资金1000万元,市财政支持资金3000万元,行业企业投入995万元,学院自筹2762万元,完成率140%。
(二)子项目资金使用情况
1.机电一体化技术专业建设资金使用情况
机电一体化技术专业建设预算总资金643万元,其中省财政支持资170万元,市财政支持资金380万元,行业企业投入50万元,学院自筹43万元。完成建设资金791万元,其中省财政支持资金170万元,市财政支持资金380万元,行业企业投入50万元,学院自筹191万元,完成率123%。
表4-3机电一体化技术专业建设资金执行情况一览表
单位:万元
支出项目
预算投入
实际执行

总额
省
财政
市
财政
行业企业
学院自筹
总额
省
财政
市
财政
行业投入
学院自筹
体制机制建设
18
16


2
18
16


2
人才培养模式与培养
方案
19
15


4
19
15


4
课程体系构建与核心
课程建设
74
63


11
80
63


17
教学团队建设
102
7
80

15
144
7
80

57
实训条件建设
253.82

200
50
3.82
289

200
50
39
社会服务能力建设
20
17


3
20
17


3
专业群建设
147
43
100

4
210
43
100

67
其 它
9.18
9


0.18
11
9


2
总    额
643
170
380
50
43
791
170
380
50
191
执行率%

123
100
100
100
445
2.建筑工程技术专业建设资金使用情况
建筑工程技术专业建设预算总资金530万元,其中省财政支持资金170万元,市财政支持资金310万元,行业企业投入50万元。完成建设资金623万元,其中省财政支持资金170万元,市财政支持资金310万元,行业企业投入55万元,学院自筹88万元,完成率118%。
表4-4建筑工程技术专业建设资金执行情况一览表
单位:万元
支出项目
预算投入
实际执行

总额
省
财政
市
财政
行业
投入
学院自筹
总额
省
财政
市
财政
行业投入
学院自筹
体制机制建设
18

18


22

18

4
人才培养模式与培养方案
26.15

26.15


27

26.15

1
课程体系构建与核心课程建设
91.83
83
8.83


92
83
8.83


教学团队建设
110
87
13
10

130
87
13
10
20
实训条件建设
141.02

141.02


200

141.02

59
社会服务能力建设
13


13

17


14
3
专业群建设
125

103
22

127

103
23
1
其 它
5


5

8


8

总    额
530
170
310
50

623
170
310
55
88
执行率%

118
100
100
110

3.会计电算化专业建设资金使用情况
会计电算化专业建设预算总资金627.8万元,其中省财政支持资金170万元,市财政支持资金350万元,行业企业投入50万元,学院自筹57.8万元。完成建设资金1114万元,其中省财政支持资金170万元,市财政支持资金350万元,行业企业投入63万元,学院自筹531万元,完成率177%。
表4-5会计电算化专业建设资金执行情况一览表
单位:万元
支出项目
预算投入
实际执行

总额
省财政
市财政
行业投入
学院自筹
总额
省财政
市财政
行业投入
学院自筹
体制机制建设
19.5
6
7.5
6.0

30
6
7.50
8
9
人才培养模式与培养
方案
9
6
3


11
6
3

2
课程体系构建与核心
课程建设
106.3
48
28.5
18.0
11.8
144
48
28.5
24
43
教学团队建设
138.2
84
33
18.0
3.2
185
84
33
22
46
实训条件建设
220.4

170
8.0
42.4
550

170
9
371
社会服务能力建设
11
6
5


22
6
5

11
专业群建设
110
10
100


157
10
100

47
其 它
13.4
10
3

0.4
15
10
3

2
总    额
627.8
170
350
50
57.8
1114
170
350
63
531
执行率%

177
100
100
126
919
4.旅游管理专业建设资金使用情况
旅游管理专业建设预算总资金530万元,其中省财政支持资金  170万元,市财政支持资金271万元,行业企业投入50万元,学院自筹39万元。完成建设资金664万元,其中省财政支持资金170万元,市财政支持资金271万元,行业企业投入70万元,学院自筹153万元,完成率125%。
表4-6旅游管理专业建设资金执行情况一览表
单位:万元
支出项目
预算投入
实际执行

总额
省财政
市财政
行业投入
学院自筹
总额
省财政
市财政
行业投入
学院自筹
体制机制建设
19
15
4


20
15
4

1
人才培养模式与培养
方案
10
10



12
10


2
课程体系构建与核心
课程建设
81
65
10
6

93
65
10
6
12
教学团队建设
89
45
26
14
4
123
45
26
14
38
实训条件建设
145

104
8
33
210
0
104
8
98
社会服务能力建设
25
15
4
5
1
26
15
4
5
2
专业群建设
151
20
113
17
1
170
20
113
37

其 它
10

10


10

10


总    额
530
170
271
50
39
664
170
271
70
153
执行率%

125
100
100
140
393
5.计算机应用技术专业建设资金使用情况
计算机应用技术专业建设预算总资金530.2万元,其中省财政支持资金160万元,市财政支持资金310万元,行业企业投入50万元,学院自筹10.2万元。完成建设资金634万元,其中省财政支持资金160万元,市财政支持资金310万元,行业企业投入94万元,学院自筹70万元,完成率120%。
表4-7计算机应用技术专业建设资金执行情况一览表
单位:万元
支出项目
预算投入
实际执行

总额
省财政
市财政
行业投入
学院自筹
总额
省财政
市财政
行业投入
学院自筹
体制机制建设
29
12
6
11

34
12
6
11
5
人才培养模式与培养
方案
8
6
2


9
6
2

1
课程体系构建与核心
课程建设
71.2
46
10
13
2.2
99
46
10
20
23
教学团队建设
105
72
16
17

131
72
16
17
26
实训条件建设
187

170
9
8
188

170
9
9
社会服务能力建设
18
12
6


55
12
6
37

专业群建设
110
10
100


114
10
100

4
其 它
2
2



4
2


2
总    额
530.2
160
310
50
10.2
634
160
310
94
70
执行率%

120
100
100
188
686
6.园艺技术专业建设资金使用情况
园艺技术专业建设预算总资金517.19万元,其中省财政支持资金160万元,市财政支持资金280万元,行业企业投入50万元,学院自筹27.19万元。完成建设资金617万元,其中省财政支持资金160万元,市财政支持资金280万元,行业企业投入56万元,学院自筹121万元,完成率120%。
表4-8园艺技术专业建设资金执行情况一览表
单位:万元
支出项目
预算投入
实际执行

总额
省财政
市财政
行业投入
学院自筹
总额
省财政
市财政
行业投入
学院自筹
体制机制建设
16
8
4
4

18
8
4
4
2
人才培养模式与培养
方案
16
8
4
4

21
8
4
4
5
课程体系构建与核心
课程建设
85.46
50
20
6
9.46
87
50
20
6
11
教学团队建设
89.73
82

6
1.73
106
82

6
18
实训条件建设
188

152
20
16
208

152
26
30
社会服务能力建设
18
12

6

46
12
0
6
28
专业群建设
104

100
4

131

100
4
27
其 它










总    额
517.19
160
280
50
27.19
617
160
280
56
121
执行率%

120
100
100
112
445
7.汽车电子技术专业建设资金使用情况
汽车电子技术专业建设预算总资金527万元,其中市财政支持资金215万元,行业企业投入100万元,学院自筹212万元。完成建设资金790万元,其中市财政支持资金215万元,行业企业投入184万元,学院自筹391万元,完成率150%。
表4-9汽车电子技术专业建设资金执行情况一览表
单位:万元
支出项目
预算投入
实际执行

总额
省
财政
市
财政
行业投入
学院自筹
总额
省
财政
市
财政
行业投入
学院自筹
体制机制建设
12

12


15

12

3
人才培养模式与培养
方案
28

28


30

28

2
课程体系构建与核心
课程建设
70

60
10

92

60
13
19
教学团队建设
80

70
10

100

70
13
17
实训条件建设
272


60
212
449


108
341
社会服务能力建设
24

14
10

25

14
11

专业群建设
36

26
10

73

26
39
8
其 它
5

5


6

5

1
总    额
527

215
100
212
790

215
184
391
执行率%

150

100
184
185
8.服装设计专业建设资金使用情况
服装设计专业建设预算总资金303万元,其中市财政支持资金 110万元,行业企业投入50万元,学院自筹143万元。完成建设资金488万元,其中市财政支持资金110万元,行业企业投入61万元,学院自筹317万元,完成率161%。
表4-10    服装设计专业建设资金执行情况一览表
单位:万元
支出项目
预算投入
实际执行

总额
省
财政
市
财政
行业投入
学院自筹
总额
省
财政
市
财政
行业投入
学院自筹
体制机制建设
9

5
4

28

5

23
人才培养模式与培养
方案
18

10
8

19

10

9
课程体系构建与核心
课程建设
49

44
2
3
130

44
58
28
教学团队建设
46

33
13

67

33

34
实训条件建设
157


17
140
202


3
199
社会服务能力建设
8

8


14

8

6
专业群建设
16

10
6

28

10

18
其 它










总    额
303

110
50
143
488

110
61
317
执行率%

161

100
122
221
9.珠宝首饰工艺及鉴定专业建设资金使用情况
珠宝首饰工艺及鉴定专业建设预算总资金321.2万元,其中市财政支持资金105万元,行业企业投入50万元,学院自筹166.2万元。完成建设资金732万元,其中市财政支持资金105万元,行业企业投入362万元,学院自筹265万元,完成率228%。
表4-11珠宝首饰工艺及鉴定专业建设资金执行情况一览表
单位:万元
支出项目
预算投入
实际执行

总额
省财政
市财政
行业投入
学院自筹
总额
省财政
市财政
行业投入
学院自筹
体制机制建设
2

1
1

6

1

5
人才培养模式与培养
方案
17

10
1
6
19

10

9
课程体系构建与核心
课程建设
18.2

14
2
2.2
23

14

9
教学团队建设
30

20
10

40

20

20
实训条件建设
244

50
36
158
627

50
362
215
社会服务能力建设
1

1


3

1

2
专业群建设
7

7


10

7

3
其 它
2

2


4

2

2
总    额
321.2

105
50
166.2
732

105
362
265
执行率%

228

100
724
159
10.校企合作体制机制改革建设项目建设资金使用情况
校企合作体制机制改革建设项目预算总资金50万元,其中市财政支持资金50万元。完成建设资金52万元,其中市财政支持资金50万元,学院自筹2万元,完成率104%。
表4-12 校企合作体制机制改革建设项目资金执行情况一览表
单位:万元
支出项目
预算投入
实际执行

总额
省财政
市财政
行业投入
学院自筹
总额
省财政
市财政
行业投入
学院自筹
健全校企合作组织,完善校企合作制度
7

7


7

7


拓展校企合作空间,促进校企深度融合
13

13


15

13

2
完善“厂校一体,资源共享”的校企合作模式
30

30


30

30


总    额
50

50


52

50

2
执行率%

104

100


11.教学质量监控与保障体系建设项目资金使用情况
教学质量监控与保障体系建设项目预算总资金70万元,其中市财政支持资金40万元,学院自筹30万元。完成建设资金186万元,其中市财政支持资金40万元,学院自筹146万元,完成率266%。
表4-13教学质量监控与保障体系建设项目资金执行情况一览表
单位:万元
支出项目
预算投入
实际执行

总额
省财政
市财政
行业投入
学院自筹
总额
省财政
市财政
行业投入
学院自筹
监控主体
10

7

3
15 

7

8
教学质量监控与保障的标准体系
11

6

5
22

6

16
教学质量监控与评价体系
10

6

4
61

6

55
完善人才培养工作数据采集平台
2

1

1
4

1

3
教学管理信息平台
27

14

13
30

14

16
教学质量反馈与调控体系
10

6

4
54

6

48
总    额
70

40

30
186

40

146
执行率%

266

100

487
12.泰山文化为特色的校园文化建设项目资金使用情况
泰山文化为特色的校园文化建设项目预算总资金210万元,其中市财政支持资金150万元,学院自筹60万元。完成建设资金283万元,市财政支持资金150万元,学院自筹133万元,完成率135%。
表4-14泰山文化为特色的校园文化建设项目资金执行情况一览表
单位:万元
支出项目
预算投入
实际执行

总额
省财政
市财政
行业投入
学院自筹
总额
省财政
市财政
行业投入
学院自筹
泰山书院建设
90

30

60
98 

30

68
环境文化建设
60

60


73 

60

13
制度文化建设
10

10


15 

10

5
行为文化建设
30

30


46 

30

16
精神文化建设
20

20


51 

20

31
总    额
210

150

60
283

150

133
执行率%

135

100

218
13.数字化校园建设项目资金使用情况
数字化校园建设项目预算总资金680.48万元,其中市财政支持资金429万元,学院自筹251.48万元。完成建设资金783万元,市财政支持资金429万元,学院自筹344万元,完成率114%。
表4-15泰山文化为特色的校园文化建设项目资金执行情况一览表
单位:万元
支出项目
预算投入
实际执行

总额
省财政
市财政
行业投入
学院自筹
总额
省财政
市财政
行业投入
学院自筹
校园网软硬件平台建设
380

380


388 

380

8
教学资源库平台建设
77



77
93 



93
校园一卡通平台拓展
80



80
90 



90
信息化管理平台建设
113

30

83
167 

30

137
文献信息服务平台建设
30.48

19

11.48
45 

19

26
总    额
680.48

429

251.48
783

429

354
执行率%

114

100

141
三、专项资金管理情况
根据国家有关法律法规和《山东省高等教育名校建设工程专项资金管理办法》(鲁财教〔2012〕84号)的规定,结合教育厅和财政厅的相关要求,学院严格规范使用专项资金。专项资金的管理坚持“集中使用,突出重点,总体规划,分年实施,项目管理,绩效考评”的原则,建立项目资金管理责任制。建设项目资金纳入学院总体预算,按照专账核算、专款专用的管理模式,严格项目管理,规范和加强对名校建设项目专项资金的管理,凡纳入政府采购目录的支出项目,严格按照规定实行政府采购,凡建设资金形成的资产一律纳入学院的资产管理系统统一管理。名校建设期间,任何项目和系部不得截留、挪用和挤占名校建设资金,保证项目顺利实施,提高了建设资金使用效益。
(一)健全机构,细化责任
学院成立特色名校建设工作领导小组作为专项资金的管理机构,负责对专项资金进行总体规划和分配,名校建设工作小组下设专项资金管理组是专项资金的具体管理部门,按计划落实有关上级部门的支持资金,确保项目专项资金落实到位。做好建设项目预算的审核,参与年度建设项目或单项项目的验收,负责项目的结算、统计与审计工作,负责项目建设的财务管理、资金使用的监督,定期向项目建设领导小组汇报资金的使用情况,各重点建设项目单位是专项资金的具体使用部门,负责按照预算内容和经费权限,对各项目建设资金的运用和效益分析等实行全过程管理,学院纪检部门负责对专项资金使用的全过程进行监督和检查。
(二)建章立制、规范程序
以《山东省教育厅  山东省财政厅关于山东省高等教育名校实施意见》(鲁教高字〔2011〕14号)等文件为依据,结合学院实际,特出台了《泰山职业技术学院省级特色名校建设项目专项资金管理办法》等规章制度,各建设项目组、职能组严格按照专项资金使用与管理办法使用专项资金。
在使用过程中,对专项资金的适应范围,管理权限,审批以及报销制定了详细的操作细则,分别制定了《名校项目建设专项资金审批流程》《名校项目建设专项资金报销流程》等工作标准、使用规范。
(三)指导执行,过程监控
为规范名校建设专项资金的管理,学院财务处对名校建设专项资金管理进行全过程业务指导和监督,及时分析和反馈专项资金预算执行情况、资金使用效益以及在项目建设过程中遇到的困难和问题;借助信息技术手段,强化过程监控;建立定期“职能组-子项目”对接机制,资金管理组每月与各建设项目资金子项目负责人核对资金支出总额,确保资金支出准确,控制资金使用进度,并在名校建设调度会上进行通报。
四、专项资金使用效果
(一)省财政专项资金引导带动作用明显
根据 《山东省教育厅  山东省财政厅关于山东省高等教育名校建设工程实施意见》(鲁教高字〔2011〕14号)文件精神,学院名校建设遵循“整体设计、重点建设、示范带动、全面推进”的建设原则,积极争取泰安市财政和市教育局的支持,多渠道筹集建设资金,确保名校建设资金顺利到位。项目建设期间,省财政投入专项资金1000万元,共计带动市级财政投入3000万元,行业企业投入1262万元、学院自筹资金6579万元,其中包含学院自筹资金3817万元,建设一栋19191平方米的实训一体化大楼和新接建4栋8466平方米学生公寓,全面落实对省教育厅、省财政厅的承诺,超出建设方案6301万元,省财政资金的引导和带动作用明显,达到了建设山东省特色名校的目的。
(二)学院办学实力进一步提升
通过省财政、市财政、行业企业、学院自筹等多渠道筹集建设资金,学院在三年建设期内严格项目管理运行。在体制机制创新、人才培养模式改革、课程体系构建、教学团队建设、实训条件改善、社会服务能力提升等方面取得了显著成效。通过三年的建设,学院资产总额由建设初期的4.1亿元增加到现在的4.64亿元,增长了0.54亿元,增幅达13%。其中,教学科研设备总值由建设初期的4966万元,增长到目前的7907万元,增加了2941万元,增长幅度为59%。同时,新增一栋19191平方米的实训中心,新接建4栋8466平方米的学生公寓,学生的实训、实习条件和生活条件得到了显著的改善,通过省级特色名校的建设,学院的可持续发展能力大幅增强,为今后的快速发展打下了坚实的基础。


第五部分 思考与展望

建设期内,学院恰逢中国职教发展的黄金三年,国家围绕现代职业教育体系加快构建,召开全国职教会议、出台系列文件,基本完成了职业教育的顶层设计,职业院校发展迎来新契机。正是全新的时代机遇和无限大的政策红利,让泰职依托省特色名校建设平台,深度思考、锐意进取、积极实践,用发展的新常态开启了一段新历程。省特色名校建设任务全面完成,取得重要成就,为今后的发展奠定了坚实基础。相对于有形的物化成果而言,境界提升、理念革新,意义更显深远。
全面总结名校建设的经验与不足,立足“后特色名校”建设的新起点,胸怀使命、勇于创新、顺势而为,积极培育发展新动能,精准发力供给侧改革,谋求内涵建设新突破,这应是名校建设带给我们的重要启示。
放眼“十三五”或更长一段时间里,学院将牢固树立和自觉践行五大发展理念,突出办学“服务”价值趋向,以创建全国优质专科高等职业院校为目标,以全国教学工作诊断与改进试点校建设为抓手,创新办学体制机制,深化教育教学改革,推动内涵发展,积极培育办学特色,全面提高管理水平和办学质量、效益。为此,学院发展要在这些方面进一步聚焦、发力和提升。
一、突出办学导向,服务国家(区域)战略和创新人才培养
进一步突出服务价值取向,主动顺应“中国制造2025”“大众创业 万众创新”“互联网+”、山东“黄蓝两大国家级战略”“一圈一带”和泰安建设山东省旅游文化产业高地等战略,在服务经济社会发展中,提升办学的内涵,彰显大学价值。要将培养学生的创新意识和创新思维融入教育教学全过程,促进专业教育与创新创业教育有机融合,以发掘学生潜能、专业兴趣和支持学业自主选择为目标,优化创新创育环境,培育学生创新创业意识和能力,研究建立相关教育和管理机制。
二、深化产教融合,加快推进校企一体、协同育人
加强技术技能积累,与企业深度合作,加强应用技术的传承应用研发能力,提高技术服务的附加值;加强民族文化、民间技艺的传承发展和人才培养,努力发展成为国家技术技能积累与创新的资源集聚地。继续完善名校建设所形成的校企合作机制,探索创新校企合作办学、合作育人、合作发展的协调机制,继续深化“德技并重、理实一体”的人才培养模式改革,积极探索和实践现代学徒制等校企联合培养规模,拓展校企协同育人的深度和广度,全面提升办学水平和培养质量;发挥学院资源优势,向社区开放服务,推进学习型社会建设。
三、全力打造品牌,进一步提高专业服务产业能力
着眼“中国制造2025”和“互联网+”时代的到来,紧跟产业转型升级,进一步形成动态调整专业结构的机制,集中力量办好区域经济社会发展需要的特色优势专业(群)。及时总结特色名校建设9个重点专业的建设经验,结合新一轮高校品牌专业建设工程的内涵要求,建立与国家接轨的专业教学标准和管理评价体系;以学院被列入教学诊断与改进工作试点校为契机,加强质量保证能力建设。以品牌专业建设为引领,以各专业的龙头专业为重点,建设一批特色鲜明、省内领先、全国一流的专业。以高质量的人才培养和技术呼应企业发展需求,形成专业办学品牌。
四、强化立德树人,切实提升学生发展核心竞争力
实施“贴心工程”升级版。深度开发泰山书院,完善“学院+书院”育人模式,建立书院导师制度和学生活动管理制度,推行书院式管理模式,以书院式教学、书院式公寓管理为核心,缔造和谐的师生关系和育人环境,建设书院式校园。关注学生个性化成长,多样化的发展需求,为学生立业成才提供更多路径和教育支撑。进一步完善综合素质教育体系,夯实学生发展基础;以“工匠”之心育“工匠”之才,提升教师教育教学理念和水平,强化工学结合,改进教学模式,增强学生实力和学习能力。实施大学文化建设提升和品牌塑造工程,深入挖掘泰山文化中的育人要素,实施泰山文化进课堂。深入推行“7S”管理,提高学生的职业核心能力。突出产教融合,打造特色鲜明的专业文化。
宏图已绘就,扬帆再出发。有三年省名校建设夯实的厚基础,有泰职师生对发展愿景的新期许和图强、跨越、赶超的真干劲,未来的泰职,定会拥有崭新格局;一个办学实力更强、人才培养质量更优、服务发展水平更高的泰职,在争创全国优质专科高职院校道路上定会再创新局,奏响特色名校发展的时代强音。



\message{ !name(总结报告.tex) !offset(-9647) }
